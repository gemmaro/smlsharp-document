\chapter{\txt
{型無しパターン言語:\code{patterncalc}}
{The untyped pattern language : \code{patterncalc}}
}
\label{chap:PatternCalc}

\ifjp%>>>>>>>>>>>>>>>>>>>>>>>>>>>>>>>>>>>>>>>>>>>>>>>>>>>>>>>>>>>>>>>>>>>
\begin{enumerate}
\item ソースロケーションとファイル構成
ディレクトリ \code{src/compiler/patterncalc/main}下の以下のファイルからなる.
\begin{itemize}
\item \code{PatternCalc.ppg} 型無しパターン言語のソース言語定義ファイル
\item \code{PatternCalcInterface.ppg} 型無しパターン言語のインターフェイス言語定義ファイル
\end{itemize}

\item 他モジュールとのインターフェイス
\begin{itemize}
\item 構文論的評価モジュール\code{src/compiler/elaborater}の出力言語.
\item 名前評価・モジュールコンパイルモジュール\code{src/compiler/nameevaluation}の入力言語.
\end{itemize}
\end{enumerate}
\else%%%%%%%%%%%%%%%%%%%%%%%%%%%%%%%%%%%%%%%%%%%%%%%%%%%%%%%%%%%%%%%%%%%%
\fi%%%%<<<<<<<<<<<<<<<<<<<<<<<<<<<<<<<<<<<<<<<<<<<<<<<<<<<<<<<<<<<<<<<<<<

\section{\txt{中間言語定義の詳細}{The details of the language definitions}}
\ifjp%>>>>>>>>>>>>>>>>>>>>>>>>>>>>>>>>>>>>>>>>>>>>>>>>>>>>>>>>>>>>>>>>>>>
	抽象構文木(\code{absyn}モジュール)との主な違いは,以下3点である.
\begin{enumerate}
\item 2項演算子がコンパイルされ消去されている
\item \code{\_sqlserver}文を除くSQL構文がコンパイルされ消去されている
\item ユーザの型指定に現れる型変数のスコープがすべて明示されている
\end{enumerate}

\subsection{\code{PatternCalc.ppg}の型定義}
	\code{PatterCalc}ストラクチャの中に以下の型が定義される.

\begin{tabular}{ll}
\code{caseKind} & パタンマッチングの種別(エラープリント用)
\\
\code{plexbind} & 例外定義と例外エイリアス定義
\\
\code{ffiTy} & C関数の型指定
\\
\code{ffiArg} & C関数の引数型.\code{PLFFIARGSIZEOF}は,型のサイズを表す
単元型.(レコード計算のインデックス型の一種)
\\
\code{plexp} & 核言語の式
\\
\code{pdecl} & 核言語の宣言
\\
\code{plpat} & 核言語のパターン
\\
\code{plstrdec} & モジュール宣言
\\
\code{plstrexp} & モジュール式
\\
\code{plsigexp} & シグネチャ式
\\
\code{plspec} & シグネチャ式の要素
\\
\code{pltopdec} & トップレベル構文
\end{tabular}

\subsection{\code{PatternCalcInterface.ppg}の型定義}
	以下の型が定義される.

\begin{tabular}{ll}
\code{runtimeTy} & 実行時の型.
\\
\code{pidec} & インターフェイス宣言
\\
\code{pistrexp} & ストラクチャ
\\
\code{funbind} & ファンクタ宣言
\\
\code{pitopdec} & トップレベル宣言
\\
\code{interfaceDec} & インターフェイス定義
\\
\code{interface} & インターフェイス
\\
\code{compileUnit} & コンパイル単位
\\
\code{topdec} & \code{PatternCalc}のトップレベル宣言中間表現
\\
\code{source} & コンパイル単位の中間表現
\end{tabular}

注釈
\begin{itemize}
\item \code{runtimeTy}は,\code{type foo (= bar)}の形の不透明な型定義
宣言の\code{bar}を解釈してえられる実行時の型であり,以下のように定義される.
\begin{programPlain}
datatype runtimeTy = BUILTINty of BuiltinTypeNames.bty | LIFTEDty of longsymbol
\end{programPlain}
\code{LIFTEDty}は,この不透明な型定義がファンクタの本体に現れる場合でか
つ\code{bar}がそのファンクタの引数で\code{type bar}で宣言されている場合
である.

\item \code{compileUnit}は,構文論的評価(\code{elaboration}モジュール)の出力言語であり,
かつ名前評価・モジュールコンパイル(\code{nameevaluation}モジュール)の
入力である.

\item 
	\code{source}は,\code{nameevaluation}モジュールの
\code{SpliceFunProvide}で変換される中間的な表現である.
\item \code{topdec}は,\code{SpliceFunProvide}の変換結果である
\code{source}を定義するための\code{PatternCalc}のトップレベル宣言型の表
現である.
\end{itemize}

\else%%%%%%%%%%%%%%%%%%%%%%%%%%%%%%%%%%%%%%%%%%%%%%%%%%%%%%%%%%%%%%%%%%%%
\fi%%%%<<<<<<<<<<<<<<<<<<<<<<<<<<<<<<<<<<<<<<<<<<<<<<<<<<<<<<<<<<<<<<<<<<


