\chapter{\txt
{構文解析モジュール:\code{parser2}}
{Parsing module : \code{parser2}}
}
\label{chap:parsing}
\ifjp%>>>>>>>>>>>>>>>>>>>>>>>>>>>>>>>>>>>>>>>>>>>>>>>>>>>>>>>>>>>>>>>>>>>
\begin{enumerate}
\item ソースロケーション \code{src/compiler/parser2/main}以下のファイル
\item 機能概要
	構文解析処理
\item 処理概要
以下のデータと処理を定義
\begin{enumerate}
\item ソース言語の構文解析のための文法と語彙の定義
\item インターフェイス言語の構文解析のための文法の定義
\item ソース言語のパーザの定義
\item インタフェイス言語のパーザの定義
\end{enumerate}
\end{enumerate}
\else%%%%%%%%%%%%%%%%%%%%%%%%%%%%%%%%%%%%%%%%%%%%%%%%%%%%%%%%%%%%%%%%%%%%
\fi%%%%<<<<<<<<<<<<<<<<<<<<<<<<<<<<<<<<<<<<<<<<<<<<<<<<<<<<<<<<<<<<<<<<<<


\section{\txt{\code{parser2}モジュールの処理の詳細}{The details of \code{parser2} module}}
\ifjp%>>>>>>>>>>>>>>>>>>>>>>>>>>>>>>>>>>>>>>>>>>>>>>>>>>>>>>>>>>>>>>>>>>>
	構文解析モジュールディレクトリ\code{src/compiler/parser2/main/}
を構成するファイルの機能を記述する.

\subsection{\code{iml.grm}}
\begin{enumerate}
\item 機能概要 ソース言語の文法記述.smlyaccの入力ファイル
\item 処理概要 
\begin{itemize}
\item ソース言語とインターフェイス言語共通の語彙の定義
\item ソース言語の文法規則の定義
\end{itemize}
\item 定義の詳細
\begin{itemize}
\item レコードラベルを除く種々の名前(識別子)は,文字列ではなく,位置を
データの中にもつ\code{Symbol.symbol}として定義されている.
\item SQLの語彙との共存のため,プログラム変数に関わる識別子は,終端記号
を直接文法で用いるのではなく,以下の非終端記号として定義している.
\begin{tabular}{ll}
\code{id of Symbol.symbol} 
& 
\smlsharp{}文脈での識別子.SQLのキーワードも含む.\\
\code{id\_noSQL of Symbol.symbol}
& 
SQL文脈での識別子.SQLキーワードは含まない.
\end{tabular}

\end{itemize}
\end{enumerate}

\subsection{\code{iml.lex}}
\begin{enumerate}
\item 機能概要 ソース言語とインターフェイス言語共通の語彙解析規則の定義,smllexの入力ファイル
\item 処理概要 
\begin{enumerate}
\item トークンデータ型\code{T}の定義
\item 各トークンの正規言語表現の定義
\end{enumerate}
\item インターフェイス

ソース言語パーザ\code{Parser.sml}とインターフェイス言語パーザ
\code{InterfaceParser.sml}双方で使用されるトークンを定義している.

この構造のため,本定義では,smlyaccが定義するトークンと同型の独自のトー
クン\code{T}を定義し,各パーザがこのデータ構造を,それぞれのパーザが
使用するトークンデータに変換している.

	この状況は,通常のStandard MLプログラムでは,smlyaccが出力するトー
クンを受け取るファンクタとして定義されるのが自然であろうが,smllexの出力
する巨大なコンスタントリストリテラルと\smlsharp{}のファンクタコンパイラ
が生成する巨大な引数をもつ関数の組合せが,現状の\smlsharp{}コンパイラ
のコンパイル時間を増大させているため,現状のような構造となっている.
\end{enumerate}

\subsection{\code{interface.grm}}
\begin{enumerate}
\item 機能概要 インターフェイス言語のパーザ
\item 処理概要 
\begin{itemize}
\item インターフェイス言語の文法規則の定義
\end{itemize}
\end{enumerate}

\subsection{\code{Parser.sml}}
\begin{enumerate}
\item 機能概要 ソース言語のパーザー
\item 処理概要 iml.grmのyacc出力に対する標準的なパーザ
\item インターフェイス
\code{src/compiler/toplebel2/main/Top.smi}から参照される.
\end{enumerate}

\subsection{\code{InterfaceParser.sml}}
\begin{enumerate}
\item 機能概要 インターフェイス言語のパーザー
\item 処理概要 interface.grmのyacc出力に対する標準的なパーザ
\item インターフェイス
\code{src/compiler/loadfile/main/LoadFile.smi}から参照される.
\end{enumerate}
\else%%%%%%%%%%%%%%%%%%%%%%%%%%%%%%%%%%%%%%%%%%%%%%%%%%%%%%%%%%%%%%%%%%%%
\fi%%%%<<<<<<<<<<<<<<<<<<<<<<<<<<<<<<<<<<<<<<<<<<<<<<<<<<<<<<<<<<<<<<<<<<

