\part{\txt{チュートリアル}{Tutorials}}
\label{part:tutorial}

\chapter{\txt{\smlsharp{}プログラミング環境の準備}
{Setting up \smlsharp{} programming environment}
}
\label{chap:tutorialEnvironment}

\ifjp%>>>>>>>>>>>>>>>>>>>>>>>>>>>>>>>>>>>>>>>>>>>>>>>>>>>>>>>>>>>>>>>>>>>
	第2部では,\smlsharp{}でプログラミングをマスターするための
チュートリアルを提供します.
	まず\smlsharp{}プログラミング環境を整えましょう.
\else%%%%%%%%%%%%%%%%%%%%%%%%%%%%%%%%%%%%%%%%%%%%%%%%%%%%%%%%%%%%%%%%%%%%
	The part~\ref{part:tutorial} provides tutorials to get started
writing \smlsharp{} programs.

	Let us begin by setting up a programming environment.
\fi%%%%<<<<<<<<<<<<<<<<<<<<<<<<<<<<<<<<<<<<<<<<<<<<<<<<<<<<<<<<<<<<<<<<<<

\section{
\txt{Unix系OS,Emacsエディタ,その他ツールの整備}
    {Unix-family OS,Emacs,and other tools}}
\label{sec:tutorialEnvironmemt}

\ifjp%>>>>>>>>>>>>>>>>>>>>>>>>>>>>>>>>>>>>>>>>>>>>>>>>>>>>>>>>>>>>>>>>>>>
	プログラミングのためには,
\begin{itemize}
\item 使いやすい高機能エディタ
\item コンパイラとリンカー
\end{itemize}
を含むプログラミング環境が必要です.
	\smlsharp{}でのプログラミングに必要とされるプログラミング環境は,
\smlsharp{}コンパイラのインストールを除けば,C言語の場合と同様です.
	Javaなどの言語では,これらを統合した対象言語に特化したEclipseなど
の統合開発環境を使用する場合が多いですが,Cとの直接連携機能を用いたシス
テムプログラミングやSQLを使ったデータベース操作などの\smlsharp{}の先端機
能を駆使したプログラミングを十分に楽しむためには,以下のような標準的なプ
ログラミング環境を整えることをお勧めします.

\begin{itemize}
\item {\bf Unix系のOS.}
	高度なプログラム開発には,Linux,FreeBSD,Mac OS等のUnix
系OSが豊富なツールを含んでいて便利です.
	Windows系のOSがインストールされたPCの場合は,VMware,VirtualBoxなど
の仮想マシンを利用してLinuxなどの環境を容易に構築できます.
	Windows系OSでは,Cygwinでもほぼ同等の環境が得られます.
	Windows系OSで,Emacsが自由に使いこなせれば,MinGW(+MSYS)でも十
分にプログラミングを楽しむことができると思われます.

\item {\bf Emacsエディタ.}
	プログラミングにもっとも適したエディタの一つです.
	プログラミングは,テキスト形式の文を書き,コンパイルしテストをし,
修正する,という作業の繰り返しであり,その大部分の時間はテキストエディタ
との付き合いとなります.
	そこで,強力でカスタマイズ可能なテキストエディタの選択が重要です.
	Emacs系テキストエディタ(GNU Emacs,XEmacsなど)は,複数のバッファ
の操作,ディレクトリなどのファイルの管理,コンパイラなどのコマンド起動,
さらに,それらの柔軟なカスタマイズ機能を備えた強力なエディタであり,プロ
グラミングやLaTeX文書などの複雑で高度な文書の作成に最適なエディタです.
	使い始める時に少々練習が必要ですが,使いこなせば,プログラム開発
の強力な道具となります.
	Unix系OSでは通常デフォルトでインストール済みです.

\item {\bf Cコンパイラ.}
	\smlsharp{}コンパイラは,ソースコードをx86アーキテクチャのネイ
ティブコードにンパイルし,OS標準のオブジェクトファイルを作成し,さらに
それらをリンクし実行形式ファイルを作成します.
	この過程でCコンパイラドライバ({\tt gcc}や{\tt clang}など)を
通してリンカーを呼び出します.
	Cコンパイラは,Unix系OSやCygwin,MinGWのインストール時にプログ
ラム開発用に適したシステムを選択すれば標準でインストールされているはずです.

	\smlsharp{}プログラムのみをコンパイルしリンクする場合は,ユーザ
が直接Cコンパイラを呼び出す必要はありませんが,C言語との直接連携
機能を使いより高度なプログラミングを行うためにも,Cコンパイラに
慣れておくことをお勧めします.

\item {\bf データベースシステム.}
	\smlsharp{}のデータベース機能を使用するにはデータベースシステムを
インストールする必要があります.
	第\version{}版はPostgreSQL,MySQL,ODBCに対応しています.
	データベースアクセス機能を使用するには,これらのうちいずれかの
データベースシステムをインストールする必要があります.
	これらのうち,\smlsharp{}との連携が最もテストされているのは
PostgreSQLです.
	詳細はOSに依存しますが,いずれの場合も簡単にインストールできるは
ずです.
\end{itemize}

%	第\ref{sec:tutorialInstall}節で説明する通り,Windowsのユーザ
%に対しても,コンパイル済みのバイナリープログラムを用意しています.
%	このバイナリを使用すれば,Windowsのコマンドプロンプトで
%\smlsharp{}コマンドが使用可能になります.
%	その場合でも,Windows上にEmacsをインストールすることをおすすめし
%ます.
%	Emacsが持つshell機能をを使えば,\smlsharp{}コンパイラだけでも,
%十分にに本格的な\smlsharp{}言語のプログラムを楽しむことができるはずです.
\else%%%%%%%%%%%%%%%%%%%%%%%%%%%%%%%%%%%%%%%%%%%%%%%%%%%%%%%%%%%%%%%%%%%%
	To enjoy writing programs, you need to set up a programming
environment including 
\begin{itemize}
\item a good text editor, and
\item a compiler and a linker.
\end{itemize}
	The environment required for \smlsharp{} programming is
essentially the same as in any other programming, except of course for
the \smlsharp{} compiler.
	In Java and other languages, an integrated development
environment such as Eclipse is often used, but for \smlsharp{}
programming, we recommend the following standard system development
environments.

\begin{itemize}
\item {\bf OS in the Unix family.}
	A Unix-family OS such as Linux, FreeBSD (including Mac OS)
provides a rich collection of programming tools.
	This would be your first choice.
	It is easy to set up Linux on Windows using a virtual machine
such as VMWare.
	Cygwin on Windows OS provides a similar environment.

	If you do not plan to develop a large system with various tools
such as {\tt make} and {\tt configure/autoconf} then
the combination of Emacs and MinGW (+ MSYS) on Windows would be a
reasonable alternative.
	
\item {\bf Emacs editor.}
	This is one of the best editor for programming, which is a
repeated process of editing an ASCII text file and and compiling it.
	In most of the time, you are interacting with your text editor.
	So choosing a highly custormisable high-performance editor is
important.
	Among various choices, we recommend one in the Emacs-family (GNU
Emacs, XEmacs).
	Emacs is a powerful custormisable text editor, and it can also
perform command execution and file system management.
	It requires some practice at the beginnings, but once you mastered
its basic functionality, it will become a powerful tool in programming.

\item {\bf C compiler.}
	\smlsharp{} compiler generates x86 native code, creates an
object file in a standard format (e.g. ELF), and generates an executable
code by linking object files with C libraries.
	In this process, it calls C compiler deriver command such as
{\tt gcc} or {\tt clang}.
	One of them should already be installed in a Unix-family OS
including Cygwin and MinGW.

	If your purpose is to make a small program entirely within
\smlsharp{}, then you will not need to invoke a C compiler directly.
	However, if you want to make a paractical program, you may want to
call some system library functions or you may write some part of your
system in C and call that function from your ML code.
	This is straightforward in \smlsharp{}.
	To exploit this feature, we recommend that you familiarize
yourself with C compiler.

\item {\bf database systems.}
	\smlsharp{} seamlessly integrates SQL.
	\smlsharp{} version \version{} supports PostgreSQL, MySQL and ODBC.
	If you set up one of them, then you can use a database system
directly within your \smlsharp{} code.
\end{itemize}

%	As we explain in Section~\ref{sec:tutorialInstall},
%we prepared a binary package for Windows users.
%	By installing this package, \smlsharp{} compiler can be used 
%as a Windows application.
%	In this case, we still recommend to install Emacs.
%	With Emacs shell, you can enjoy \smlsharp{} programming more
%comfortably.
\fi%%%%<<<<<<<<<<<<<<<<<<<<<<<<<<<<<<<<<<<<<<<<<<<<<<<<<<<<<<<<<<<<<<<<<<


\section{
\txt{\smlsharp{}コンパイラの構造とブートストラップ}
    {Bootstrapping the \smlsharp{} compiler}
}
\label{sec:tutorialBootstrap}

\ifjp%>>>>>>>>>>>>>>>>>>>>>>>>>>>>>>>>>>>>>>>>>>>>>>>>>>>>>>>>>>>>>>>>>>>
	この節の内容は,\smlsharp{}コンパイラをインストールし使用する上で
理解する必要はありませんが,やや時間がかかる\smlsharp{}コンパイラのインス
トール処理の理解や,さらにコンパイラの一般的な構造を理解する上で有用と思
います.

	\smlsharp{}\version{}版コンパイラは,\smlsharp{}言語で書かれた
ファイルを分割コンパイルし,x86ネイティブコードを生成します.
	対話型モードも,このコンパイラを使い,
(1)現在の環境下で分割コンパイル,
(2)システムとのリンク,
(3)オブジェクトファイルの動的ロード
を繰り返すことによって実現しています.
	\smlsharp{}コンパイラは\smlsharp{}言語とC言語で書かれており,
さらに自分自身のコンパイル時に以下のツールを使用しています.
\begin{itemize}
\item ml-lex,ml-yacc.字句解析および構文解析ツール.
\item SMLFormat.プリンタ自動生成ツール
\item 基本ライブラリ
\end{itemize}
	これらも\smlsharp{}言語で書かれています.

	\smlsharp{}コンパイラは,各\smlsharp{}ファイル(Standard MLファ
イルを含む)をシステム標準のオブジェクトファイルにコンパイルします.
	コンパイルしたファイルは,システム標準のリンカによって
実行形式ファイルに結合できます.
	従って,\smlsharp{}コンパイラは,C言語のコンパイラと\smlsharp{}
コンパイラがあれば構築できます.
	しかし,もちろん,\smlsharp{}コンパイラを初めてインストールする
時には,\smlsharp{}コンパイラはまだ存在しません.
	\smlsharp{}\version{}版では,以下の手順でこのブートストラップの
問題を解決し,\smlsharp{}をビルドしインストールしています.

\begin{enumerate}
\item
	Cで書かれたランタイムライブラリをコンパイルし静的リンクライブラリを
生成します.
\item
	\smlsharp{}コンパイラのソースコードをコンパイルするのに十分な大きさの
\smlsharp{}コンパイラ({\tt minismlsharp})を,古い\smlsharp{}コンパイラで
事前にコンパイルしておき,
\smlsharp{}のソースツリーにコンパイル済みのLLVM IRファイルを用意しています.
\item
	インストール先のシステムの下で,
{\tt minismlsharp}のLLVM IRファイルをLLVMを用いてコンパイルし,
ランタイムライブラリとリンクし,
{\tt minismlsharp}コマンドを生成します.
\item
	{\tt minismlsharp}コマンドを用いて,\smlsharp{}をコンパイルするための
各種ツールとライブラリ,および\smlsharp{}コンパイラをコンパイルします.
	このコンパイルは,ツールとライブラリの依存関係から,おおよそ
以下の順番で行われます.
\begin{enumerate}
\item 基本ライブラリ(Basis Library)のコンパイル.
\item {\tt smllex}コマンドのコンパイルとリンク.
\item ML-yaccライブラリと{\tt smlyacc}コマンドのコンパイルとリンク.
\item {\tt smllex}および{\tt smlyacc}コマンドによるパーザの生成.
\item {\tt smlformat}コマンドのコンパイルとリンク.
\item {\tt smlformat}コマンドによるプリンタの生成.
\item すべてのライブラリのコンパイル.
\item {\tt smlsharp}コマンドのコンパイルとリンク.
\end{enumerate}
\item 以下のファイルを所定の位置にインストールします.
\begin{itemize}
\item ランタイムライブラリのライブラリファイル.
\item ライブラリのインターフェースファイル,オブジェクトファイル,および
シグネチャファイル.
\item {\tt smllex},{\tt smlyacc},{\tt smlformat},{\tt smlsharp}の
各コマンド.
\end{itemize}
\end{enumerate}

	これらは以上のステップが示す様に,各ソースファイルやコマンド間に
は複雑な依存関係があります.
	さらに,いくつかのソースファイルの処理は,OSの環境に依存しています.
	これらは,大規模なシステムの典型です.
	これら依存関係を制御する一つの手法は,%システムで提供されている
GNU Autoconfによる{\tt configure}スクリプトと{\tt make}コマンドの使用です.
	\smlsharp{}コンパイラは,インターフェイスファイルに従って各ファイ
ルをコンパイルします.
	個別のファイルが必要とするファイルは,このインターフェイスファイル
に記述されています.
	\smlsharp{}コンパイラは,この依存関係を{\tt make}コマンドが読み
込めるフォーマットで出力する機能を持っています.
\begin{enumerate}
\item {\tt smlsharp -M $\mathit{smlFile}$}.
	ソースファイル$\mathit{smlFile}$を読み,このソースファイルを
コンパイルするために必要なファイルのリストをMakefile形式で出力
します.
\item {\tt smlsharp -Ml $\mathit{smiFile}$}.
	インターフェイスファイル$\mathit{smiFile}$をトップレベルの
ファイルとみなし,このトップレベルから生成されるべき実行形式コマンドの
リンクに必要なオブジェクトファイルのリストをMakefile形式で出力します.
\end{enumerate}	

	\smlsharp{}システムは,この機能を利用して,上記の手順を実行する
{\tt Makefile}を生成しています.
	この{\tt Makefile}によって,通常の大規模システムの開発と同様,再
コンパイルが必要なファイルのみ{\tt make}コマンドによってコンパイル,リン
クが実行されます.

\else%%%%%%%%%%%%%%%%%%%%%%%%%%%%%%%%%%%%%%%%%%%%%%%%%%%%%%%%%%%%%%%%%%%%

	This section outlines the structure of \smlsharp{} compiler and
the method to building (bootstrapping) it.
	You do not have to understand this section for installing and
using \smlsharp{} compiler, but you may find this section informative in
understanding various messages during compilation of \smlsharp{} and also 
a structure of a compiler in general.

	\smlsharp{} system consists of a single compiler that performs
separate compilation. 
	Its interactive mode is realized by the top-level-loop
performing the following steps:
\begin{enumerate}
\item compile the user input using the current static environment as its
interface,
\item link the object file with the current system to generate a shared
executable file,
\item dynamically load the shared executable in the current system, and
call its entry point.
\end{enumerate}

	\smlsharp{} is written in \smlsharp{} and C.
	In addition, it uses the following tools during compiling the
\smlsharp{} compiler.
\begin{itemize}
\item ml-lex,ml-yacc: a lexical analyzer generator and a parser generator.
\item SMLFormat: a printer generator.
\item The Standard ML Basis Library.
\end{itemize}
	All of them are written in Standard ML.

	\smlsharp{} compiler compiles each \smlsharp{} (which is a super
set of Standard ML) source file ({\tt source.sml}) into a system standard
object file ({\tt sample.o}).
	To generate an executable file, the compiled files are then linked
by the standard linker ({\tt ld} in Unix-family OS) invoked through
C compiler deriver command ({\tt gcc} or {\tt clang}).
	So, in order to build the \smlsharp{} compiler, it is sufficient
to have a C compiler and an \smlsharp{} compiler.
	But of course, at the time when the \smlsharp{} compiler is
first built, an \smlsharp{} compiler is not available.
	The standard step of solving this bootstrap problem is the
following.

\begin{enumerate}
\item
	Compile \smlsharp{} runtime library written in C and archive it
as a static link library.
\item 
	Obtain a pre-compiled LLVM IR source file of a minimal \smlsharp{}
compiler
{\tt minismlsharp} that is sufficient for compiling all the source files
used in the \smlsharp{} compiler. 
	The pre-compiled files are typically generated by an older version
of \smlsharp{} compiler. 
\item 
	In the system where the target \smlsharp{} compiler is
installed, assemble the {\tt minismlsharp} LLVM IR file, link them
with the runtime library, and create a {\tt minismlsharp} command.
\item
	By using this {\tt minismlsharp} command, compile all of the tools
and libraries, and the full-featured \smlsharp{} compiler.
	This procedure is roughly performed as follows.
\begin{enumerate}
\item Compile the Basis Library.
\item Compile and link {\tt smllex} command.
\item Compile and link ML-yacc library and {\tt smlyacc} command.
\item Generate parser source code by {\tt smllex} and {\tt smlyacc} command.
\item Compile and link {\tt smlformat} command.
\item Generate printer source code by {\tt smlformat} command.
\item Compile all of the libraries bundled to the \smlsharp{} distribution.
\item Compile and link {\tt smlsharp} command.
\end{enumerate}
\item The following files are installed to specified destination directories.
\begin{itemize}
\item The static link library of the runtime library.
\item Interface files, object files, and signature files of the libraries
bundled to the \smlsharp{} distribution.
\item {\tt smllex}, {\tt smlyacc}, {\tt smlformat}, {\tt smlsharp} command.
\end{itemize}
\end{enumerate}

	As outlined above, there are complex dependencies among source
files and commands.
	Furthermore, processing some of these files depend on the
underlying OS.
	This is a typical situation in a large system development.
	One well established method to solve these dependency problems is
to use {\tt configure} script generated by GNU Autoconf and
{\tt make} command.

	\smlsharp{} compiler compiles each source file according to its
interface file, which describes the set of files require by the source
file.
	\smlsharp{} compiler can also generate a list of files on which
each source file depends in the {\tt Makefile} format that can be
processed by {\tt make} command. 
	\smlsharp{} compiler does this task, when it is given one of the
following switch.

\begin{enumerate}
\item {\tt smlsharp -M $\mathit{smlFile}$}.
	The compiler generates the dependency for the source file
$\mathit{smlFile}$ to be compiled in the Makefile format. 
\item {\tt smlsharp -Ml $\mathit{smiFile}$}.
	The compiler assumes that the file $\mathit{smiFile}$ specifies the
top level system, and generates the list of necessary object files in the
Makefile format. 
\end{enumerate}	

	In the \smlsharp{} project, we make a {\tt Makefile} that
performs the above described complicated sequence of compilation and
linking steps using the above functionality of \smlsharp{}.
	Invoking {\tt make} command on {\tt Makefile} re-compiles only
the necessary files to build \smlsharp{} compiler.
\fi%%%%<<<<<<<<<<<<<<<<<<<<<<<<<<<<<<<<<<<<<<<<<<<<<<<<<<<<<<<<<<<<<<<<<<

\section{
\txt{\smlsharp{}のインストール}
    {Installing \smlsharp{}}}
\label{sec:tutorialInstall}

\ifjp%>>>>>>>>>>>>>>>>>>>>>>>>>>>>>>>>>>>>>>>>>>>>>>>>>>>>>>>>>>>>>>>>>>>
	\smlsharp{}\version{}版は以下のプラットフォームで動作します.
\begin{itemize}
\item Linux(x86版.または32ビット開発環境のあるamd64版)
\item Mac OS X(Intel版)
%\item WindowsのMinGW(32ビット版)
\end{itemize}

	また,\smlsharp{}のコンパイルと実行には以下のサードパーティ製
ライブラリが必要です.
\begin{itemize}
\item GNU Multiple Precision Arithmetic Library (GMP)
\item LLVM 3.4
\end{itemize}
	GMPはLGPL(GNU Lesser General Public License)の下で配布されている
フリーソフトウェアです.
	LLVMはBSDスタイルライセンスの下で配布されているフリーソフトウェア
です.
	\smlsharp{}コンパイラは,これらのライブラリをリンクして生成されます.
	また,\smlsharp{}コンパイラが生成したすべての実行形式ファイルには
GMPがリンクされます.
	これらのライブラリは\smlsharp{}の配布パッケージには含まれていない
ので,\smlsharp{}コンパイラをインストールするユーザは,これらのライブラリを
別途インストールする必要があります.
	多くの場合,OSのパッケージ管理システムなどで簡単に導入できるはずです.

	LLVMはバージョン3.4をご用意ください.
	3.4より前のバージョンのLLVMでは\smlsharp{}をコンパイルできません.
	また,LLVMはバージョンによってAPIが大きく異なることが多いため,
3.4よりも新しいバージョンでコンパイルできるとは限りません.

	以上の環境があれば,\smlsharp{}は,簡単な手順でインストールでき
ます.
	各OS毎のインストール処理の詳細は以下のとおりです.
\else%%%%%%%%%%%%%%%%%%%%%%%%%%%%%%%%%%%%%%%%%%%%%%%%%%%%%%%%%%%%%%%%%%%%
	The \smlsharp{} version \version{} works on one of the following
platforms.
\begin{itemize}
\item Linux (x86, or amd64 with 32 bit development environment)
\item Mac OS X (Intel version)
%\item MinGW (32 bit version) on Windows 
\end{itemize}

	\smlsharp{} compiler requires the following libraries.
\begin{itemize}
\item GNU Multiple Precision Arithmetic (GMP)
\item LLVM 3.4
\end{itemize}
	GMP is a free software distributed under LGPL(GNU Lesser General
Public License).
	LLVM is a open-source software distributed under a BSD-style
license.
	\smlsharp{} compiler will be linked with all of the above libraries.
	The executable file of any user program compiled by the \smlsharp{}
compiler is linked with GMP.
	Since these libraries are not included in the \smlsharp{}
distribution package,
you need to install it before installing \smlsharp.

	Note that LLVM version 3.4 is exactly required.
	With older version of LLVM than 3.4, compilation of \smlsharp{} shall
fail.
	Since LLVM provide different APIs in different versions,
we cannot make sure whether \smlsharp{} can compile with newer version of
LLVM than 3.4 or not.

	Below, we show the details of \smlsharp{} installation steps for
each of the supported operating systems.
\fi%%%%<<<<<<<<<<<<<<<<<<<<<<<<<<<<<<<<<<<<<<<<<<<<<<<<<<<<<<<<<<<<<<<<<<

\subsection{\txt{Mac OS X}{Mac OS X}}
\ifjp%>>>>>>>>>>>>>>>>>>>>>>>>>>>>>>>>>>>>>>>>>>>>>>>>>>>>>>>>>>>>>>>>>>>
	MacPortsのPortfileが用意されていますので,MacPortsを使って
インストールするのが最も簡単です.
	MacPortsをセットアップ後,\smlsharp{}のPortfileをダウンロードし,
ローカルなPortfileとして設定した上で,{\tt port install smlsharp} を実行すれば
インストールされます.
	詳細は,以下のとおりです.
\begin{enumerate}
\item 
\url{http://www.macports.org/}を参照し,
MacPorts をセットアップする.

\item 
	適当なディレクトリを作成し,ローカルなPortfileリポジトリとして設定する.
	例えば,ローカルなPortfileを {\tt /opt/var/macports/sources/local}
に置く場合は,そのディレクトリを作成した上で,
{\tt /opt/etc/macports/sources.conf}ファイルに
\begin{program}
file:///opt/var/macports/sources/local [nosync]
\end{program}
と記述する.
	詳しくは,
\url{http://guide.macports.org/#development.local-repositories}
を参照.

\item 
	下記URLより\smlsharp{}のPortfileのアーカイブをダウンロードする.
\url{http://www.pllab.riec.tohoku.ac.jp/smlsharp/download/smlsharp-macports.zip}

\item 
	ローカルPortfileリポジトリのディレクトリで上記アーカイブを展開する.
\begin{program}
\$ cd /opt/var/macports/sources/local\\
\$ unzip smlsharp-macports.zip
\end{program}
	この結果,{\tt lang/smlsharp/Portfile}が現れる.

\item
	 ローカルPortfileリポジトリのディレクトリに対して{\tt portindex}
コマンドを実行する.
\begin{program}
\$ portindex /opt/var/macports/sources/local
\end{program}

\item 
	以上で\smlsharp{}パッケージがMacPortsに設定されたので,{\tt
port}コマンドで\smlsharp{}コンパイラをインストールする.
\begin{program}
\$ port install smlsharp
\end{program}
\end{enumerate}

	\smlsharp{}は32ビットモード(x86アーキテクチャ)のみ対応しています.
	Mac OS X 10.6 以降で64ビット版のMacPortsを使う場合,
GMPライブラリなどの\smlsharp{}が必要とするソフトウェアも
32ビット版または32ビット版を含むユニバーサル
バイナリである必要があります.
	この依存関係は\smlsharp{}のインストールの際に{\tt port}コマンドに
よって自動的に解消されるはずです.
	この解消のために,{\tt port}コマンドはインストール済みの64ビット版
GMPライブラリなどをユニバーサルバイナリ化するために再ビルドする場合がある
ことに注意してください.

\else%%%%%%%%%%%%%%%%%%%%%%%%%%%%%%%%%%%%%%%%%%%%%%%%%%%%%%%%%%%%%%%%%%%%

	We prepare an \smlsharp{} Portfile.
	After setting up MacPorts,download the \smlsharp{} Portfile,
set it up as a local Portfile, and do {\tt port install smlsharp}.
	We show some more details below.

\begin{enumerate}
\item 
Consult \url{http://www.macports.org/} and set up MacPorts.

\item 
	Create a directory and set it as a local Portfile repository.
	For example, if you plan to put local Portfiles in
 {\tt /opt/var/macports/sources/local}, create the directory
and write the following line
\begin{program}
file:///opt/var/macports/sources/local [nosync]
\end{program}
in the file {\tt /opt/etc/macports/sources.conf}.
	For more details, consult:
\url{http://guide.macports.org/#development.local-repositories}

\item 
	Download a Portfile archive from
\url{http://www.pllab.riec.tohoku.ac.jp/smlsharp/download/smlsharp-macports.zip}

\item 
	Extract the local Portfile repository in the repository
directory as:
\begin{program}
\$ cd /opt/var/macports/sources/local\\
\$ unzip smlsharp-macports.zip
\end{program}
	After this step, you will see {\tt lang/smlsharp/Portfile}.
\item
	 Execute {\tt portindex} command on the local Portfile
repository directory as:
\begin{program}
\$ portindex /opt/var/macports/sources/local
\end{program}

\item 
	The above steps should have set \smlsharp{} package to MacPorts.
	You can now install \smlsharp{} by {\tt port} command as:
\begin{program}
\$ port install smlsharp
\end{program}
\end{enumerate}

	\smlsharp{} version \version{} only support 32 bit mode on x86
architecture.
	So in Mac OS X 10.6 or later version with 64 bit MacPorts,
GMP library must be a 32 bit version or a universal binary that include
a 32 bit version.
	This dependency should be automatically resolved when you
install \smlsharp{} by {\tt port} command.
	To resolve this dependency, {\tt port} command may re-build
64-bit GMP to make it as a universal binary.
\fi%%%%<<<<<<<<<<<<<<<<<<<<<<<<<<<<<<<<<<<<<<<<<<<<<<<<<<<<<<<<<<<<<<<<<<

\subsection{\txt{Ubuntu}{Ubuntu}}
\ifjp%>>>>>>>>>>>>>>>>>>>>>>>>>>>>>>>>>>>>>>>>>>>>>>>>>>>>>>>>>>>>>>>>>>>
	Ubuntu用のdeb形式のバイナリパッケージが用意されています.
	このバイナリパッケージを{\tt dpkg}コマンドでインストールすれば,
依存する他のパッケージも自動的にインストールされ,
\smlsharp{}コンパイラが使用可能になります.
	詳細は,以下のとおりです.

\begin{enumerate}
\item 下記URLより\smlsharp{}のdebパッケージをダウンロードする.
\begin{itemize}
\item Ubuntu i386版:
\url{http://www.pllab.riec.tohoku.ac.jp/smlsharp/download/smlsharp-\version{}-1_ubuntu-i386.deb}
\item Ubuntu amd64版:
\url{http://www.pllab.riec.tohoku.ac.jp/smlsharp/download/smlsharp-\version{}-1_ubuntu-amd64.deb}
\end{itemize}
	最新のdebパッケージは
\url{http://www.pllab.riec.tohoku.ac.jp/smlsharp/?Download}からも取得できます.
\item ダウンロードしたパッケージを{\tt dpkg --install}コマンドでインストール
する.
このとき,もし\smlsharp{}が必要とするパッケージがインストールされていなければ,
その旨を表すエラーメッセージが表示されるので,必要なパッケージを
{\tt apt-get}コマンドでインストールしてから再度{\tt dpkg}コマンドを実行する.
例えば,i386版であれば,以下のコマンドを実行する.
\begin{program}
%\$ apt-get install gcc\\
%\$ apt-get install libgmp-dev\\
\$ sudo dpkg --install smlsharp-\version{}-1\_i386.deb
\end{program}
\end{enumerate}

	なお,このdebパッケージはDebianまたはUbuntuの公式なパッケージで
はないことに注意してください.
	公式のパッケージとは異なり,メンテナによる署名は行われていません.
	また,amd64版のdebパッケージも32ビット版\smlsharp{}のみを含むことに
ご注意ください.
	\smlsharp{}コンパイラは64ビットコードを出力しません.

\else%%%%%%%%%%%%%%%%%%%%%%%%%%%%%%%%%%%%%%%%%%%%%%%%%%%%%%%%%%%%%%%%%%%%

	\mbox{.deb} packages for both i386 and amd64 version are available.
	Download it and install it by {\tt dpkg} command, then
\smlsharp{} compiler is available on your system.
	We show some more details below.

\begin{enumerate}
\item Download one of the \mbox{.deb} packages from the following URL.
\begin{itemize}
\item Ubuntu i386 version:
\url{http://www.pllab.riec.tohoku.ac.jp/smlsharp/download/smlsharp-\version{}-1_ubuntu-i386.deb}
\item Ubuntu amd64 version:
\url{http://www.pllab.riec.tohoku.ac.jp/smlsharp/download/smlsharp-\version{}-1_ubuntu-amd64.deb}
\end{itemize}
	The latest deb packages is also available from
\url{http://www.pllab.riec.tohoku.ac.jp/smlsharp/?Download}.
\item 
	Do {\tt dpkg --install} with the downloaded \mbox{.deb} package.
	If other packages required for \smlsharp{} has not been installed
on your system, {\tt dpkg} command fails and produces an error message with
a list of the required packages.
	In this case, you need to install the required package by {\tt apt-get}
command and do {\tt dpkg --install} command again.
	For example, on i386 Debian GNU/Linux, do the following commands.
\begin{program}
%\$ apt-get install gcc\\
%\$ apt-get install libgmp-dev\\
\$ sudo dpkg --install smlsharp-\version{}-1\_i386.deb
\end{program}
\end{enumerate}

	Note that these \mbox{.deb} packages are not a part of an official release
of Debian or Ubuntu system.
	Unlike official packages, they do not include any signature of
package maintainer.
	Also note that the \mbox{.deb} package for amd64 only includes 32 bit
version of \smlsharp{} compiler.
	It does not produce any 64 bit code.

\fi%%%%<<<<<<<<<<<<<<<<<<<<<<<<<<<<<<<<<<<<<<<<<<<<<<<<<<<<<<<<<<<<<<<<<<

\subsection{\txt{Debian GNU/Linux}{Debian GNU/Linux}}

\ifjp%>>>>>>>>>>>>>>>>>>>>>>>>>>>>>>>>>>>>>>>>>>>>>>>>>>>>>>>>>>>>>>>>>>>

	Debian GNU/Linuxには\smlsharp{}が公式リリースの一部として
取り込まれています.
	例えば以下のような標準的な{\tt aptitude}コマンドの使用により
\smlsharp{}をインストールすることができます.
\begin{program}
\$ sudo aptitude install smlsharp
\end{program}
	\smlsharp{}の新しいリリースがDebian公式パッケージとして配布される
までの間にタイムラグがあります.
	最新版の\smlsharp{}をすぐに利用したい場合はソースからビルドして
ください.

\else%%%%%%%%%%%%%%%%%%%%%%%%%%%%%%%%%%%%%%%%%%%%%%%%%%%%%%%%%%%%%%%%%%%%

	Debian GNU/Linux contains \smlsharp{} as a part of official
distribution.
	You can install \smlsharp{} through {\tt aptitude} command.
	For example,
\begin{program}
\$ sudo aptitude install smlsharp
\end{program}
	If you want to use the latest version of \smlsharp{} that is not
available from the Debian official distribution, compile \smlsharp{}
from source code.

\fi%%%%<<<<<<<<<<<<<<<<<<<<<<<<<<<<<<<<<<<<<<<<<<<<<<<<<<<<<<<<<<<<<<<<<<

%\subsection{Windows}

%\ifjp%>>>>>>>>>>>>>>>>>>>>>>>>>>>>>>>>>>>>>>>>>>>>>>>>>>>>>>>>>>>>>>>>>>>
%	32ビット版MinGW用のバイナリパッケージが,
%\url{http://www.pllab.riec.tohoku.ac.jp/smlsharp/download/smlsharp-1.0.1-1-mingw32-bin.tar.lzma}
%に用意されています.
%	32ビット版MinGWのCコンパイラとGMPライブラリをセットアップした後,
%このバイナリパッケージをMinGWのルートディレクトリで展開すれば,
%\smlsharp{}コンパイラが使用可能になります.
%
%	単に展開する代わりに以下の手順を実行すれば,\smlsharp{}コンパイ
%ラがパッケージ管理下に入り,GMPなど必要なソフトウェアの自動インストールや,
%将来のバージョンアップなどにも簡単に対応できるようになります.
%\begin{enumerate}
%\item MinGWのホームページを参考に,MinGWをセットアップする.
%以下では,MinGWは
%\begin{verbatim}
% C:\MinGW
%\end{verbatim}
%にインストールされたものと仮定する.
%\item 
%ファイル
%\begin{verbatim}
% C:\MinGW\var\lib\mingw-get\data\profile.xml
%\end{verbatim}
%の{\tt <profile>}要素の中に以下の設定を書き加える.
%\begin{verbatim}
% <repository uri="http://www.pllab.riec.tohoku.ac.jp/smlsharp/download/mingw32/%F.xml.lzma">
%   <package-list catalogue="smlsharp-package-list" />
% </repository>
%\end{verbatim}
%\item 以下のコマンドを実行する.
%\begin{program}
%mingw-get update\\
%mingw-get install smlsharp
%\end{program}
%\end{enumerate} 
%\else%%%%%%%%%%%%%%%%%%%%%%%%%%%%%%%%%%%%%%%%%%%%%%%%%%%%%%%%%%%%%%%%%%%%
%	A binary package for 32 bit MinGW is available from:
%\url{http://www.pllab.riec.tohoku.ac.jp/smlsharp/download/smlsharp-1.0.1-1-mingw32-bin.tar.lzma}.
%	After setting up a C compiler and GMP library for 32 bit MinGM,
%simply extract the binary package in the MinGW root directory.
%
%	Instead of simply extracting the binary package, the following
%steps make \smlsharp{} compiler managed by the package system.
%\begin{enumerate}
%\item Set up MinGW consulting MinGW home page.
%	In this explanation, we assume that MinGW is installed at:
%\begin{verbatim}
% C:\MinGW
%\end{verbatim}
%
%\item 
%Add the description
%\begin{verbatim}
% <repository uri="http://www.pllab.riec.tohoku.ac.jp/smlsharp/download/mingw32/%F.xml.lzma">
%   <package-list catalogue="smlsharp-package-list" />
% </repository>
%\end{verbatim}
%to the {\tt <profile>} element in the file:
%\begin{verbatim}
% C:\MinGW\var\lib\mingw-get\data\profile.xml
%\end{verbatim}
%
%\item Execute the following commands:
%\begin{program}
%mingw-get update\\
%mingw-get install smlsharp
%\end{program}
%\end{enumerate} 
%\fi%%%%<<<<<<<<<<<<<<<<<<<<<<<<<<<<<<<<<<<<<<<<<<<<<<<<<<<<<<<<<<<<<<<<<<

\subsection{\txt{ソースからビルドする場合}{Building from the source}}
\ifjp%>>>>>>>>>>>>>>>>>>>>>>>>>>>>>>>>>>>>>>>>>>>>>>>>>>>>>>>>>>>>>>>>>>>
	その他のシステムではソースからビルドしてください.
	ソースからのビルドには,以下の開発ツールとライブラリが事前に
インストールされている必要があります.
\begin{enumerate}
\item プログラム開発用のツール群GNU binutils(GNU Binary Utilities).
\item CおよびC++コンパイラ(gccまたはclang)
\item make(GNU makeを推奨)
\item GMPのライブラリ本体とヘッダーファイル
\item LLVM 3.4のコマンド,ライブラリ,およびヘッダーファイル
\end{enumerate}
	64ビットOSを使う場合,これらのツールおよびライブラリの32ビット版が
必要です.

	32ビット版GMPおよびLLVMがOSのパッケージシステムによって
提供されていない場合,それらもソースからビルドする必要があります.
	これらのコンパイルの詳細は,各ソフトウェアの公式ドキュメントなどを
参照してください.
	以下に,本文書の執筆時点での大まかなインストール手順を示します.

\begin{itemize}
\item GMP

	GMPのWebページ \url{http://gmplib.org/} などから
ソースコードを入手,展開し,{\tt configure}に{\tt ABI=32}オプションを与える
ことで32ビット版GMPをビルドすることができます.
\begin{program}
\$ ./configure ABI=32
\$ make
\$ make install
\end{program}
	すでにインストールされているGMPライブラリとの衝突を防ぐためにも,
さらに{\tt --prefix}オプションなどを与えると良いでしょう.

\item LLVM
	LLVMのWebページ \url{http://llvm.org/} などから
ソースコード {\tt llvm-3.4.src.tar.gz} を入手,展開し,
{\tt configure}に{\tt CC='gcc -m32'}および{\tt CXX='g++ -m32'}オプションを
与えることで32ビット版LLVMをビルドすることができます.
\begin{program}
\$ ./configure CC='gcc -m32' CXX='g++ -m32'
\$ make
\$ make install
\end{program}
	GMP同様,さらに{\tt --prefix}オプションを与えると良いでしょう.

\end{itemize}

	以上の環境の下で,以下の手順で\smlsharp{}をソースからビルドします.
\begin{enumerate}
\item ソースパッケージ(
\url{http://www.pllab.riec.tohoku.ac.jp/smlsharp/download/smlsharp-\version{}.tar.gz}
)をダウンロード.
	最新のソースパッケージは
\url{http://www.pllab.riec.tohoku.ac.jp/smlsharp/?Download}からも取得できます.
\item 適当な\smlsharp{}ソースディレクトリを決め,そこに展開.
\item 適当な\smlsharp{}のインストール先ディレクトリを決める.そのディレクトリを
$\mathit{prefix}$とする.
\item
	\smlsharp{}ソースディレクトリにて{\tt configure}スクリプトを実行する.
        このとき,{\tt --prefix}オプションに$\mathit{prefix}$を指定する.
	また,64ビットOSでは,C/C++コンパイラおよびリンカが32ビット版となる
ように{\tt CC},{\tt CXX}および{\tt LD}を適切に設定する.
	マルチスレッド(pthread)サポートを有効化したい場合は,
{\tt --enable-thread}スイッチを加える.

i386 Linux の場合:
\begin{program}
\$ ./configure --prefix=$\mathit{prefix}$
\end{program}
amd64 Linux の場合:
\begin{program}
\$ ./configure CC='gcc -m32' LD='ld -m elf\_i386' --prefix=$\mathit{prefix}$
\end{program}
\item
	{\tt make}コマンドを実行しビルドとインストールを行う.
\begin{program}
\$ make\\
\$ make install
\end{program}
	システム管理の都合上,インストールされるファイルを
$\mathit{prefix}$とは異なるディレクトリ$\mathit{prefix}'$に出力させたい場合は,
{\tt make install}コマンドに{\tt DESTDIR=$\mathit{prefix}'$}オプションを指定する.
\end{enumerate}

	以上により,以下のファイルがインストールされます.
\begin{enumerate}
\item {\tt smlsharp}コマンド:{\tt $\mathit{prefix}$/bin/smlsharp}
\item {\tt smlformat}コマンド:{\tt $\mathit{prefix}$/bin/smlformat}
\item {\tt smllex}コマンド:{\tt $\mathit{prefix}$/bin/smllex}
\item {\tt smlyacc}コマンド:{\tt $\mathit{prefix}$/bin/smlyacc}
\item ライブラリファイル:{\tt $\mathit{prefix}$/lib/smlsharp/}ディレクトリ以下のファイル
\end{enumerate}
以下のコマンドで\smlsharp{}コンパイラが起動できるはずです.
\begin{program}
\$ $\mathit{prefix}$/bin/smlsharp
\end{program}

補足とヒント:
\begin{itemize}
\item マルチコアCPUをお使いの方は,GNU makeならば{\tt make}コマンドを実行
するとき{\tt -j\nonterm{n}}スイッチを指定すると,並列にコンパイルされ,
実行時間が短縮される場合があります.\nonterm{n}は並列度です.コアの数程度
に指定すると良い結果が得られると思います.
\end{itemize}

\else%%%%%%%%%%%%%%%%%%%%%%%%%%%%%%%%%%%%%%%%%%%%%%%%%%%%%%%%%%%%%%%%%%%%
	For Linux and other systems, you need to build from the
\smlsharp{} source distribution.
	To do this, the following tools and libraries are required:
\begin{enumerate}
\item GNU binutils(GNU Binary Utilities),
\item C and C++ compiler (gcc or clang),
\item make (GNU make is recommended),
\item GMP library and its header files, and
\item LLVM 3.4.
\end{enumerate}
	In a 64 bit OS, these tools and libraries must be of 32 bit version.

	If your 64 bit OS does not provide a 32 bit GMP and LLVM,
you need to build those of 32 bit version from the source.
	See official documents of GMP and LLVM for details of this
procedure.
	For your information, we roughly present how to compile them
at the time when we write this document.

\begin{itemize}
\item GMP

	Get the GMP source from GMP web page \url{http://gmplib.org/},
and {\tt configure} it with {\tt ABI=32} option as in the following:
\begin{program}
\$ ./configure ABI=32\\
\$ make\\
\$ make install
\end{program}
	If your system have an official distribution of GMP, then you may
want to specify the install location of the 32 bit version through
{\tt --prefix} option.

\item LLVM
	Get the LLVM 3.4 source package {\tt llvm-3.4.src.tar.gz} from
LLVM web page \url{http://llvm.org/},
and {\tt configure} it with {\tt CC='gcc -m32'} and {\tt CXX='g++ -m32'}
options as in the following:
\begin{program}
\$ ./configure CC='gcc -m32' CXX='g++ -m32'\\
\$ make\\
\$ make install
\end{program}
	Similar to GMP, you may want to specify the install location
through {\tt --prefix} option.

\end{itemize}

	With these preparation, \smlsharp{} can be build in the
following steps.
\begin{enumerate}
\item Download the source distribution from:
\url{http://www.pllab.riec.tohoku.ac.jp/smlsharp/download/smlsharp-\version{}.tar.gz}.
	The latest version of the source package is also available from
\url{http://www.pllab.riec.tohoku.ac.jp/smlsharp/?Download}.
\item Select an \smlsharp{} source directory and extract the tar archive there.
\item Select an \smlsharp{} installation destination directory.
Let $\mathit{prefix}$ be the path to the directory.
\item
	In the \smlsharp{} source directory, execute {\tt configure} script.
	In a 64 bit OS,set {\tt CC}, {\tt C++} and {\tt LD} to a 32 bit
C/C++ compiler and a 32 bit linker, respectively.
	To enable multithread (POSIX thread) support, give
{\tt --enable-thread} switch to {\tt ./configure} script.

In i386 Linux, you would do the following:
\begin{program}
\$ ./configure --prefix=$\mathit{prefix}$
\end{program}
In amd64 Linux:
\begin{program}
\$ ./configure CC='gcc -m32' CXX='g++ -m32' LD='ld -m elf\_i386' --prefix=$\mathit{prefix}$
\end{program}

\item
	Do the following:
\begin{program}
\$ make\\
\$ make install
\end{program}
\end{enumerate}

	The following files are installed.
\begin{enumerate}
\item {\tt smlsharp} command at {\tt $\mathit{prefix}$/bin/smlsharp}
\item {\tt smlformat} command at {\tt $\mathit{prefix}$/bin/smlformat}
\item {\tt smllex} command at {\tt $\mathit{prefix}$/bin/smllex}
\item {\tt smlyacc} command at {\tt $\mathit{prefix}$/bin/smlyacc}
\item library files in the {\tt $\mathit{prefix}$/lib/smlsharp/} directory.
\end{enumerate}
	If successful, you can invoke \smlsharp{} by typing:
\begin{program}
\$ $\mathit{prefix}$/bin/smlsharp
\end{program}

Some hints:
\begin{itemize}
\item 
	This process compiles all the source files including those of
tools, which takes some time.
	If you have a CPU with $n$ cores and use GNU make, then try to
give {\tt -j$m$} ($m \le n$) switch to {\tt make} command for
parallel processing, where $m$ indicate the degree of parallelism. 
\end{itemize}
\fi%%%%<<<<<<<<<<<<<<<<<<<<<<<<<<<<<<<<<<<<<<<<<<<<<<<<<<<<<<<<<<<<<<<<<<

\section{\txt{未サポートのプラットフォーム}{Unsupported platforms}}

\subsection{Windows}

\ifjp%>>>>>>>>>>>>>>>>>>>>>>>>>>>>>>>>>>>>>>>>>>>>>>>>>>>>>>>>>>>>>>>>>>>

	\smlsharp{} \version{}版のWindows上での動作は未確認です.
	将来のリリースで対応の予定です.

\else%%%%%%%%%%%%%%%%%%%%%%%%%%%%%%%%%%%%%%%%%%%%%%%%%%%%%%%%%%%%%%%%%%%%

	\smlsharp{} \version{} does not work on Windows.
	We shall support Windows in a future version.

\fi%%%%<<<<<<<<<<<<<<<<<<<<<<<<<<<<<<<<<<<<<<<<<<<<<<<<<<<<<<<<<<<<<<<<<<

\section{
\txt{\smlsharp{}の対話型モードを使ってみよう}
    {Let's try \smlsharp{} interactive mode}}
\label{sec:tutorialInteractive}

\ifjp%>>>>>>>>>>>>>>>>>>>>>>>>>>>>>>>>>>>>>>>>>>>>>>>>>>>>>>>>>>>>>>>>>>>

	インストールが終了すると,\smlsharp{}コンパイラが{\tt smlsharp}
という名前のコマンドとして使用可能となります.
	このコマンドをコマンドシェルやEmacsのshellモードから起動すること
によって,\smlsharp{}プログラムをコンパイル,実行できます.
	もっとも簡単な使用法は,\smlsharp{}コンパイラを対話モードで実行
することです.
	{\tt smlsharp}を引数なしで起動すると,対話モードの起動要求とみな
し,起動メッセージを表示しユーザからの入力待ちの状態となります.
%%%%%%%%%%%%%%%%%%ToDo: 日付
\begin{tt}
\begin{quote}
\$ smlsharp\\
SML\# version 2.0.0 (2014-03-31 10:23:45 JST) for x86-linux with LLVM 3.4\\
\# 
\end{quote}
\end{tt}
	「{\tt \#\ }」は\smlsharp{}システムが印字するプロンプト文字です. 
	またこの文書では,シェル(コマンド)のプロンプトを「{\tt \$\ }」と仮定します

	この後,コンパイラは以下の処理を繰り返します.
\begin{enumerate}
\item 区切り文字{\tt ;}まで,端末からでプログラムを読み込む.
\item 読み込んだプログラムをコンパイルし,プログラムを呼び出し実行する.
\item プログラムが返す結果を表示する.
\end{enumerate}
	以下に簡単な実行例を示します.
\begin{tt}
\begin{quote}
\# "Hello";\\
val it = "Hello" : string
\end{quote}
\end{tt}
	最初の行がユーザの入力です.
	2行目が,ユーザの入力に対する\smlsharp{}システムの応答です.
	この例のように,コンパイラは,プログラムの実行結果に加えコンパイ
ラが推論した型を表示します.
	さらに,実行結果に名前をつけ,これに続くセッションで利用可能にし
ます.
	ユーザによる指定がなければ,{\tt it}という名前が付けられます.
\else%%%%%%%%%%%%%%%%%%%%%%%%%%%%%%%%%%%%%%%%%%%%%%%%%%%%%%%%%%%%%%%%%%%%

	Include the \smlsharp{} installation directory in your command
load path,  so that you can run \smlsharp{} by the name {\tt smlsharp}.
	When invoked without any parameter, \smlsharp{} stars its
interactive session by printing the following message and waits for your
input.
%%%%%%%%%%%%%%%%%%ToDo: 日付
\begin{program}
\$ smlsharp\\
SML\# version 2.0.0 (2014-03-31 10:23:45 JST) for x86-linux with LLVM 3.4\\
\# 
\end{program}
	The ``{\tt \#\ }'' character is the prompt \smlsharp{}
prints.
	In this document, we write ``{\tt \$\ }'' for the 
shell prompt.

	After this message, the compiler repeats the following steps. 
\begin{enumerate}
\item Read the user input up to ``{\tt ;}''.
\item Compile the input program and execute it.
\item Print the result.
\end{enumerate}
	The following is a simple example.
\begin{tt}
\begin{quote}
\# "Hello world";\\
val it = "Hello world" : string
\end{quote}
\end{tt}
	The first line is the user input.
	The second line is the response of \smlsharp{} system.
	As seen in this example, the compiler prints the result value with
its type, and binds it to a name for subsequent use.
	If the user does not specify the name, the compiler uses  ``{\tt
it}'' as the default name.

\fi%%%%<<<<<<<<<<<<<<<<<<<<<<<<<<<<<<<<<<<<<<<<<<<<<<<<<<<<<<<<<<<<<<<<<<

\section{
\txt
{\smlsharp{}のコンパイルモードを試してみよう}
{Let's try \smlsharp{} compile mode}
}
\label{sec:tutorialCompile}

\ifjp%>>>>>>>>>>>>>>>>>>>>>>>>>>>>>>>>>>>>>>>>>>>>>>>>>>>>>>>>>>>>>>>>>>>
	\smlsharp{}コンパイラは,対話型以外に,Cコンパイラのようにファイ
ルをコンパイルすることができます.

	まず,簡単な例として,前節での入力{\tt \# "Hello world";\\}をファ
イル書いてコンパイルしてみましょう.
	そのために,{\tt hello1.sml}を以下のような内容で作成します.
\begin{program}
"Hello world";
\end{program}
	最後のセミコロンはあってもなくても同じです.
	この作成したファイルは,以下のようにしてコンパイルできます.
\begin{program}
\$ smlsharp hello1.sml\\
\$ file a.out\\
a.out: ELF 32-bit LSB executable, Intel 80386, version 1 (SYSV), dynamically linked (uses shared libs), for GNU/Linux 2.6.18, not stripped\\
\$
\end{program}
	\smlsharp{}は,引数としてファイル名のみが与えられると,Cコンパイ
ラ同様,そのファイルをコンパイル,リンクし実行形式ファイルを作成します.
	以上のようにシステム標準の実行形式ファイルが{\tt a.out}作成され
ていることが確認できます.
	作成されるファイル名は{tt -o}スイッチで指定できます.
\begin{program}
\$ smlsharp -o hello1 hello1.sml\\
\$ ls hello1*
hello1 hellp1.sml
\$ 
\end{program}
	作成したファイルは,通常のコマンドと同じように実行できます.
\begin{program}
\$ ./hello1\\
\$ 
\end{program}
	何も表示されませんが,これで正常です.
	ファイルのコンパイルでは,対話型モードと違い,結果の値と型を表示
するコードは付加されません.

	結果を表示したければ,結果を表示する関数を呼ぶ必要があります.
	そこで,{\tt hello2.sml}を以下のような内容で作成します.
\begin{program}
print "hello world!\verb|\|n"
\end{program}
	このファイルには,このファイルで定義していない{\tt print}関数が
使用されています.
	我々の意図は基本ライブラリで定義された{\tt print : string ->
unit}を使用することですが,{\tt print}は自由に再定義できますから,
\smlsharp{}コンパイラにとっては,名前{\tt print}が
基本ライブラリ関数の{\tt print}を指すとはかぎりません.
	そこでこのファイルをコンパイルするためには,{\tt print}が定義さ
れているファイルをコンパイラに通知する必要があります.

	このようにファイルに定義された名前以外の名前を使用するファイルを
コンパイルするためには,そのファイルで使用する名前が他のソースファイルで
定義されていることを,{\bf インターフェイスファイル}で宣言する必要があり
ます. 
	\smlsharp{}は,ソースファイル名に対応するインターフェイスファイル
{\tt "hello2.smi"}を探します.
	{\tt hello.sml}を実行するためには,以下の{\tt "hello2.smi"}ファ
イルを以下のように作成する必要があります.
\begin{program}
\verb|_|require "basis.smi"
\end{program}
	この宣言は,{\tt hello2.sml}がインターフェイスファイル{\tt
"basis.smi"}で宣言されている資源を必要としていることを表しています.
{\tt "basis.smi"}はStandard MLの基本ライブラリで定義されているすべての名
前が宣言されているインターフェイスファイルです.

	この用意の下で,以下のコマンドを実行すれば,実行形式プログラムが
作成されます.
\begin{program}
\$ smlsharp hello.sml -o hello\\
\$ ./hello\\
hello word!
\end{program}
\else%%%%%%%%%%%%%%%%%%%%%%%%%%%%%%%%%%%%%%%%%%%%%%%%%%%%%%%%%%%%%%%%%%%%
	The main function of \smlsharp{} compiler  is to compile a file
and to create an executable program, just like {\tt gcc}.

	As a simple example, let us write the previous user input
{\tt \# "Hello world";} into a file {\tt hello1.sml}:
\begin{program}
"Hello world";
\end{program}
	The trailing semicolon does not have any effect in a file.
	This file can be compiled as follows.
\begin{program}
\$ smlsharp hello1.sml\\
\$ file a.out\\
a.out: ELF 32-bit LSB executable, Intel 80386, version 1 (SYSV), dynamically linked (uses shared libs), for GNU/Linux 2.6.18, not stripped\\
\$
\end{program}
	When only a file name is given, \smlsharp{} compiles the file,
link it and creates an executable file, whose default name is {\tt
a.out}.
	As shown by the Linux {\tt file} command, it is an ordinary
executable file.
	The output file name can be specified by a {\tt -o} switch.
\begin{program}
\$ smlsharp -o hello1 hello1.sml\\
\$ ls hello1*\\
hello1 hellp1.sml\\
\$ 
\end{program}
	The generated executable file can then be executed.
\begin{program}
\$ ./hello1\\
\$ 
\end{program}
	Nothing is printed by this program.
	This is what you expect.
	\smlsharp{} compiles the source code itsel; it does not attach
code to print the result value and its type.
	If you want to see something printed, you need to write code
to print it explicitly.
	Now let's create a file {\tt hello2.sml} to print this message.
	The contents can be the following.
\begin{program}
print "hello world!\verb|\|n"
\end{program}
	This file contains a function {\tt print} which is not defined
in this file.
	Our intention is that {\tt print} denotes the function
{\tt print : string -> unit} in Standard ML Basis Library.
	However, since name {\tt print} can be freely re-defined,
for \smlsharp{} compiler, it does not necessarily mean the library
function.
	In order to compile {\tt hello2.sml}, you need to notify the
compiler where the name {\tt print} is defined.
	For this purpose, an {\bf interface file} must be created.

	When compiling {\tt hello2.sml}, \smlsharp{} searches for 
its interface file by its default name {\tt hello2.smi}.
	So to compile {\tt hello2.sml}, you need to create file {\tt
hello2.smi}.
	Its contents can be the following.
\begin{program}
\verb|_|require "basis.smi"
\end{program}
	This declares that {\tt hello2.sml} uses names that are defined
in the interface file {\tt "basis.smi"}, which is the system supplied
file containing the declarations of all the names defined in the
Standard ML Basis Library.

	With this preparation, {\tt hello2.sml} is compiled as follows.
\begin{program}
\$ smlsharp hello.sml -o hello\\
\$ ./hello\\
hello word!\\
\end{program}
\fi%%%%<<<<<<<<<<<<<<<<<<<<<<<<<<<<<<<<<<<<<<<<<<<<<<<<<<<<<<<<<<<<<<<<<<

\section{
\txt{{\tt smlsharp}コマンドの起動モード}
    {{\tt smlsharp} command modes}
}
\label{sec:tutorialSmlsharpParameter}

\ifjp%>>>>>>>>>>>>>>>>>>>>>>>>>>>>>>>>>>>>>>>>>>>>>>>>>>>>>>>>>>>>>>>>>>>
	{\tt smlsharp}の主な実行モードは以下の4つです.
\begin{description}
\item[対話型モード]
{\tt smlsharp}をパラメタなしに起動すると,対話型セッションを実行します.
\item[コンパイル・リンクモード]
一つのソースファイルを指定して起動すると
そのファイルをコンパイルし,実行形式プログラムを作成します.
\item[コンパイルモード]
{\tt -c}スイッチを指定して起動すると,指定されたソースファイルをオブジェ
クトファイルにコンパイルします.
\item[リンクモード]
一つのインターフェイスファイル({\tt smi}ファイル)
を指定して起動すると,インターフェイスファイルに対応するソースファイルと
インターフェイスで参照されたすべてのソースファイルがコンパイル済みとみなし,
それらすべてのオブジェクトファイルをリンクし,実行形式プログラムを作成し
ます.
\end{description}
	これら実行を制御する起動スイッチとパラメタには以下のものが含まれ
ます.
\begin{description}
\item[--help] ヘルプメッセージをプリントし終了.
\item[-v] 種々のメッセージを表示.
\item[-o \nonterm{file}] 出力ファイル(オブジェクトファイル,実行形式ファイル)の名前を指定.
\item[-c]  コンパイルしオブジェクトファイルを生成.
\item[-S]  コンパイルしアセンブリファイルを作成.
\item[-M] コンパイルの依存関係を表示
\item[-MM] システムライブラリを除いたコンパイルの依存関係を表示
\item[-Ml] リンクの依存関係を表示
\item[-MMl] システムライブラリを除いたリンクの依存関係を表示
\item[-I \nonterm{dir}] ソースファイルのサーチパスの追加
\item[-L \nonterm{dir}] リンカーへのライブラリファイルのサーチパスの追加
\item[-l \nonterm{libname}]  リンク時にライブラリ<libname>をリンク
\item[-Wl,\nonterm{args}]  リンカーへ<args>をパラメタとして渡す
\end{description}
\else%%%%%%%%%%%%%%%%%%%%%%%%%%%%%%%%%%%%%%%%%%%%%%%%%%%%%%%%%%%%%%%%%%%%
	{\tt smlsharp} has the following execution modes.
\begin{description}
\item[interactive mode]
When {\tt smlsharp} is invoked without any parameter as shown below:
\begin{program}
\$ smlsharp\\
SML\# version 1.0.1 (2012-04-24 10:23:45 JST) for x86-linux\\
\# 
\end{program}
it executes the interactive session.
\item[compile and link mode]
When {\tt smlsharp} is invoked with a single source file as shown below:
\begin{program}
\$ smlsharp \nonterm{file}.sml\\
\$ 
\end{program}
it compiles the source file {\tt \nonterm{file}.sml}, links the resulting
object file and creates an executable program (with the name {\tt a.out}
by default).
\item[compile mode]
When {\tt smlsharp} is invoked with {\tt -c} switch as shown below:
\begin{program}
\$ smlsharp -c \nonterm{file1}.sml $\cdots$ \nonterm{fileN}.sml \\
\$ 
\end{program}
it compiles the given files {\tt \nonterm{file1}.sml $\cdots$
\nonterm{fileN}.sml}
into object files {\tt \nonterm{file1}.o $\cdots$
\nonterm{fileN}.o}.
\item[link mode]
When {\tt smlsharp} is invoked with a single interface files as shown below:
\begin{program}
\$ smlsharp \nonterm{file}.smi \\
\$ 
\end{program}
it assumes that {\tt \nonterm{file}.smi} is a top-level interface file,
links all the object files corresponding to the interface files
referenced from {\tt \nonterm{file}.smi}, 
and generates an executable file.
\end{description}
	\smlsharp{} also recognizes various command-line switches.
\begin{program}
\$ smlsharp --help
\end{program}
will print a help message describing available switches.
\fi%%%%<<<<<<<<<<<<<<<<<<<<<<<<<<<<<<<<<<<<<<<<<<<<<<<<<<<<<<<<<<<<<<<<<<

\chapter{\txt{MLプログラミング入門}{Introduction to ML programming}}
\label{chap:tutorialMlprogramming}

\ifjp%>>>>>>>>>>>>>>>>>>>>>>>>>>>>>>>>>>>>>>>>>>>>>>>>>>>>>>>>>>>>>>>>>>>
	\smlsharp{}言語は,Standard ML言語と後方互換性のあるプログラミン
グ言語です.
	\smlsharp{}の高度な機能を使いこなすために,本章でまず,Standard
MLプログラミングの基礎を学びましょう.

% 現在この章は未完です.
% 第\ref{sec:tutorialPolymorphicfunction}節に続き近い将来,以下の
% 節を追加する予定です.
% \begin{quote}
% パターンマッチングによる関数定義\\
% リストプログラミング\\
% 高階のリスト処理関数\\
% データ型の定義\\
% パターンマッチングによるデータ型の利用\\
% モジュールシステム\\
% Structure\\
% local宣言\\
% signature制約\\
% 種々のデータ型ライブラリ\\
% Functor
% \end{quote}

\else%%%%%%%%%%%%%%%%%%%%%%%%%%%%%%%%%%%%%%%%%%%%%%%%%%%%%%%%%%%%%%%%%%%%
	The \smlsharp{} language is a super set of Standard ML.
	To write a program in \smlsharp{} exploiting its advanced
features, you must first learn Standard ML programming.
	This chapter provides ML programming tutorial for those who
first learn ML programming.

% \begin{small}
% This chapter is incomplete
% 
% The following sections will be added in near future.
% \begin{quote}
% list programming\\
% higher-order functions for lists\\
% data type definitions\\
% pattern matching \\
% overview of the module system\\
% Structure\\
% local declaration\\
% signature constraint\\
% various data types and libraries\\
% functors
% \end{quote}
% \end{small}

\fi%%%%<<<<<<<<<<<<<<<<<<<<<<<<<<<<<<<<<<<<<<<<<<<<<<<<<<<<<<<<<<<<<<<<<<

\section{\txt{ML言語について}{About the ML language family}}
\label{sec:tutorialMllanguage}

\ifjp%>>>>>>>>>>>>>>>>>>>>>>>>>>>>>>>>>>>>>>>>>>>>>>>>>>>>>>>>>>>>>>>>>>>
	\smlsharp{}はML系関数型プログラミング言語の一つです.

	ML言語はEdinburgh LCF\cite{gord79}のメタ言語({\bf M}eta {\bf
L}anguage)として開発されました.
	メタという接頭語のこの特別な使われ方は,ギリシャ語の``ta meta ta
phusika''という用例に遡るといわれています.
	このフレーズは単に,自然学(physics)のあとに置かれた名前が付
けられていない本を示すギリシャ語でしたが,その本の内容が哲学(形而上学,
metaphysics)であったことから,このメタという接頭語に,通常の思考の次元
をこえて物事を分析する哲学的な態度,という意味が与えられるようになりまし
た.
	言語学の文脈では,メタ言語とは,ある言語の分析や記述をするための
言語を意味します.
	プログラミング言語を変換したり処理するための言語もメタ言語とみな
すことができます.
	Edinburgh LCFは計算可能な関数(つまりプログラム)を対象とした一
種の定理証明システムであり,その対象となる関数はPPLAMBDAと呼ばれる型付ラ
ムダ計算で表現されました.
	LCF MLはPPLAMBDAのメタ言語,つまりPPLAMBDAで書かれたプログラムを操
作しするためのプログラミング言語でした.
	MLの名前はこの歴史的事実に由来しています.
	関数を表すラムダ式を柔軟に操作する目的のために,ML自身ラムダ計
算を基礎とする関数型言語として設計されました.
	さらにプログラムを操作するプログラムを柔軟にかつ信頼性を持って記
述するために,型の整合性を自動的に検査する型推論機構が初めて開発されまし
た.

	このLCF MLは,PPLAMBDAの操作言語にとどまらない汎用のプロ
グラミング言語として価値があることが認識され,CardelliによってLCFとは独
立したプログラミング言語としてのMLコンパイラが開始され,PDP11やVAX VMSな
どの上に実装され使用されました.
	その後,Milner,Harper,MacQueen等によってプログラミング言語とし
ての仕様策定の努力がなされ,さらにこの言語がEdinburgh大学によってCardelli
のML上に実装されました.
	これらの成果を踏まえ,Standard MLの仕様\cite{sml}が確定し,その
後改訂され現在のStandard ML言語仕様\cite{sml97}として完成しました.
\else%%%%%%%%%%%%%%%%%%%%%%%%%%%%%%%%%%%%%%%%%%%%%%%%%%%%%%%%%%%%%%%%%%%%
	\smlsharp{} is a programming language in the ML family.

	ML was first developed as a {\bf M}meta {\bf L}anguage of
Edinburgh LCF\cite{gord79} system.
	The prefix {\em meta\/} in this particular usage probably traces
back to Greek phrase ``ta meta ta phusika'', which  simply denotes the
book placed after the phusika (Physics, or the book on the nature).
	The denoted book happens to be on philosophy (metaphysics), which
made this prefix receive a meaning of the (philosophical) attitude of
analyzing things beyond the ordinary way of thinking.
	In the context of languages, a meta language is a language used
to analyze the ordinary usage (writing/reading/speaking) of a language.
	Objects of LCF system are computable functions (programs)
represented by a language called PPLAMBDA.
	LCF ML was the meta language of PPLAMBDA, namely, a programming
language to manipulate PPLAMBDA expressions (functions).

	ML's name as well as its features are originated from this
historical role of LCF ML. 
	For flexible manipulation of programs, LCF ML itself was defined
as a functional language.
	In addition, a type inference system was introduced for reliable
manipulation of function terms.
	Since the success of LCF ML, ML has evolved as a general purpose
programming language, and several compilers have been developed for ML.
	Based on these efforts, its specification is formally defined as
Standard ML \cite{sml,sml97}.
\fi%%%%<<<<<<<<<<<<<<<<<<<<<<<<<<<<<<<<<<<<<<<<<<<<<<<<<<<<<<<<<<<<<<<<<<


\section{\txt{宣言的プログラミング}{Declarative programming}}
\label{sec:tutorialDeclarative}

\ifjp%>>>>>>>>>>>>>>>>>>>>>>>>>>>>>>>>>>>>>>>>>>>>>>>>>>>>>>>>>>>>>>>>>>>
	時折,関数型プログラミングは「宣言的」であると言われます.
	この言葉は厳密な技術的用語ではなく,プログラムが実現したいことそ
のものの記述するといった意味の曖昧性のある言葉です.
	しかし,これから書こうとするプログラムが表現すべきものがあらかじ
め理解されている場合,宣言的な記述は,より明確な意味を持ちます.
	
	書こうとするプログラムが入力を受け取り出力を返す関数と理解できる
場合,そのプログラムを,入力と出力の関係を表す関数として直接表現できれば,
より宣言的記述となり,したがってより分かりやすく簡潔なプログラムとなるは
ずです.
	MLは,この意味でより宣言的な記述を可能にするプログラミング言語と
いえます.
	MLプログラミングをマスターする鍵の一つは,この意味での宣言的プロ
グラミングの考え方を身につけることです.

	簡単な例として,数学の教科書で学んだ階乗を計算するプログラムを考
えてみましょう.
	自然数$n$の階乗$n !$は,以下のように定義されます.
\begin{eqnarray*}
0 ! &=& 1
\\
n ! &=& n \times (n - 1) !
\end{eqnarray*}
MLでは,このプログラムを以下のように記述できます.
\begin{tt}
\begin{quote}
fun fact 0 = 1\\
\ \ \ \ \ | fact n = n * fact (n - 1)
\end{quote}
\end{tt}
	この宣言によって{\tt fact}と言う名前の関数が定義されます.
	この定義は``{\tt |}''によって区切られた2つの場合からなっています.
	上段は,引数が{\tt 0}の場合は{\tt 1}を返すことを表し,下段はそれ
以外の場合,引数{\tt n}と{\tt n - 1}の階乗とを掛けた結果を返すことを表し
ています.
	それぞれが,上記の階乗の定義とほぼ完全に一致していることがおわかりでしょう.
	
	このような単純な関数に限らず,種々のプログラムを,その意図する意
味に従い宣言的な関数として表現できれば,複雑なプログラムをより簡潔かつ分
かりやすく書き下すことができるかもしれません.
	プログラミング言語MLは,この考え方に従い,大規模で複雑な計算をプ
ログラムする言語と言えます.
	以下に続く節で,その基本となる考え方を学んでいきましょう.
\else%%%%%%%%%%%%%%%%%%%%%%%%%%%%%%%%%%%%%%%%%%%%%%%%%%%%%%%%%%%%%%%%%%%%
	Functional programming is sometimes described as ``declarative
programming''.
	This is not a precise technical term.
	It somewhat vaguely suggests that programs naturally and
directly express what they should realize.
	If the problem to be solved is procedural in nature then a
procedural description may well be declarative.
	However, when the thing to be represented by a program has a
well defined meaning, then declarative programming has more precise
meaning.
	
	When a program to be written is best understood as a function
that takes an input and returns a result, then functional programming
would be naturally declarative.
	As a simple example, consider the factorial function.
	The factorial of $n$,  $n !$, is defined by the following mathematical
equations.
\begin{eqnarray*}
0 ! &=& 1
\\
n ! &=& n \times (n - 1) !
\end{eqnarray*}
In ML, it is coded below.
\begin{tt}
\begin{quote}
fun fact 0 = 1\\
\ \ \ \ \ | fact n = n * fact (n - 1)
\end{quote}
\end{tt}
	This declaration defines a function named {\tt fact}.
	It consists of two cases separated by ``{\tt |}''.
	The first case says {\tt fact} returns {\tt 1} if the parameter
is {\tt 0}, and the second case says it otherwise returns the multiple
of {\tt n} and the factorial of {\tt n - 1}.
	These two cases exactly correspond to the mathematical
definition of this function.
	
	
	Various programs other than such a very simple mathematical
function can be written in a concise and readable fashion if you
can represent them as declarative functions according to the intended
meaning.
	ML is a programming language that promotes this way of declarative
programming.
	A key to master ML programming is to obtain a skill of writing
declarative code in this sense.
	In the subsequent sections, we learn the essence of ML
programming.
\fi%%%%<<<<<<<<<<<<<<<<<<<<<<<<<<<<<<<<<<<<<<<<<<<<<<<<<<<<<<<<<<<<<<<<<<

\section{
\txt{式の組み合わせによる計算の表現}
{Representing computation by composing expressions}
}
\label{sec:tutorialExpression}

\ifjp%>>>>>>>>>>>>>>>>>>>>>>>>>>>>>>>>>>>>>>>>>>>>>>>>>>>>>>>>>>>>>>>>>>>
	前節の階乗の計算では,引数が{\tt 0}の場合は{\tt 1}を返し,それ以
外の一般の{\tt n}の場合は,{\tt n * fact (n - 1)}を返すようにプログラム
されていました.
	このように,関数が返すべき値は,値を表す式で表現されます.
	最初の場合の{\tt 1}も,値そのものではなく,ゼロという値を表す式
とみなします.
	{\tt n * fact (n - 1)}は,変数{\tt n}と関数を含む式です.
	一般にML言語の大原則は,
\begin{quote}
{\bf MLのプログラミングは,必要な値をもつ式を定義することを通じて行う}
\end{quote}
というものです.

	式は以下の要素を組み合わせて構成されます.
\begin{enumerate}
\item {\tt 0}などの定数式
\item 関数の引数やすでに定義されている値を表す変数
\item 関数呼び出し
\item 関数を含む種々のデータ構造の構成
\end{enumerate}
	1から3までの組み合わせは,高校などで慣れ親しんだ算術式と同じ構
造を持っています.
	例えば,$1$から$n$までの自然数の2乗の和
$S_n = 1^2 + 2^2 + \cdots + n^2$
は以下の公式で求められます.
\[
S_n = \frac{n (n + 1) (2n + 1)}{6}
\]
この式は,MLで直接以下のようにプログラムできます.
\begin{program}
val Sn = (n * (n + 1) * (2*n + 1)) div 6
\end{program}
	{\tt *}と{\tt div}はそれぞれ自然数の乗算と除算を表します.
	この式は$n$が定義されていれば,正しく$S_n$の値を計
算します.
\begin{program}
\# val n = 10;\\
val n = 10 : int\\
val Sn = (n * (n + 1) * (2*n + 1)) div 6;\\
val Sn = 385 : int
\end{program}

\else%%%%%%%%%%%%%%%%%%%%%%%%%%%%%%%%%%%%%%%%%%%%%%%%%%%%%%%%%%%%%%%%%%%%
	In the factorial example, the program returns {\tt 1} 
if the parameter is {\tt 0},and returns {\tt n * fact (n - 1)}
otherwise.
	In this way, the value returned by a function is represented by
an {\em expression}.
	{\tt 1} in the first case is an expression representing the
natural number $1$.
	{\tt n * fact (n - 1)} is an expression involving variable {\tt
n} and function application.
	The basic principle of ML programming is 
\begin{quote}
{\bf programming is done by defining an expression that represents the desired value}.
\end{quote}

	An expression consist of the following components:
\begin{enumerate}
\item constants such as {\tt 1},
\item variables representing function parameters and defined values.
\item function applications (function calls), and
\item functions and other data structure constructions
\end{enumerate}
	The items 1 through 3 are the same as in mathematical expressions,
we have leaned in school.
	For example, the sum of the arithmetic sequence 
$S_n = 1^2 + 2^2 + \cdots + n^2$
is given by the following equation.
\[
S_n = \frac{n (n + 1) (2n + 1)}{6}
\]
	This is directly programmed as follows.
\begin{program}
val Sn = (n * (n + 1) * (2*n + 1)) div 6
\end{program}
	{\tt *} and {\tt div} are integer multiplication and division,
respectively.
	When $n$ is defined, it correctly computes $S_n$ as seen below.
\begin{program}
\# val n = 10;\\
val n = 10 : int\\
val Sn = (n * (n + 1) * (2*n + 1)) div 6;\\
val Sn = 385 : int
\end{program}
% 
% \begin{tt}
% \begin{quote}
% \# val pi = 3.14;\\
% val pi = 3.14 : real\\
% \# val r = 1.5;\\
% val r = 1.5 : real\\
% \# 2.0 * pi * r;\\
% val it = 9.41999999999 : real\\
% \# pi * r * r;\\
% val it = 7.06499999999 : real\\
% \# r * (Math.sin (pi / 6.0));\\
% val it = 0.749655153964 : real\\
% \# Math.sqrt 2.0;\\
% val it = 1.414213562373 : real\\
% \# it  = Math.pow (2.0, 3.0);\\
% val it = 8.0 : real\\
% \# val negative = \verb|~|1.0;\\
% val negative = \verb|~|1.0 : real
% \end{quote}
% \end{tt}
% 	{\tt Math.sqrt $n$} and {\tt Math.pow($n$,$m$)} are primitive
% functions to compute $\sqrt{n}$ and $n^m$ respectively.
% 	As in the last example, a negative constant is written by
% prefixing the tilde symbol ``{\tt \verb|~|}''.

\fi%%%%<<<<<<<<<<<<<<<<<<<<<<<<<<<<<<<<<<<<<<<<<<<<<<<<<<<<<<<<<<<<<<<<<<

\section{
\txt{定数式と組込み関数}
{Constants literals and built-in primitives}
}
\label{sec:tutorialConstants}

\ifjp%>>>>>>>>>>>>>>>>>>>>>>>>>>>>>>>>>>>>>>>>>>>>>>>>>>>>>>>>>>>>>>>>>>>
	型付き言語であるMLでは,定数式はすべて決まった型を持ちます.
	MLでよく使われる組込み型には以下のものがあります.

\begin{center}
\begin{tabular}{|l|l|}
\hline
型 & 内容
\\ \hline
int & (マシン語1ワードの大きさの)符号付き整数
\\ \hline
real & 倍精度浮動小数点
\\ \hline
char & 文字型
\\ \hline
string & 文字列型
\\ \hline
word & (マシン語1ワードの大きさの)符号無し整数
\\ \hline
bool & 真理値型
\\ \hline
\end{tabular}
\end{center}
	真理値型以外は,他のプログラミング言語と考え方は同じです.
	これらの組込み方の定数式と基本組込み演算を覚えましょう.
	真理値型の扱いは次節で説明します.	

\subsection{int型}
	{\tt int}型は整数型です.
	\smlsharp{}では,C言語の\code{long}と同一の2の補数表現で表され
る1語長の符号付き整数です.
	{\tt int}型定数には通常の十進数の記述以外に{\tt 0x\nonterm{h}}形式の16進
数でも記述できます.
	\nonterm{h}は{\tt 0}から{\tt 9}の数字と{\tt a}({\tt A}でもよい)
から{\tt f}({\tt F}でもよい)までのアルファベットの並びです.
	負数の定数はチルダ記号\verb|~|を付けて表します.
	\smlsharp{}は{\tt int}型の値を十進数で表示します.
\begin{program}
\# 10;\\
val it = 10 : int\\
\# 0xA;\\
val it = 10 : int\\
\# \verb|~|FF;\\
val it = \verb|~|255 : int
\end{program}
	{\tt int}型に対しては,通常の四則演算{\tt +, -, *, div}が2項演算
子として定義されています.
\begin{program}
\# 10 + 0xA;\\
val it = 20 : int\\
\# 0 - 0xA;\\
val it = \verb|~|10 : int\\
\# 10 div 2;\\
val it = 5 : int\\
\# 10 * 2;\\
val it = 20 : int
\end{program}
	これらの組合せは,通常の算術演算の習慣に従い,{\tt *,div}が{\tt
+, -}より強く結合します.
	また,同一の演算子は左から順に計算されます.
\begin{program}
\# 1 + 2 * 3;\\
val it = 7 : int\\
\# 4 - 3 - 2;\\
val it = \verb|~|1 : int
\end{program}

\subsection{real型}
	{\tt real}は浮動小数点データの型です.
	\smlsharp{}では,C言語の\code{double}に相当する64ビットの倍精
度浮動小数点です.
	{\tt real}型の定数は{\tt \nonterm{s}.\nonterm{d}E\nonterm{s}}
の形で表現します.
	ここで\nonterm{s}は符号付き十進数,\nonterm{d}は符号なし十進数で
す.
	整数部,小数部,指数部のいずれも省略できますが.小数部と指数部の
両方を省略すると{\tt real}型ではなく{\tt int}型と解釈されます.
\begin{program}
\# 10.0;\\
val it = 10.0 : real\\
\# 1E2;\\
val it = 10.0 : real\\
\# .001;\\
val it = 0.001 : real\\
\# \verb|~|1E\verb|~|2;\\
val it = 0.001 : real\\
\# 10;\\
val it = 10 : int
\end{program}
	{\tt real}型に対しても{\tt int}型同様に四則演算が定義されていま
す.
	ただし,除算は{\tt div}ではなく{\tt /}です.
\begin{program}
\# 10.0 + 1E1;\\
val it = 20.0 : real\\
\# 0.0 - 0xA;\\
val it = \verb|~|10.0 : real\\
\# 10.0 / 2.0;\\
val it = 5.0 : real\\
\# 10.0 * 2.0;\\
val it = 20.0 : real
\end{program}

\subsection{char型}
	{\tt char}は1文字を表すデータ型です.
	文字\nonterm{c}を表す{\tt char}型定数は{\tt \#"\nonterm{c}"}と書きます.
	\nonterm{c}には通常文字以外に以下の特殊文字のエスケープが書けます.
\begin{center}
\begin{tabular}{|l|l|}
\hline\\
\verb|\a| & {warning (ASCII 7)}\\
\verb|\b| & {backspace (ASCII 8)}\\
\verb|\t| & {horizontal tab(ASCII 9)}\\
\verb|\n| & {new line (ASCII 10)}\\
\verb|\v| & {vertical tab (ASCII 11)}\\
\verb|\f| & {home feed (ASCII 12)}\\
\verb|\r| & {carrige return(ASCII 13)}\\
\verb|\^|$C$ & {control character $C$}\\
\verb|\"| & {character {\tt \verb|"|}}\\ %"
\verb|\\| & {character \verb|\|}\\
\verb|\|$ddd$ &  {the character whose code is $ddd$ in decimal}
\\\hline
\end{tabular}
\end{center}
\begin{program}
\# \#"a";\\
val it = \#"a" : char\\
\# \#"\verb|\|n";\\
val it = \#"\verb|\|n" : char
\end{program}
	文字型に対して以下の組込み演算が定義されています.
\begin{program}
val chr : int -> char\\
val ord : char -> int
\end{program}
	{\tt chr}は与えられた数字をASCIIコードとみなし,そのコードの文字
を返します.
	{\tt ord}は文字型データのASCIIコードを返します.
\begin{program}
\# ord \#"a";\\
val it = 97 : int\\
\# chr(97 + 7);\\
val it = \#"h" : char\\
\# ord \#"a" - ord \#"A";\\
val it = 32 : int\\
\# chr(ord \#"H" + 32);\\
val it = \#"h" : char
\end{program}

\subsection{string 型}
	{\tt string}は文字列を表すデータ型です.
	{\tt string}定数は{\tt "}と{\tt "}で囲んで文字列表します.
	この文字列の中に特殊文字を含める場合は,{\tt char}型で定義され
ているエスケープシーケンスを使います.
	{\tt string}データには,標準出力にプリントする{\tt print}と
2つの文字列を連結する{\tt \verb|^|}が組込み関数として定義されています.
\begin{program}
\# "SML" \verb|^| "\verb|\n|";
val it = "SML\#\verb|\n|" : string\\
\# print it;\\
SML\#\\
val it = () : unit
\end{program}

\subsection{word 型}
	{\tt word}は符号なし整数型です.
	{\tt word}型定数は{\tt 0w\nonterm{d}}または{\tt 0wx\nonterm{h}}
の形で表現します.
	ここで\nonterm{d}はそれぞれ十進数,\nonterm{h}は16進の数字列です.
	\smlsharp{}は{\tt word}型定数を16進表現で表示します.
\begin{program}
\# 0w10;\\
val it = 0xA : word\\
\# 0wxA;\\
val it = 0xA : word
\end{program}
	{\tt word}型データに対しても{\tt int}型と同じ四則演算が定義され
ています.
	さらに{\tt word}型データに対しては,種々のビット演算が定義されて
います.

\else%%%%%%%%%%%%%%%%%%%%%%%%%%%%%%%%%%%%%%%%%%%%%%%%%%%%%%%%%%%%%%%%%%%%
	\smlsharp{} cointains the following atomic types, with the
associated literal constants.

\begin{center}
\begin{tabular}{|l|l|}
\hline
type & its values
\\ \hline
{\tt int} & signed integres (of one machine word size)
\\ \hline
{\tt real} & double presicion floating point numbers
\\ \hline
{\tt char} & characters
\\ \hline
{\tt string} & character strings
\\ \hline
{\tt word} & unsigned intergers (of one machine word size)
\\ \hline
bool & boolean values
\\ \hline
\end{tabular}
\end{center}

	We review basis expressions for these atomic types, except for
{\tt bool} which will be treated in the next section.

\subsection{int type}
	This is a standard type of integers.
	In \smlsharp{}, a value of {\tt int} is a two's compliment
integer and is the same as \code{long} in C.
	Constants of {\tt int} are written either the usual decimal
notation or hexadecimal notation of the form {\tt 0x\nonterm{h}} 
where \nonterm{h} is a sequence of digits from {\tt 0} to {\tt 9} and
letters from {\tt a}(or {\tt A}) to {\tt f}(or {\tt F}).
	A negative constant is written with \verb|~|.
	\smlsharp{} prints {\tt int} values in decimal notation.
\begin{program}
\# 10;\\
val it = 10 : int\\
\# 0xA;\\
val it = 10 : int\\
\# \verb|~|FF;\\
val it = \verb|~|255 : int
\end{program}
	Binary arithmetic operations {\tt +, -, *, div} are defined on 
{\tt int}.
\begin{program}
\# 10 + 0xA;\\
val it = 20 : int\\
\# 0 - 0xA;\\
val it = \verb|~|10 : int\\
\# 10 div 2;\\
val it = 5 : int\\
\# 10 * 2;\\
val it = 20 : int
\end{program}
	As in arithmetic expressions in mathematics, {\tt *,div}
associate stronger than {\tt +, -}, and operators of the same strength
are computed from left to right.
\begin{program}
\# 1 + 2 * 3;\\
val it = 7 : int\\
\# 4 - 3 - 2;\\
val it = \verb|~|1 : int
\end{program}

\subsection{real type}
	{\tt real} is a type of floating point data.
	In \smlsharp{}, a value of {\tt real} is a 64 bit double
precision floating point numbers and is the same as \code{double} in C.
	{\tt real} constants are written in the notation {\tt
\nonterm{s}.\nonterm{d}E\nonterm{s}}, where \nonterm{s} is a decimal
string possibly with the negation symbol and \nonterm{d} is a decimal
string.
	Each of the three parts can be omitted, but decimal string is
interpreted as a value of  {\tt int}.
\begin{program}
\# 10.0;\\
val it = 10.0 : real\\
\# 1E2;\\
val it = 10.0 : real\\
\# .001;\\
val it = 0.001 : real\\
\# \verb|~|1E\verb|~|2;\\
val it = 0.001 : real\\
\# 10;\\
val it = 10 : int
\end{program}
	Binary arithmetic operators {\tt +, -, *, /} are defined for {\tt real}.
	Note that the division operator on {\tt real} is {\tt /} and not
{\tt div}.
\begin{program}
\# 10.0 + 1E1;\\
val it = 20.0 : real\\
\# 0.0 - 0xA;\\
val it = \verb|~|10.0 : real\\
\# 10.0 / 2.0;\\
val it = 5.0 : real\\
\# 10.0 * 2.0;\\
val it = 20.0 : real
\end{program}

\subsection{char type}
	{\tt char} is a type for one character data.
	A constant for a character \nonterm{c} is written as {\tt \#"\nonterm{c}"}.
	\nonterm{c} may be any printable character or one of the
following special secape sequences.
\begin{center}
\begin{tabular}{|l|l|}
\hline\\
\verb|\a| & {warning (ASCII 7)}\\
\verb|\b| & {backspace (ASCII 8)}\\
\verb|\t| & {horizontal tab(ASCII 9)}\\
\verb|\n| & {new line (ASCII 10)}\\
\verb|\v| & {vertical tab (ASCII 11)}\\
\verb|\f| & {home feed (ASCII 12)}\\
\verb|\r| & {carrige return(ASCII 13)}\\
\verb|\^|$C$ & {control character $C$}\\
\verb|\"| & {character {\tt \verb|"|}}\\ %"
\verb|\\| & {character \verb|\|}\\
\verb|\|$ddd$ &  {the character whose code is $ddd$ in decimal}
\\\hline
\end{tabular}
\end{center}
\begin{program}
\# \#"a";\\
val it = \#"a" : char\\
\# \#"\verb|\|n";\\
val it = \#"\verb|\|n" : char
\end{program}
	The following primitive functions are defined on {\tt char}.
\begin{program}
val chr : int -> char\\
val ord : char -> int
\end{program}
	{\tt chr $n$} assumes that $n$ is an ASCII code and returns the
character of that code.
	{\tt ord $c$} returns the ASCII code of $c$.
\begin{program}
\# ord \#"a";\\
val it = 97 : int\\
\# chr(97 + 7);\\
val it = \#"h" : char\\
\# ord \#"a" - ord \#"A";\\
val it = 32 : int\\
\# chr(ord \#"H" + 32);\\
val it = \#"h" : char
\end{program}

\subsection{string type}
	{\tt string} is a type for character strings.
	{\tt string} constants are written with {\tt "} and {\tt "}.
	A {\tt string} constant may contain escape sequences shown
above.
	For {\tt string} data, a print function {\tt print} and a
concatenation binary operator {\tt \verb|^|} are defined.
\begin{program}
\# "SML" \verb|^| "\verb|\n|";
val it = "SML\#\verb|\n|" : string\\
\# print it;\\
SML\#\\
val it = () : unit
\end{program}

\subsection{word type}
	{\tt word} is a type for unsigned integers.
	{\tt word} constants are written as {\tt 0w\nonterm{d}} or
{\tt 0wx\nonterm{h}} where \nonterm{d} is a decimal digit sequence and
\nonterm{h} is a hexadecimal digit sequence.
	\smlsharp{} prints {\tt word} data in a hexadecimal notation.
\begin{program}
\# 0w10;\\
val it = 0xA : word\\
\# 0wxA;\\
val it = 0xA : word
\end{program}
\fi%%%%<<<<<<<<<<<<<<<<<<<<<<<<<<<<<<<<<<<<<<<<<<<<<<<<<<<<<<<<<<<<<<<<<<

\section{
\txt{{\tt bool}型と条件式}
{Type {\tt bool} and conditional expressions}
}
\label{sec:tutorialConditional}

\ifjp%>>>>>>>>>>>>>>>>>>>>>>>>>>>>>>>>>>>>>>>>>>>>>>>>>>>>>>>>>>>>>>>>>>>
	条件を含む計算も
\begin{quote}
{\bf MLのプログラミングは,必要な値をもつ式を定義することを通じて行う}
\end{quote}
とううMLの原則に従い式で表されます.
	式
\begin{program}
if $E1$ then $E2$ else $E3$
\end{program}
は値を表す式です.
	その値は,以下のように求められます.
\begin{enumerate}
\item 式$E1$を評価し値を求める.
\item もしその値が{\tt true}であれば$E2$を評価しその値を返す.
\item もし$E1$の値が{\tt false}であれば$E2$を評価しその値を返す.
\end{enumerate}
	この$E1$の式が持つ値{\tt true},{\tt false}はそれぞれ「真」と「偽」
を表す型{\tt bool}の2つの定数です.
	{\tt bool}型の値を持つ式は,これら定数の他に,数の大小比較や論理
演算などがあります.
\begin{program}
\# 1 < 2;\\
val it = true : bool\\
\# 1 < 2 andalso 1 > 2;\\
val it = false : bool\\
\# 1 < 2 orelse 1 > 2;\\
val it = true : bool
\end{program}
	条件式は,これら{\tt bool}型を持つ式を使って定義します.
\begin{program}
\# if 1 < 2 then 1 else 2;\\
val it = 1 : int
\end{program}
	この条件式も値を持つ式ですから,他の式と自由に組み合わせることが
できます.
\begin{program}
\# (if 1 < 2 then 1 else 2) * 10;\\
val it = 10 : int
\end{program}
\else%%%%%%%%%%%%%%%%%%%%%%%%%%%%%%%%%%%%%%%%%%%%%%%%%%%%%%%%%%%%%%%%%%%%
	Following the ML principle,
\begin{quote}
{\bf programming is done by defining an expression that represents the desired value}.
\end{quote}
conditional computation is represented as an expression.
	A conditional expression
\begin{program}
if $E1$ then $E2$ else $E3$
\end{program}
is an expression that computes a value in the following steps.
\begin{enumerate}
\item Evaluate $E1$ and obtain a value.
\item If the value is {\tt true} then evaluate $E2$ and return its value.
\item If the value of $E1$ is {\tt false} then evaluate $E3$ return its value.
\end{enumerate}
	{\tt true}, {\tt false} are two constants of type {\tt bool}.
	Expressions of type {\tt bool} contains comparison expressions
and logical operations.
\begin{program}
\# 1 < 2;\\
val it = true : bool\\
\# 1 < 2 andalso 1 > 2;\\
val it = false : bool\\
\# 1 < 2 orelse 1 > 2;\\
val it = true : bool
\end{program}
	Conditional expressions are made up with expressions of type
{\tt bool}.
\begin{program}
\# if 1 < 2 then 1 else 2;\\
val it = 1 : int
\end{program}
	Since this is an expression, it can be combined with other
expressions as show below.
\begin{program}
\# (if 1 < 2 then 1 else 2) * 10;\\
val it = 10 : int
\end{program}
\fi%%%%<<<<<<<<<<<<<<<<<<<<<<<<<<<<<<<<<<<<<<<<<<<<<<<<<<<<<<<<<<<<<<<<<<


\section{
\txt{複雑な式と関数}
{Compound expressions and function definitions}
}
\label{sec:tutorialFunction}

\ifjp%>>>>>>>>>>>>>>>>>>>>>>>>>>>>>>>>>>>>>>>>>>>>>>>>>>>>>>>>>>>>>>>>>>>
	定数,変数,組込み関数の組み合わせだけでは,もちろん,複雑な問題
を解くプログラムを記述することはできません. 
	MLプログラミングの原則は式を書くことによってプログラムすることで
すが,その中で重要な役割を果たす式が関数を表す式です.

	関数は,すでに第\ref{sec:tutorialDeclarative}節で学んだように,
\begin{program}
{\bf fun}\ {\it funName}\ {\it param} = {\it expr}
\end{program}
の構文で定義されます.
	この宣言以降,変数{\it funName}は,引数{\it param}を受け取り,式
{\it expr}の値を計算する関数として使用可能となります.
	例えば$1$から$n$までの自然数の2乗の和の公式は,$n$を受け取る関
数とみなせます.
\[
\mbox{$S(n)$} = \frac{n (n + 1) (2n + 1)}{6}
\]
	この定義は{\tt fun}構文により直接MLのプログラムとして定義できます.
\begin{program}
\# fun S(n) = (n * (n + 1) * (2*n + 1)) div 6;\\
val S = fn : int -> int
\end{program}
	この関数は,以下のように使用できます.
\begin{program}
\# S 10;\\
val it = 385 : int\\
\# S 20;\\
val it = 2870 : int
\end{program}

\else%%%%%%%%%%%%%%%%%%%%%%%%%%%%%%%%%%%%%%%%%%%%%%%%%%%%%%%%%%%%%%%%%%%%
	Of course, constant, variables and primitive functions are not
sufficient for writing a program that solves a complex problem.
	The important expressions in ML programming are those that
represent functions.

	As seen in an example in Section~\ref{sec:tutorialDeclarative}, 
a function is defined by the following syntax:
\begin{program}
{\bf fun}\ {\it funName}\ {\it param} = {\it expr}
\end{program}
	After this declaration, the variable {\it funName} is bound to a
function that takes {\it param} as its argument and returns the result
computed by {\it expr}.
	For example, the equation for an arithmetic sequence we have
seen before can be regarded as a function that takes a natural number
$n$:
\[
\mbox{$S(n)$} = \frac{n (n + 1) (2n + 1)}{6}
\]
	This definition can be programmed directly using {\tt fun} as follows.
\begin{program}
\# fun S(n) = (n * (n + 1) * (2*n + 1)) div 6;\\
val S = fn : int -> int
\end{program}
	After this definition, {\tt S} can be used as follows.
\begin{program}
\# S 10;\\
val it = 385 : int\\
\# S 20;\\
val it = 2870 : int
\end{program}

\fi%%%%<<<<<<<<<<<<<<<<<<<<<<<<<<<<<<<<<<<<<<<<<<<<<<<<<<<<<<<<<<<<<<<<<<


\section{\txt{再帰的な関数}{Recursive functions}}
\label{sec:tutorialRecursion}

\ifjp%>>>>>>>>>>>>>>>>>>>>>>>>>>>>>>>>>>>>>>>>>>>>>>>>>>>>>>>>>>>>>>>>>>>
	前節で学んだ{\tt fun}構文は再帰的な定義を許します.
	すなわち,この構文で定義される関数{\it funName}を,関数定義本体
{\it expr}の中で使用することができます.
	このような再帰的な関数定義は,以下のような手順で設計していけば,
通常の算術関数と同じように自然に定義できるようになります.
\begin{enumerate}
\item 関数{\it funName}の振る舞いをあらかじめ想定する.
\item 想定された振る舞いをする関数{\it funName}があると仮定し,
受け取った引数が{\it param}の場合に返すべきか値を表す式{\it expr}
を定義する.
\end{enumerate}
	例えば,第\ref{sec:tutorialDeclarative}節で紹介した階乗を計算す
る関数{\tt fact}は以下のように設計できます.
\begin{enumerate}
\item {\tt fact}を階乗を計算する関数と想定します.
\item 階乗を計算する関数{\tt fact}を使い,階乗の定義に従い,返すべき値を
以下のように定義する.
\begin{enumerate}
\item 引数が{\tt 0}の場合,0の階乗は1であるから,{\tt 1}を返す.
\item {\tt 0}以外の一般の引数{\tt n}の場合,階乗の定義から,
{\tt n * fact(n - 1)}を返す.
\end{enumerate}
\end{enumerate}
	この2つの場合を単に書き下せば,以下の定義が得られます.
\begin{tt}
\begin{quote}
\# fun fact 0 = 1\\
\verb|>| \ \ \ \ \ | fact n = n * fact (n - 1);\\
val fact = fn : int -> int
\end{quote}
\end{tt}

	以上の考え方に従い種々の再起関数を簡単に定義できます.
	たとえば,すべての数の和を求める関数{\tt sum}や引数$n$に対して与
えられた定数$c$の$n$乗を計算する{\tt power}なども,これら関数がすでにあ
ると思うことによって,以下のように簡単に定義できます.
\begin{tt}
\begin{quote}
fun sum 0 = 0\\
\ \ \ \ \ | sum n = n + sum (n - 1)\\
fun power 0 = 1\\
\ \ \ \ \ | power n = c * power (n - 1)
\end{quote}
\end{tt}
	{\tt sum}や{\tt power}が想定した振る舞いをすると仮定すれば,
関数が返す値は正しいので,関数全体の定義も正しいことがわかります.

\else%%%%%%%%%%%%%%%%%%%%%%%%%%%%%%%%%%%%%%%%%%%%%%%%%%%%%%%%%%%%%%%%%%%%
	The {\tt fun} syntax allows recursive function definitions. 
	The function name {\it funName} being defined in this
declaration can be used in the defining body {\it expr} of this
declaration.

	A recursive function can naturally be designed in the following step.
\begin{enumerate}
\item 
	Presume the expected behavior of the function {\it funName} for
all the cases, and assume that such function exists.
\item 
	For each case of the parameter {\tt param}, write the {\it expr}
using {\tt param} and {\tt funName} for that case.
\end{enumerate}
	For example,function {\tt fact} shown in
Section~\ref{sec:tutorialDeclarative} is designed in the following steps.
\begin{enumerate}
\item Assume that you have the function {\tt fact} that correctly
computes the factorial of any non-negative number.
\item Using {\tt fact}, write down the body of each case as follows.
\begin{enumerate}
\item If the parameter is {\tt 0}, then since $0!=1$, the {\tt expr} is {\tt 1},
\item otherwise, from the definition of the factorial function, the {\tt
expr} is {\tt n * fact(n - 1)}.
\end{enumerate}
\end{enumerate}
	This yields the following recursive function definition.
\begin{program}
\# fun fact 0 = 1\\
\verb|>| \ \ \ \ \ | fact n = n * fact (n - 1);\\
val fact = fn : int -> int
\end{program}

	The summation function {\tt sum} and exponentiation function {\tt
power}, for example, can similarly be written by presuming that these
functions were already given.
\begin{tt}
\begin{quote}
fun sum 0 = 0\\
\ \ \ \ \ | sum n = n + sum (n - 1)\\
fun power 0 = 1\\
\ \ \ \ \ | power n = c * power (n - 1)
\end{quote}
\end{tt}
	Under the presumed behavior of {\tt sum} and {\tt power}, each
case correctly computes the expected value, and therefore the entire
definitions are correct.

	This is a recursive way of reasoning.
	Think in this recursive way, and you can naturally write various
recursive functions.
\fi%%%%<<<<<<<<<<<<<<<<<<<<<<<<<<<<<<<<<<<<<<<<<<<<<<<<<<<<<<<<<<<<<<<<<<

\section{\txt{複数の引数を取る関数}{Functions with multiple arguments}}
\label{sec:tutorialMultiargfun}

\ifjp%>>>>>>>>>>>>>>>>>>>>>>>>>>>>>>>>>>>>>>>>>>>>>>>>>>>>>>>>>>>>>>>>>>>
	前節の{\tt power}関数はあらじめ定義された定数{\tt C}の指数乗を計算す
る関数でしたが,では,この指数の底{\tt C}も引数に加え,{\tt n}と{\tt C}
を受け取って{\tt C}の{\tt n}乗を計算する関数を定義するにはどうしたらよい
でしょうか.
	これには,2つの書き方があります.
	以下の対話型セッションでは,その型情報とともに示します.
\begin{tt}
\begin{quote}
\# fun powerUncurry (0,C) = 1\\
> \ \ \ \ \ | powerUncurry (n,C) = C * powerUncurry(n - 1,C)\\
val powerUncurry : int * int -> int\\
\# powerUncurry (2,3);\\
val it = 9 : int\\
\ \\
\# fun powerCurry 0 C= 1\\
> \ \ \ \ \ | powerCurry n C = C * (powerCurry (n - 1) C)\\
val powerCurry : int -> int -> int\\
\# powerCurry 2 3;\\
val it = 9 : int
\end{quote}
\end{tt}
	{\tt powerUncurry}の型情報{\tt int * int -> int}は,引数を組として
受け取り整数を返す関数であることを示しています.
	これに対して{\tt powerCurry}の型情報{\tt int -> int -> int}は,
引数を2つ順番に受け取り整数を返す関数であることを示しています.
	上の例から想像される通り,MLでは関数定義の引数は,その関数が使わ
れる形で書きます.
	例えば{\tt powerUncurry}は{\tt fun powerUncurry (C,n) = ...}と定
義されていますから,{\tt powerUncurry(2,3)}のように使います.
\else%%%%%%%%%%%%%%%%%%%%%%%%%%%%%%%%%%%%%%%%%%%%%%%%%%%%%%%%%%%%%%%%%%%%
	The function {\tt power} defined in the previous section
computers the power $C^n$ of $C$ for a given {\tt C}.
	There are two ways of generalizing this function so that it takes 
both {\tt n} and {\tt C} and return $C^n$.
	They are shown in the following interactive session.
\begin{program}
\# fun powerUncurry (0,C) = 1\\
> \ \ \ \ \ | powerUncurry (n,C) = C * powerUncurry(n - 1,C)\\
val powerUncurry : int * int -> int\\
\# powerUncurry (2,3);\\
val it = 9 : int\\
\ \\
\# fun powerCurry 0 C= 1\\
> \ \ \ \ \ | powerCurry n C = C * (powerCurry (n - 1) C)\\
val powerCurry : int -> int -> int\\
\# powerCurry 2 3;\\
val it = 9 : int
\end{program}
	The type {\tt int * int -> int} of {\tt powerUncurry} indicates
that it is a function that takes a pair of integers and returns an
integer.
	In contrast, the type {\tt int -> int -> int} of {\tt
powerCurry} says that it is a function that takes an integer and returns
a function of type {\tt int -> int}.
	As seen in these example, in ML, a function is defined in the
format as it will be used.
	For example, {\tt powerUncurry} is defined {\tt fun powerUncurry
(C,n) = ...} so it is used as {\tt powerUncurry(2,3)}.
\fi%%%%<<<<<<<<<<<<<<<<<<<<<<<<<<<<<<<<<<<<<<<<<<<<<<<<<<<<<<<<<<<<<<<<<<

\section{\txt{関数適用の文法}{Function application syntax}}
\label{sec:tutorialApplysyntax}

\ifjp%>>>>>>>>>>>>>>>>>>>>>>>>>>>>>>>>>>>>>>>>>>>>>>>>>>>>>>>>>>>>>>>>>>>
	{\tt powerUncurry}は,複数引数を取るC言語の関数と同様です.
	ML言語の強みは,これ以外に,{\tt powerCurry}にように
引数を順番に受け取る関数が書けることです.
	これを正確に理解するために,ここでMLの関数式の構文を復習しておき
ましょう.
	数学などの式では関数の利用(適用)は,関数の引数を括弧でかこみ
$f(x)$のように記述しますが,MLなどラムダ計算の考え方を基礎とした関数型言
語では,単に関数と引数を並べて{\tt f x}のように書きます.

	式の集合を$expr$で表すと,定数,変数,関数適用を含むMLの構文は一
部は,以下のような文法となります.
\begin{tt}
\begin{eqnarray*}
expr &\mbox{\ \ }::=& c                  \mbox{\myem\myem (定数)} \\
     &|& x                    \mbox{\myem\myem (変数)} \\
     &|& expr\mbox{\ } expr   \mbox{\ (関数適用)} \\
     &|& \cdots               \mbox{\myem\myem その他の構文}
\end{eqnarray*}
\end{tt}

	この構文規則のみでは関数適用式が続いた場合,その適用順番をきめる
ことができません.
	例えば,式に割り算の構文$expr / expr$と足し算
$expr + expr$
を導入した場合を考えてみましょう.
	この文法は,{\tt 60 / 6 / 2 + 1}のような式をゆるしますが,どの割り算
が先に行われるのかで2通りの解釈が可能です.
	この算術式では,通常左から括弧があるとみなし,{\tt ((60 / 10) /
2) + 1 }と解釈されます.
	このような解釈を,割り算の構文は「左結合」し,かつ足し算の構文より「結
合力が強い」,と言います.

	ラムダ計算を下敷きとした関数適用をもつ言語では,関数適用式に関し
て以下の重要な約束があります.
\begin{quote}
関数適用式は左結合し,その結合力は最も強い
\end{quote}
	関数式は左結合すると約束されているので,{\tt powerCurry 2 3}は
{\tt ((powerCurry 2) 3)}と解釈されます.
	一般に,{\tt e1\ e2\ e3\ $\cdots$\ en}と書いた式は
{\tt ($\cdots$ ((e1\ e2)\ e3)\ $\cdots$\ en)}とみなされます.
\else%%%%%%%%%%%%%%%%%%%%%%%%%%%%%%%%%%%%%%%%%%%%%%%%%%%%%%%%%%%%%%%%%%%%
	{\tt powerUncurry} is just like a C function taking a pair of
arguments.
	Unlike C and other procedural languages, ML also allows the
programmer to write a function that takes multiple arguments in a
sequence like {\tt powerCurry}.
	For proper understanding of this feature, let us review function
application syntax.
	In mathematics, function application is written as $f(x)$,
but in ML it is written as {\tt f x}, simply juxtaposing function and its
argument. 

	Let $expr$ be the expressions consisting of constants, variables
and function applications.
	Its syntax is given by the following grammar.
\begin{tt}
\begin{eqnarray*}
expr &\mbox{\ \ }::=& c                  \mbox{\myem\myem (constants)} \\
     &|& x                    \mbox{\myem\myem (variables)} \\
     &|& expr\mbox{\ } expr   \mbox{\ (function application)} \\
     &|& \cdots
\end{eqnarray*}
\end{tt}

	This grammar is ambiguous in the sequence of function
applications.
	To understand this problem, let us examine familiar arithmetic
expressions.
	Suppose we introduce 
$expr + expr$,
$expr - expr$, and
$expr * expr$
for integer addition, subtraction and multiplication.
	Then we can write {\tt 10 - 6 - 2} but this expression
itself does not determine the order of subtractions.
	As a consequence, two different results are possible:
{\tt (10 - 6) - 2 = 2} 
or
{\tt 10 - (6 - 2) = 4}.
	As in elementary school arithmetic, ML interprets this 
as {\tt (10 - 6) - 2 = 2}.
	We say that the subtraction {\em associates to the left} or
{\em is left associative}.
	For anther example, ML interprets {\tt 10 + 6 * 2} as {\tt 10
+ (6 * 2)}.
	We say that the multiplication {\tt *} associates {\em stronger
than} addition {\tt +}.
	
	In ML and other lambda calculus based functional languages, 
there is the following important rule:
\begin{quote}
function application associates to the left and its associatibity is the strongest
\end{quote}
	{\tt e1\ e2\ e3\ $\cdots$\ en} is interpreted as {\tt ($\cdots$
((e1\ e2)\ e3)\ $\cdots$\ en)}.
	For example, {\tt powerCurry 2 3} is interpreted as {\tt
((powerCurry 2) 3)}.
\fi%%%%<<<<<<<<<<<<<<<<<<<<<<<<<<<<<<<<<<<<<<<<<<<<<<<<<<<<<<<<<<<<<<<<<<

\section{\txt{高階の関数}{Higher-order functions}}
\label{sec:tutorialHeigher-order-fun}

\ifjp%>>>>>>>>>>>>>>>>>>>>>>>>>>>>>>>>>>>>>>>>>>>>>>>>>>>>>>>>>>>>>>>>>>>
	{\tt powerCurry 2 3}は{\tt ((powerCurry 2) 3)}と解釈されるという
ことは,ML言語における以下の重要な性質を意味しています.
\begin{quote}
{\tt (powerCurry 2)}は関数である.
\end{quote}
	このようにMLでは,関数は{\tt powerUncurry}のように関数定義構文を通じて定義
された関数名だけではなくプログラムで作ることができます.
	このように返された値は,{\tt powerCurry}の例のように,使うことが
できます.
	さらに,名前を付けたら関数に渡したりすることができます.
	例えば,以下のようなコードを書くことができます.
\begin{tt}
\begin{quote}
\# val square = power 2;\\
val square = int -> int\\
\# squqre 3;\\
val it = 9 : int\\
\# fun apply f = f 3;\\
val apply : (int -> int) -> int\\
\# apply square;\\
val it = 9 : int
\end{quote}
\end{tt}
	{\tt apply}に対して推論された型は,この関数が,関数を受け取って
整数を返す関数であることを示しています.

	以上の例から理解される通り,MLは以下の機能を持っています.
\begin{quote}
	関数は,整数型などのデータと同様,プ
ログラムが返すことができる値でああり,自由に受け渡し,使うことができる.
	特に関数は,別の関数を引数として受け取ることができる.
\end{quote}
	これが,「プログラムを式で定義していく」という大原則の下での,MLプ
ログラミングの最も重要な原則です.
	関数を返したり受け取ったりする関数を{\bf 高階の関数}と呼びます.
\else%%%%%%%%%%%%%%%%%%%%%%%%%%%%%%%%%%%%%%%%%%%%%%%%%%%%%%%%%%%%%%%%%%%%
	As we learned that {\tt powerCurry 2 3} is interpreted as {\tt
((powerCurry 2) 3)}. 
	This implies:
\begin{quote}
{\tt (powerCurry 2)} is a function.
\end{quote}
	In ML, functions can be generated by a program.
	{\tt powerCurry} is a program that takes an integer and return a
function.
	Generated functions can be named and passed to other functions.	
	For example, one can write the following code.
\begin{quote}
\# val square = power 2;\\
val square = int -> int\\
\# squqre 3;\\
val it = 9 : int\\
\# fun apply f = f 3;\\
val apply : (int -> int) -> int\\
\# apply square;\\
val it = 9 : int
\end{quote}
	The type of {\tt apply} means that this is a function that takes
a function on integers and return an integer.

	In summary, ML has the following power.
\begin{quote}
	Just like integers, functions are values that can be returned
from a function or passed to a function.
	In particular, a function can take a function as a parameter. 
% 	関数は,整数型などのデータと同様,プ
% ログラムが返すことができる値でああり,自由に受け渡し,使うことができる.
% 	特に関数は,別の関数を引数として受け取ることができる.
\end{quote}
	This is the important principle of ML under the fundamental
principle of ML programming saying that programs are expressions.
% 	これが,「プログラムを式で定義していく」という大原則の下での,MLプ
% ログラミングの最も重要な原則です.
	A function that takes a function as a parameter or return a
function is called a {\bf higher-order function}.
\fi%%%%<<<<<<<<<<<<<<<<<<<<<<<<<<<<<<<<<<<<<<<<<<<<<<<<<<<<<<<<<<<<<<<<<<


\section{\txt{高階の関数の利用}{Using higher-order functions}}
\label{sec:tutorialHeigher-order-use}


\ifjp%>>>>>>>>>>>>>>>>>>>>>>>>>>>>>>>>>>>>>>>>>>>>>>>>>>>>>>>>>>>>>>>>>>>
	MLでクールな,つまり読みやすくかつ簡潔なプログラミングを効率良く
書いていく鍵は,高階関数の機能を理解し,高階の関数を使い,式を組み合わせ
ていくことです.
	この技術の習得には,当然習熟が必要でこのチュートリアルで尽くせる
ものではありませんが,その基本となる考え方を理解することは重要と思います.
	そこで,以下,簡単な例を用いて,高階の関数の定義と利用に関する考
え方を学びましょう.

	数学で数列の和に関する$\Sigma$記法を勉強したと思います.
	この記法は以下のような等式に従います.
\begin{eqnarray*}
\sum_{k=1}^0 f(k) &=& 0\\
\sum_{k=1}^{n+1} f(k) &=& f(n) + \sum_{k=1}^{n} f(k)
\end{eqnarray*}
	高校の教科書などでこの記法を学ぶ理由は,もちろん,この$\Sigma$で
表現される働きに汎用性があり便利だからです.
	記法$\sum_{k=1}^n f(k)$の簡単な分析をしてみましょう.
\begin{itemize}
\item $\sum$記法には$k,n,f$の3つの変数が含まれる.
この中で$k$は,$1$から$n$まで変化することを表す仮の変数である.
\item 式$\sum_{k=1}^n f(k)$は,与えられた自然数$n$と関数$f$に
対して
$
f(1) + f(2) + \cdots + f(n)
$
の値を表す式である.
\end{itemize}
	従って,$\sum$は$n$と整数を引数とし整数を返す関数$f$を受け取る
関数,つまり高階の関数とみなすことができます.
	高階の関数をプログラミングの基本とするMLでは,この関数を以下のよ
うに直接定義可能です.
\begin{tt}
\begin{quote}
\# fun Sigma f 0 = 0\\
\ \ \ \ \ \ \ \ \  | Sigma f n = f n + Sigma f (n - 1)\\
val Sigma = fn : (int -> int) -> int -> int
\end{quote}
\end{tt}
この{\tt Sigma}を使えば,任意の自然数関数の和を求めることができます.
\begin{tt}
\begin{quote}
\# Sigma square 3:\\
val it = 14 : int\\
\# Sigma (power 3) 3:\\
val it = 36 : int
\end{quote}
\end{tt}
	高階の関数は,このようにひとまとまりの機能を表現する上で大きな力
をもちます.
	複雑なプログラムを簡潔で読みやすいMLのコードとして書き下していく
能力を身につける上で重要なステップは,ひとまとまりの仕事を高階の関数として
抜き出し,それらを組み合わせながら,式を作りあげていく技術をマスターする
ことです.
	後に第\ref{sec:tutorialPolymorphicfunction}節で説明するMLの多相
型システムは,このスタイルのプログラムをより容易にする機構といえます.
	この機構とともに,MLの高階関数を活用すれば,複雑なシステムを宣言
的に簡潔に書き下していくことができるようになるはずです.
\else%%%%%%%%%%%%%%%%%%%%%%%%%%%%%%%%%%%%%%%%%%%%%%%%%%%%%%%%%%%%%%%%%%%%
	The key to writing a cool, i.e.\ readable and concise program
efficiently is to understand the role of higher-order functions
and to compose expressions using higher-order functions.
% 	MLでクールな,つまり読みやすくかつ簡潔なプログラミングを効率良く
% 書いていく鍵は,高階関数の機能を理解し,高階の関数を使い,式を組み合わせ
% ていくことです.
	As in any programming skill, mastering higher-order functions
require certain amount of practice, and is beyond this tutorial.
	However, there are few basic concepts that are important in
writing expressions using higher-order functions.
	This tutorial attempt to exhibit them.
	In what follows, we learn these concepts through simple
examples.

	In high school mathematics, we have learned $\Sigma$ notation.
	This notation satisfies the following equations.
\begin{eqnarray*}
\sum_{k=1}^0 f(k) &=& 0\\
\sum_{k=1}^{n+1} f(k) &=& f(n) + \sum_{k=1}^{n} f(k)
\end{eqnarray*}
	The reason for learning this notation is because this notation
represents a useful abstraction, i.e.\ summing up a sequence of
expressions with integer indexes.
% 	高校の教科書などでこの記法を学ぶ理由は,もちろん,この$\Sigma$で
% 表現される働きに汎用性があり便利だからです.
	A brief analysis of the notation $\sum_{k=1}^n f(k)$ reveals the
following.
\begin{itemize}
\item The notation contains 3 variables $k,n,f$.
	Among them, $k$ is a variable that is only to indicate that
expressions are index by $k=1,\ldots,n$, and is not a real variable. 
	The variables of this expression are $n$ and $f$.
\item Expression $\sum_{k=1}^n f(k)$ denotes the value
$
f(1) + f(2) + \cdots + f(n)
$
for any given integer $n$ and a function $f$.
\end{itemize}
	$\sum$ is a function that takes an integer $n$ and a function
$f$ that takes an integer and returns an integer, i.e.\ it is a
higher-order function.
% 	高階の関数をプログラミングの基本とするMLでは,この関数を以下のよ
% うに直接定義可能です.
	In ML, this higher-order function is directly code as follows.
\begin{program}
\# fun Sigma f 0 = 0\\
\ \ \ \ \ \ \ \ \  | Sigma f n = f n + Sigma f (n - 1)\\
val Sigma = fn : (int -> int) -> int -> int
\end{program}
	This {\tt Sigma} can compute the summation of any inter function.
\begin{program}
\# Sigma square 3:\\
val it = 14 : int\\
\# Sigma (power 3) 3:\\
val it = 36 : int
\end{program}

	As demonstrated by this example, higher-order functions has the
power of representing a useful computation pattern by abstracting its
components as argument functions.
	The key to writing a cool ML code is to master the skill of
abstracting useful computation patterns as higher-order functions.
	Polymorphic typing we shall learn in Section
\ref{sec:tutorialPolymorphicfunction} can be regarded as a mechanism to
support this style of programming. 
	Higher-functions with their polymorphic typing enable the ML
programmer to code a complicated system in a concise and declarative
way.
% 	この機構とともに,MLの高階関数を活用すれば,複雑なシステムを宣言
% 的に簡潔に書き下していくことができるようになるはずです.

% 	高階の関数は,このようにひとまとまりの機能を表現する上で大きな力
% をもちます.
% 	複雑なプログラムを簡潔で読みやすいMLのコードとして書き下していく
% 能力を身につける上で重要なステップは,ひとまとまりの仕事を高階の関数として
% 抜き出し,それらを組み合わせながら,式を作りあげていく技術をマスターする
% ことです.
% 	後に第\ref{sec:tutorialPolymorphicfunction}節で説明するMLの多相
% 型システムは,このスタイルのプログラムをより容易にする機構といえます.
% 	この機構とともに,MLの高階関数を活用すれば,複雑なシステムを宣言
% 的に簡潔に書き下していくことができるようになるはずです.
\fi%%%%<<<<<<<<<<<<<<<<<<<<<<<<<<<<<<<<<<<<<<<<<<<<<<<<<<<<<<<<<<<<<<<<<<

\section{\txt{MLにおける手続き的機能}{Imperative features of ML}}
\label{sec:tutorialImperative}

\ifjp%>>>>>>>>>>>>>>>>>>>>>>>>>>>>>>>>>>>>>>>>>>>>>>>>>>>>>>>>>>>>>>>>>>>
	これまで学んできた通り,式の組み合わせでプログラムを構築していく
ML言語はプログラムを宣言的に書く上で最も適した言語の一つです.
	これまでに,$\sum$関数などの算術演算を行う関数を例に説明しました
が,第\ref{sec:tutorialDeclarative}節で説明した通り,宣言的とは,意図す
る意味をできるだけそのまま表明するということであり,決して数学的な関数と
してプログラムを書いていくことが望ましい,との表明ではありません.

	もし,実現しようとしている機能が,手続き的な操作によって最も簡潔
に表現し理解できるなら,その手続きをそのまま表現するのが望ましいはずです.
	例えば,みなさんが今ご覧のディスプレイに文字を表示する操作は,こ
れまでに書かれている内容を別な情報で上書きする操作であり,具体的には,
ディスプレイのバッファを書き換える手続き的な操作です.
	ディスプレイに限らず,外部との入出力処理は,外部の状態に依存する
本質的に手続き的な操作です.
	そのほか,例えば新しい名前や識別子の生成などの処理,さらに,
サイクルのあるグラフの変換なども,手続き的なモデルで理解しプログラムした
ほが,より宣言的な記述になる場合が多いと言えます.

	このような操作をプログラムで簡潔で分かりやすく表現するには,
手続き的な処理が記述できたほうが便利です.
	式の組み合わせでコードを書いていくことが基本である関数型言語にこ
のような手続き的な処理を導入するために必要なことは,以下の2つです.
\begin{enumerate}
\item 状態を変更する機能の導入.
	例にあげたディスプレイの表示内容は,抽象的にはディスプレイの状態
と捉えることができます.
	手続き的なプログラムの基本は,この状態を変更することをくり返し目
的とする状態を作りだすことです.
	そのためには,変更可能なデータが必要となります.
\item 評価順序の確定.
	ディスプレイの表示内容の変更操作が複数あった場合,それらの適用順
序によって,どのような像が見えるかが決まります.
	従って,手続き的なプログラムを書くためには,変更操作がどの順に行
われるかに関する知識と制御が必要となります.
\end{enumerate}
	MLには,これら2つが関数型言語の枠組みの中に導入されています.
\else%%%%%%%%%%%%%%%%%%%%%%%%%%%%%%%%%%%%%%%%%%%%%%%%%%%%%%%%%%%%%%%%%%%%
	ML is a language where programs are defined by composing
expressions, and is suitable for declarative programming.
	So far we have used numerical functions such as $\sum$, but as
pointed out in Section \ref{sec:tutorialDeclarative},the intention of
declarative programming is to write what the program should express
directly as a code, and not intend to define a program as a mathematical
function.

	If the actual problem is best expressed as a sequence of
procedure, then directly writing done that sequence would be most
declarative.
	For example, displaying a sentence on the video display you are
looking at is the process to override the previous sentence with the new
sequence of characters.
	In an actual implementation, this is done by modifying the frame
buffer of the video screen hardware.
	This process is inherently procedural, and therefore best
described as a process to update the buffer.
	Input from and output to external world inevitably have this
property.
	For those problems, procedural representation would be most
direct and therefore concise and readable, thus declarative.
	
	Integrating imperative features in a functional language
requires the following.
\begin{enumerate}
\item Introduction of updating a mutable state.
	In the previous display example, display contents can be
understood as the state of the display.
	The basis of imperative programming is to update states to
obtain the desired state.
	For this, the introduction of mutable data structures is
required.
\item A predictable evaluation order.
	In the display example, the order of updates determine the image
and its motion in the display.
	To write a procedural program, it is necessary to control the
order of update operations.
% 	ディスプレイの表示内容の変更操作が複数あった場合,それらの適用順
% 序によって,どのような像が見えるかが決まります.
% 	従って,手続き的なプログラムを書くためには,変更操作がどの順に行
% われるかに関する知識と制御が必要となります.
% 	ディスプレイの表示内容の変更操作が複数あった場合,それらの適用順
% 序によって,どのような像が見えるかが決まります.
% 	従って,手続き的なプログラムを書くためには,変更操作がどの順に行
% われるかに関する知識と制御が必要となります.
\end{enumerate}
	ML introduces these two in a framework of functional programming.
\fi%%%%<<<<<<<<<<<<<<<<<<<<<<<<<<<<<<<<<<<<<<<<<<<<<<<<<<<<<<<<<<<<<<<<<<

\section{
\txt{変更可能なメモリーセルを表す参照型}
    {Mutable memory reference types}}
\label{sec:tutorialRef}

\ifjp%>>>>>>>>>>>>>>>>>>>>>>>>>>>>>>>>>>>>>>>>>>>>>>>>>>>>>>>>>>>>>>>>>>>
	MLには,以下の型と関数が組み込まれています.
\begin{tt}
\begin{quote}
type 'a ref\\
val ref : 'a -> 'a ref\\
val ! : 'a ref -> 'a\\
val := : 'a ref * 'a -> unit
\end{quote}
\end{tt}
	型{\tt 'a ref}は「'aの値への参照」を表す型です.
	参照とはポインタのことです.
	関数{\tt ref}は,任意の型{\tt 'a}の値を受け取り,その値への参照
を作成し返す関数です.
	関数{\tt !}は,参照を受け取りそれが指す値を返す関数です.
	関数{\tt :=}は代入を行う関数,すなわち,参照と値を受け取り,参照
が指す値を受け取った値に変更する関数です.
	以下は,参照型を使用した対話型セッションの例です.
\begin{tt}
\begin{quote}
\#val x = ref 1;\\
val x = \_ : int ref\\
\# !x;\\
val it = 1 : int\\
\# x := 2;\\
val it = () : unit\\
\# !x;\\
val it = 2 : int
\end{quote}
\end{tt}
	この参照型を,状態を参照として保持することにより状態に依存する処
理などの本質的に手続き的な処理を,通常の関数として書くことができます.
\else%%%%%%%%%%%%%%%%%%%%%%%%%%%%%%%%%%%%%%%%%%%%%%%%%%%%%%%%%%%%%%%%%%%%
	ML has the following type constructor and primitives.
\begin{tt}
\begin{quote}
type 'a ref\\
val ref : 'a -> 'a ref\\
val ! : 'a ref -> 'a\\
val := : 'a ref * 'a -> unit
\end{quote}
\end{tt}
	{\tt 'a ref} is a type of references (pointers) of type {\tt 'a}.
	Function {\tt ref} takes a value of type {\tt 'a} and returns a
reference to it.
	Function {\tt !} takes a reference and returns the value of that
reference.
	Function {\tt :=} assigns a given value to a given reference.
	This destructively update the reference cell.
	The following shows an interactive session manipulating a reference.
\begin{tt}
\begin{quote}
\#val x = ref 1;\\
val x = \_ : int ref\\
\# !x;\\
val it = 1 : int\\
\# x := 2;\\
val it = () : unit\\
\# !x;\\
val it = 2 : int
\end{quote}
\end{tt}
	A function performing imperative operation can be written by
maintaining a state using a reference.
% 	この参照型を,状態を参照として保持することにより状態に依存する処
% 理などの本質的に手続き的な処理を,通常の関数として書くことができます.
\fi%%%%<<<<<<<<<<<<<<<<<<<<<<<<<<<<<<<<<<<<<<<<<<<<<<<<<<<<<<<<<<<<<<<<<<

\section{
\txt{作用順,左から右への評価戦略}
    {Left-to-right applicative order evaluation}
}
\label{sec:tutorialEvalorder}

\ifjp%>>>>>>>>>>>>>>>>>>>>>>>>>>>>>>>>>>>>>>>>>>>>>>>>>>>>>>>>>>>>>>>>>>>
	手続き的処理を書くためにはプログラムの各部分の実行順序を決める必
要があります.
	値を表す式を組み合わせてプログラムを書いていく関数型言語では,書
かれたプログラムの実行は,式の値を求めることに対応します.
	これを,式の評価と呼びます.
	MLでは,式の評価は以下のような順で行われます.
\begin{enumerate}
\item 式は,同一レベルであれば,左から右に評価する.
	たとえば,{\tt (power 2) (power 2 2)}の場合,
まず{\tt (power 2)}が評価され2乗する関数が得られ,
次に{\tt (power 2 2)}が評価され4が得られ,最後に,2乗する関数に4が適用さ
れ,14が得られます.
	また,宣言の列は,宣言の順に評価されます.
\item 関数の本体は,引数に適用されるまで評価されない.
\end{enumerate}
	さらに,手続き的な処理を容易にするために,以下の逐次評価構文が用
意されています.
\begin{tt}
\begin{eqnarray*}
expr &\mbox{\ \ }::=& \cdots\\
     &|& (expr;{\ }\cdots{\ };expr)   \mbox{\ \ \ \ (逐次評価構文)} \\
\end{eqnarray*}
\end{tt}
{\tt ($expr_1$;{\ }$\cdots${\ };$expr_n$)}は,
$expr_1$から$expr_n$まで順番に評価し,最後の式$expr_n$の値を返す構文です.

	以上の評価順序のもとで参照型を使えば,手続き的な処理を関数型言語
の枠組みで,簡単に書くことができます.
	たとえば新しい名前を生成する関数は,以下のように書けます.
\begin{tt}
\begin{quote}
fun makeNewId () =\\
\myem let\\
\myem\myem val cell = ref 0\\
\myem\myem fun newId () =\\
\myem\myem\myem let\\
\myem\myem\myem\myem val id = !cell\\
\myem\myem\myem in\\
\myem\myem\myem\myem (cell := id + 1; id)\\
\myem\myem\myem end\\
\myem in\\
\myem\myem newId\\
\myem end
\end{quote}
\end{tt}
\else%%%%%%%%%%%%%%%%%%%%%%%%%%%%%%%%%%%%%%%%%%%%%%%%%%%%%%%%%%%%%%%%%%%%
	To write a program that manipulate states, it is necessary to
fix the order of execution of each part of the program.
	In a functional language, where a program is a composition of
expression, program execution corresponds to obtain the value of an
expression.
	This process is called {\em evaluation of an expression}.
	In ML, expressions are evaluated in the following order.
\begin{enumerate}
\item The same level sub-expressions are evaluated from left to right. 
	For example, {\tt (power 2) (power 2 2)} is evaluated by
first evaluating {\tt (power 2)} to obtain a function that computes
power of 2 and then {\tt (power 2 2)} is evaluated to obtain 4, and
finally the power 2 function is applied to 4 to yield 
14.
\item A sequence of declarations are evaluated in the order of the
declarations.
\item The body of a function is not evaluated until the function is
applied to an argument.
\end{enumerate}
	ML also provide the following syntax to control evaluation order.
\begin{tt}
\begin{eqnarray*}
expr &\mbox{\ \ }::=& \cdots\\
     &|& (expr;{\ }\cdots{\ };expr)   \mbox{\ \ \ \ (sequential evaluation)} \\
\end{eqnarray*}
\end{tt}
	{\tt ($expr_1$;{\ }$\cdots${\ };$expr_n$)} is evaluated by 
evaluating $expr_1$ through $expr_n$ sequentially, and yields the value
of the last expression $expr_n$.

% 	以上の評価順序のもとで参照型を使えば,手続き的な処理を関数型言語
% の枠組みで,簡単に書くことができます.
% 	たとえば新しい名前を生成する関数は,以下のように書けます.
	By manipulating reference types in the evaluation order, 
an imperative operation can be written as a function.
	For example, a function that generate a new name by maintaining
a mutable state is defined as follows.
\begin{tt}
\begin{quote}
fun makeNewId () =\\
\myem let\\
\myem\myem val cell = ref 0\\
\myem\myem fun newId () =\\
\myem\myem\myem let\\
\myem\myem\myem\myem val id = !cell\\
\myem\myem\myem in\\
\myem\myem\myem\myem (cell := id + 1; id)\\
\myem\myem\myem end\\
\myem in\\
\myem\myem newId\\
\myem end
\end{quote}
\end{tt}
\fi%%%%<<<<<<<<<<<<<<<<<<<<<<<<<<<<<<<<<<<<<<<<<<<<<<<<<<<<<<<<<<<<<<<<<<
	
\section{\txt{手続き的制御}{Procedural control}}
\label{sec:tutorialControl}

\ifjp%>>>>>>>>>>>>>>>>>>>>>>>>>>>>>>>>>>>>>>>>>>>>>>>>>>>>>>>>>>>>>>>>>>>
	手続き的な処理を学んだところで,手続き的な言語での制御構造の記述
と式による関数の記述の関係を振り返ってみましょう.
	状態の変更を伴う手続き的処理は,外部の状態を変更するI/Oなどの機能
の表現には欠かせないものです.
	これらは,手続き的な操作として表現するのがもっとも分かりやすいプ
ログラムとなります.
	手続き型言語では,これらに加え,ループや条件式などの種々の制御構
造も手続き的な機能で実現されるのに対して,MLでは式の組み合わせで表現され
ます.
	これまで見てきた通り,MLではプログラム構造は手続き的機能に関係の
ない概念ですから,表現すべき内容を式で表すMLの方がより宣言的で理解しやす
いプログラムと言えます.

	しかし,この点に注意が必要です.
	Cのような手続き型言語では,手続き的処理が基本ですから,階乗を求
める関数{\tt fact}は,再帰ではなく,ループを使って以下のように書くのが一
般的と思われます.
\begin{tt}
\begin{quote}
\myem int fact (int n) \{\\
\myem\myem   int s = 1;\\
\myem\myem   while (n > 0) \{\\
\myem\myem\myem     s = s * n;\\
\myem\myem\myem     n = n - 1;\\
\myem\myem   \}\\
\myem\myem  return s;\\
\myem \}
\end{quote}
\end{tt}
	このようにループを使って書くほうが,再帰を使った書き方,たとえば
MLでの
\begin{tt}
\begin{quote}
\myem fun fact 0 = 1\\
\myem \ \ \ \ \ | fact n = n * fact (n - 1)
\end{quote}
\end{tt}
のようなコードより効率的です.
	しかし,この事実は,関数型言語の本質的な非効率さを意味するもので
はありません.
\else%%%%%%%%%%%%%%%%%%%%%%%%%%%%%%%%%%%%%%%%%%%%%%%%%%%%%%%%%%%%%%%%%%%%
	We have learned imperative (procedural) programming.
	At this point, let us review the relation between control
statements in a procedural language and functions. 

	Inherently procedural operations such as external IO are
naturally expressed as procedural programming.
	However, in a procedural language, control structures such as
loops and conditionals are also implemented using states. 
% 	状態の変更を伴う手続き的処理は,外部の状態を変更するI/Oなどの機能
% の表現には欠かせないものです.
% 	これらは,手続き的な操作として表現するのがもっとも分かりやすいプ
% ログラムとなります.
% 	手続き型言語では,これらに加え,ループや条件式などの種々の制御構
% 造も手続き的な機能で実現されるのに対して,MLでは式の組み合わせで表現され
% ます.
	Since program structures are independent of imperative
operation, representing them as expressions generally yield more
declarative programs.
% 	これまで見てきた通り,MLではプログラム構造は手続き的機能に関係の
% ない概念ですから,表現すべき内容を式で表すMLの方がより宣言的で理解しやす
% いプログラムと言えます.

	There is one thing to be noted however.
% 	Cのような手続き型言語では,手続き的処理が基本ですから,階乗を求
% める関数{\tt fact}は,再帰ではなく,ループを使って以下のように書くのが一
% 般的と思われます.
	In a procedural language like C, the factorial function may be
written as follows.
\begin{program}
\myem int fact (int n) \{\\
\myem\myem   int s = 1;\\
\myem\myem   while (n > 0) \{\\
\myem\myem\myem     s = s * n;\\
\myem\myem\myem     n = n - 1;\\
\myem\myem   \}\\
\myem\myem  return s;\\
\myem \}
\end{program}
	This is generally more efficient than a recursive function of
the form
\begin{program}
\myem fun fact 0 = 1\\
\myem \ \ \ \ \ | fact n = n * fact (n - 1)
\end{program}
% 	しかし,この事実は,関数型言語の本質的な非効率さを意味するもので
% はありません.
	This fact is however does not implies that functional
programming is inherently inefficient.
\fi%%%%<<<<<<<<<<<<<<<<<<<<<<<<<<<<<<<<<<<<<<<<<<<<<<<<<<<<<<<<<<<<<<<<<<

\section{\txt{ループと末尾再帰関数}{Loop and tail recursion}}
\label{sec:tutorialTailcall}

\ifjp%>>>>>>>>>>>>>>>>>>>>>>>>>>>>>>>>>>>>>>>>>>>>>>>>>>>>>>>>>>>>>>>>>>>
	くり返し処理(ループ)の設計の基本は,
\begin{enumerate}
\item 
求める値に至る(それを特別な場合として含む)計算の途中状態を設計し、
\item 
計算(の途中)結果を保持するための変数と
\item 
計算の進行状態を保持する変数,
\end{enumerate}
を用意し最終状態のチェックしながら,これら変数を更新するコードを書くこと
です.
	1からnまでの積を求める場合は,
\begin{itemize}
\item 
途中の計算状態:
$s = i * (i + 1) * ... * n$
\item 
計算結果を持つ変数:$s$
\item 
計算の状態を保持する変数:$i$
\end{itemize}
と考え,最終状態($i = 0$)をチェックし変数を変更するコードを書くと
前の{\tt fact}のようなコードとなります.
	この考え方に従えば,種々の複雑な処理をループとして書き下すことが
できます.

	この考え方は,手続き型言語に特有のものではありません.
	手続き型言語では,計算結果を持つ変数と状態を制御する変数の値を変
更しながらループします.
	この変数の変更とループは,
\begin{quote}
変数の現在の値を受け取り,新しい値を生成する関数
\end{quote}
と考えると,自分自身を呼び出すだけの関数となります.
	C言語の{\tt fact}関数で表現されたループコードは,以下のMLコード
で実現できます.
\begin{tt}
\begin{quote}
\myem  fun loop (0,s) = s\\
\myem \ \ \ \ \    | loop (n,s) = loop(n-1, s * n)\\
\myem fun fact n = loop (n, 1)
\end{quote}
\end{tt}
	この{\tt loop}が行う自分自身を呼び出すだけの再帰関数を末尾再帰と
呼びます.
	末尾再帰は,手続き型言語のループと同様の実行列にコンパイルされ効
率よく実行されます.
\else%%%%%%%%%%%%%%%%%%%%%%%%%%%%%%%%%%%%%%%%%%%%%%%%%%%%%%%%%%%%%%%%%%%%
	
	The basis of designing a loop program are the following.
% 	くり返し処理(ループ)の設計の基本は,
\begin{enumerate}
\item 
	Design a state of computation that leads to the desired result.
% 求める値に至る(それを特別な場合として含む)計算の途中状態を設計し、
\item 
% 計算(の途中)結果を保持するための変数と
	Set up variables that hold the computation state, and the
progress of computation.
\item 	
	Write a code to iteratively update the variables until 
the desired result is obtained.
\end{enumerate}
	A program that sums up 1 through n can be designed as follows.
\begin{itemize}
\item 
the state of computation:
$s = i * (i + 1) * ... * n (1\le i\le n)$
\item 
variables:$s, i$
\item 
update code: $s := s + 1; i := i - 1$
\end{itemize}
This design yields the C function {\tt fact} in Section~\ref{sec:tutorialControl}.
% 	この考え方に従えば,種々の複雑な処理をループとして書き下すことが
% できます.

	This way of designing an iterative computation is not limited to
procedural languages.
% 	この考え方は,手続き型言語に特有のものではありません.
% 	手続き型言語では,計算結果を持つ変数と状態を制御する変数の値を変
% 更しながらループします.
	A code to update the current state can be understood as a
function that takes the current state and generate the next state.
	Then a loop program is represented a recursive function that
takes the current state and call itself with the generated next state.
	The loop code of the C function {\tt fact} can be written as
the following ML code:
\begin{tt}
\begin{quote}
\myem  fun loop (0,s) = s\\
\myem \ \ \ \ \    | loop (n,s) = loop(n-1, s * n)\\
\myem fun fact n = loop (n, 1)
\end{quote}
\end{tt}
	This form of recursion is called {\tt tail recursion}, which is
evaluated efficiently.
% 	この{\tt loop}が行う自分自身を呼び出すだけの再帰関数を末尾再帰と
% 呼びます.
% 	{\tt }
% 	末尾再帰は,手続き型言語のループと同様の実行列にコンパイルされ効
% 率よく実行されます.
\fi%%%%<<<<<<<<<<<<<<<<<<<<<<<<<<<<<<<<<<<<<<<<<<<<<<<<<<<<<<<<<<<<<<<<<<

\section{\txt{let式}{let expressions}}
\label{sec:tutorialLetexpression}

\ifjp%>>>>>>>>>>>>>>>>>>>>>>>>>>>>>>>>>>>>>>>>>>>>>>>>>>>>>>>>>>>>>>>>>>>
	MLプログラミングの基本は,種々の式を定義しそれらに名前を付け組み
合わせていくことです.
	これまで見てきた例では,名前はすべてトップレベルに記録されていま
した.
	しかし,大きなプログラムでは,一時的に使用する多数の名前が必要と
なり,それら名前を管理する必要があります.
	例えば{\tt fact}は,トップレベルに定義された{\tt loop}を使って定
義されていますが,この名前は{\tt fact}の定義のためだけに導入されたもので
あり,他の関数ではべつなループ関数が必要です.
	そこで,これら特定の処理にのみ必要な名前のスコープ(有効範囲)を
制限できれば,より整理され構造化されたコードとなります.
	そのために,以下のlet構文が用意されています.
\begin{tt}
\begin{eqnarray*}
expr &\mbox{\ \ }::=& \cdots \\
     &|& \mbox{\tt let\ $decl\_list$\ in \ $expr$\ end}
\\
decl &\mbox{\ \ }::=& \mbox{\tt val\ $x$ = $expr$}
\\
     &|& \mbox{\tt fun\ $f$\ $p_1$ $\cdots$ $x_n$ =  $expr$}
\end{eqnarray*}
\end{tt}
	{\tt let}と{\tt in}の間には,{\tt in}と{\tt end}の式の中だけで有
効な宣言が書けます.
	例えば,末尾再帰の{\tt fact}関数は,通常以下のように定義します.
\begin{tt}
\begin{quote}
\myem  fun factorial n =
\\\myem\myem    let
\\\myem\myem\myem      fun loop (s,0) = s
\\\myem\myem\myem\myfm        | loop (s, i) = loop(s * i, i - 1)
\\\myem\myem    in
\\\myem\myem\myem      loop (1,n)
\\\myem\myem    end
\end{quote}
\end{tt}
\else%%%%%%%%%%%%%%%%%%%%%%%%%%%%%%%%%%%%%%%%%%%%%%%%%%%%%%%%%%%%%%%%%%%%
	ML programming is done by defining an expression and giving a
name to it for subsequent use.
% 	MLプログラミングの基本は,種々の式を定義しそれらに名前を付け組み
% 合わせていくことです.
	Examples we have seen so far, names are all recorded at the
top-level.
% 	これまで見てきた例では,名前はすべてトップレベルに記録されていま
% した.
	However, a large program requires a large number of names, many
of which are only temporarily used.
	For example, {\tt fact} function in the previous section is
defined using {\tt loop} function, but this name {\tt loop} is only
used in {\tt fact}; other function would need another version of {\tt
loop}.
	Such a name should be defined as a local name to the function
that use the name.
	For this purpose, ML provide the following let expression.
\begin{tt}
\begin{eqnarray*}
expr &\mbox{\ \ }::=& \cdots \\
     &|& \mbox{\tt let\ $decl\_list$\ in \ $expr$\ end}
\\
decl &\mbox{\ \ }::=& \mbox{\tt val\ $x$ = $expr$}
\\
     &|& \mbox{\tt fun\ $f$\ $p_1$ $\cdots$ $x_n$ =  $expr$}
\end{eqnarray*}
\end{tt}
	The declarations written between 
{\tt let} and {\tt in} are only visible to the code between {\tt in} and
{\tt end}.
	A tail recursive {\tt fact} function is usually written as follows.
\begin{tt}
\begin{quote}
\myem  fun factorial n =
\\\myem\myem    let
\\\myem\myem\myem      fun loop (s,0) = s
\\\myem\myem\myem\myfm        | loop (s, i) = loop(s * i, i - 1)
\\\myem\myem    in
\\\myem\myem\myem      loop (1,n)
\\\myem\myem    end
\end{quote}
\end{tt}
\fi%%%%<<<<<<<<<<<<<<<<<<<<<<<<<<<<<<<<<<<<<<<<<<<<<<<<<<<<<<<<<<<<<<<<<<

\section{\txt{リストデータ型}{List data type}}
\label{sec:tutorialList}

\ifjp%>>>>>>>>>>>>>>>>>>>>>>>>>>>>>>>>>>>>>>>>>>>>>>>>>>>>>>>>>>>>>>>>>>>
	MLプログラミングでよく使用される複合データはリストと高階多相関数
です.
	リストを用いたプログラムには,MLプログラミングの基本である以下の
要素が含まれています.
\begin{itemize}
\item 
再帰的データの生成と処理
\item 
パターンマッチングによるデータ構造の分解
\item 
多相型高階関数
\end{itemize}
	リストを用いてこれら機能をマスターしていきましょう.

リストは要素の並びです.
	MLでは``{\tt [}''と``{\tt ]}''の間に要素の式を並べると,リストが
定義されます.
\begin{tt}
\begin{quote}
\# [1,2,3];\\
val it = [1, 2, 3] : int list
\end{quote}
\end{tt}
	この式は,以下の式の略記法です.
\begin{tt}
\begin{quote}
\# 1::2::3::nil;\\
val it = [1, 2, 3] : int list
\end{quote}
\end{tt}
	この構文とその評価結果を理解しましょう.
\begin{itemize}
\item 
 {\tt e :: L}は要素を表す式{\tt e}とリストを表す式{\tt L}から,
リスト{\tt L}の先頭に{\tt e}を付け加えて得られるリストを生成する式.
\item 
{\tt nil}は空のリストを表す定数.
\item 
{\tt int list}は{\tt int}型を要素とするリストを表す型
\end{itemize}
\else%%%%%%%%%%%%%%%%%%%%%%%%%%%%%%%%%%%%%%%%%%%%%%%%%%%%%%%%%%%%%%%%%%%%
	Lists and the associated functions are commonly used data
structures in ML.
	List processing programs contain the following basic elements in
ML programming.
\begin{itemize}
\item 
	Recursively defined data.
\item 
	Pattern matching.
\item 
	Polymorphic functions
\end{itemize}
	Let us lean these elements through list programming.

	A list is a sequence of elements.
	In ML, a list of 1,2,3 is written as follows.
\begin{tt}
\begin{quote}
\# [1,2,3];\\
val it = [1, 2, 3] : int list
\end{quote}
\end{tt}
	This notation is a shorthand for the following expression.
\begin{tt}
\begin{quote}
\# 1::2::3::nil;\\
val it = [1, 2, 3] : int list
\end{quote}
\end{tt}
	{\tt ::} is a right-associative binary operator for constructing
a list.
	So {\tt 1::2::3::nil} is interpreted as {\tt 1::(2::(3::nil))}.
	 {\tt $e$ :: $L$} is the list obtained by prepending the element {\tt e} to
the list {\tt L}.
	{\tt nil} is the empty list.
	{\tt int list} is the list type whose component is type {\tt int}.
\fi%%%%<<<<<<<<<<<<<<<<<<<<<<<<<<<<<<<<<<<<<<<<<<<<<<<<<<<<<<<<<<<<<<<<<<

\section{\txt{式の組み合わせの原則}{Principle in composing expressions}}
\label{sec:tutorialTypingprinciple}

\ifjp%>>>>>>>>>>>>>>>>>>>>>>>>>>>>>>>>>>>>>>>>>>>>>>>>>>>>>>>>>>>>>>>>>>>
	MLプログラミングの原則は,式を組み合わせていくことでしたが,もち
ろんどのような組み合わせでも許されるわけではありません.
	MLでは,プログラムを構成する際,以下の基本原則に従います.
\begin{quote}
式は型が正しい限り自由に組み合わせることができる
\end{quote}
	リストの場合を例にこの原則を考えてみましょう.
	リストは,その要素の型に制限はありません.
	どのような値であれ,同じものは同一のリストにすることができます.
	以下の対話型セッションは,様々な型のリストを構築しています.
\begin{tt}
\begin{quote}
\myem \# fact 4 :: 4 + 4 :: (if factorial 1 = 0 then nil else [1,2,3]);
\\\myem  val it = [12, 24, 8] : int list
\\\myem   \# [1.1, Math.pi, Math.sqrt 2.0];
\\\myem   val it = [1.1, 3.14159265359, 1.41421356237] : real list
\\\myem   \# "I"::"became"::"fully"::"operational"::"on"::"April 2, 2012"::nil;
\\\myem   val it = ["I", "became", "fully", "operational", "on", "April 2, 2012"] : string list
\\\myem   \# [factorial, fib];
\\\myem   val it = [fn, fn] : (int  -> int) list
\\\myem   \# [\#"S", \#"M", \#"L", \#"\#"];
\\\myem   val it = [\#"S", \#"M", \#"L", \#"\#"] : char list
\\\myem   \# implode it;
\\\myem   val it = "SML\#" : string
\\\myem   \# explode it;
\\\myem   val it = [\#"S", \#"M", \#"L", \#"\#"] : char list
\end{quote}
\end{tt}
''implode''と''explode''はそれぞれ文字のリストを文字列に変換する関数お
よび文字列を文字のリストに変換する関数です.
\else%%%%%%%%%%%%%%%%%%%%%%%%%%%%%%%%%%%%%%%%%%%%%%%%%%%%%%%%%%%%%%%%%%%%
	The fundamental principle of ML programming to compose
expressions, but of course not all compositions yield meaningful
programs.
	In ML, the principle of composing expressions is the following.
\begin{quote}
{\bf expressions are freely composed as far as they are type correct.}
\end{quote}
	Let us consider this principle using lists as examples.
	List does not constrain its component types; any values can be
put into a list as far as their type is the same.
	The following interactive session shows constructions of lists
of various types.
\begin{tt}
\begin{quote}
\myem \# fact 4 :: 4 + 4 :: (if factorial 1 = 0 then nil else [1,2,3]);
\\\myem  val it = [12, 24, 8] : int list
\\\myem   \# [1.1, Math.pi, Math.sqrt 2.0];
\\\myem   val it = [1.1, 3.14159265359, 1.41421356237] : real list
\\\myem   \# "I"::"became"::"fully"::"operational"::"on"::"April 2, 2012"::nil;
\\\myem   val it = ["I", "became", "fully", "operational", "on", "April 2, 2012"] : string list
\\\myem   \# [factorial, fib];
\\\myem   val it = [fn, fn] : (int  -> int) list
\\\myem   \# [\#"S", \#"M", \#"L", \#"\#"];
\\\myem   val it = [\#"S", \#"M", \#"L", \#"\#"] : char list
\\\myem   \# implode it;
\\\myem   val it = "SML\#" : string
\\\myem   \# explode it;
\\\myem   val it = [\#"S", \#"M", \#"L", \#"\#"] : char list
\end{quote}
\end{tt}
''implode'' convert list of characters into a string and ''explode''
does its converse.
\fi%%%%<<<<<<<<<<<<<<<<<<<<<<<<<<<<<<<<<<<<<<<<<<<<<<<<<<<<<<<<<<<<<<<<<<

\section{\txt{多相型を持つ関数}{Polymorphic functions}}
\label{sec:tutorialPolymorphicfunction}

\ifjp%>>>>>>>>>>>>>>>>>>>>>>>>>>>>>>>>>>>>>>>>>>>>>>>>>>>>>>>>>>>>>>>>>>>
	「式は型が正しい限り自由に組み合わせることができる」
という原則を最大限に活かすためには,組み合わせを実現する関数は,種々の型
に対応している必要があります.
	前節のリストを構成する関数(演算子){\tt ::}の型は以下の通りです.
\begin{tt}
\begin{quote}
\# op ::;\\
val it : ['a. 'a * 'a list -> 'a list]
\end{quote}
\end{tt}
{\tt op}は,中置演算子を関数適用の構文で使うための接頭語です.
	この式は,{\tt ::}が,任意の型{\tt 'a}と
それと同じ型{\tt 'a}の要素とするリスト{\tt 'a list}を受け取って,
{\tt 'a}の要素とするリスト{\tt 'a list}を返す関数を表しています.
	ここで使われる{\tt 'a}は任意の型を表す型変数です.
	また{\tt ['a.$\cdots$]}は「すべての型
{\tt 'a}について,$\cdots$」である,との表明を表します.
	このように型変数を含み,種々の型のデータに利用できる関数を
{\bf 多相関数}と呼びます.
	さらに,これら多相関数を組み合わせて定義された関数も以下のように
多相型を持ちます.
\begin{tt}
\begin{quote}
\# fun cons  e L  = e :: L;
\\
val cons = fn : ['a .'a  ->  'a list  -> 'a list]
\end{quote}
\end{tt}
	MLは,式の組み合わせによるプログラミングを最大限にサポー
トするために,関数定義に対して,その関数のもっとも一般的な使い方
を許すような,{\bf 最も一般的な多相型}を推論します.
	以下の関数定義を考えてみましょう.
\begin{tt}
\begin{quote}
fun twice f x = f (f x);
\end{quote}
\end{tt}
	この関数は,関数と引数を受け取り,関数を引数に2回適用します.
	このふるまいを考えると,以下の型が最も一般的な利用を許す型である
とが理解できるでしょう.
\begin{tt}
\begin{quote}
 ['a. ('a -> 'a) -> 'a -> 'a]
\end{quote}
\end{tt}
	実際MLでは以下のように推論されます.
\begin{tt}
\begin{quote}
\# fun twice f x = f (f x);
\\
val twice = \_ : ['a. ('a -> 'a) -> 'a -> 'a]
\end{quote}
\end{tt}
\else%%%%%%%%%%%%%%%%%%%%%%%%%%%%%%%%%%%%%%%%%%%%%%%%%%%%%%%%%%%%%%%%%%%%
	In order to exploit the principle that  ``expressions are freely
composed as far as they are type consistent'', primitive functions used
to compose expressions should accept expressions of various types.
	The primitive {\tt ::} to construct a list has the following type.
\begin{program}
\# op ::;\\
val it : ['a. 'a * 'a list -> 'a list]
\end{program}
	{\tt op} prefix convert infix binary operator into a function
that takes a pair.
	This typing indicates that {\tt ::} is a function that takes a
value of type {\tt 'a} and a value of type {\tt 'a list}, which is a
list type whose element type is {\tt 'a}, and returns a list of the same
type.
	In this typing, {\tt 'a} represents an arbitrary type.
	The notation {\tt ['a.$\cdots$]} indicates that {\tt 'a} in
$\cdots$ can be replaced with any type, and corresponds to universally
quantified formula $\forall a.\cdots$ in logic.
	These types that quantifies over type variables are called {\em
polymorphic types}, and functions having a polymorphic type are called 
{\bf polymorphic functions}.
	
	Functions defined by composing polymorphic functions are often
polymorphic functions.
\begin{tt}
\begin{quote}
\# fun cons  e L  = e :: L;
\\
val cons = fn : ['a .'a  ->  'a list  -> 'a list]
\end{quote}
\end{tt}
	In this way, ML compiler infers {\bf a most general polymorphic
type} for an expression.
	To understand this mechanism, consider the following function.
\begin{program}
fun twice f x = f (f x);
\end{program}
	{\tt twice} takes a function and an argument and apply the
function to the argument twice.
	Since {\tt f} is applied to {\tt x}, the type of {\tt x} must be
the same as the argument type of {\tt f}.
	Moreover, since {\tt f} is applied again to the result of {\tt
f}, the result type of {\tt f} must be the same as its argument type.
	The most general type satisfying these constraint is the following.
\begin{program}
 ['a. ('a -> 'a) -> 'a -> 'a]
\end{program}
	ML indeed infers the following typing for {\tt twice}.
\begin{tt}
\begin{quote}
\# fun twice f x = f (f x);
\\
val twice = \_ : ['a. ('a -> 'a) -> 'a -> 'a]
\end{quote}
\end{tt}

	ML's principle in program construction -- ``expressions are
composed as far as they are type consistent'' -- is made possible by
this polymorphic type inference mechanism.
	{\tt twice} can be combined with any function and a value as far
as the function return the same type as its argument and that the value
has the argument type of the function.
	This freedom and constraint is automatically guaranteed by the
inferred typing of {\tt twice}.
\fi%%%%<<<<<<<<<<<<<<<<<<<<<<<<<<<<<<<<<<<<<<<<<<<<<<<<<<<<<<<<<<<<<<<<<<

\chapter{
\txt{\smlsharp{}の拡張機能:レコード多相性}
    {\smlsharp{} feature: record polymorphism}
}
\label{chap:tutorialRecordpolymorphism}

\ifjp%>>>>>>>>>>>>>>>>>>>>>>>>>>>>>>>>>>>>>>>>>>>>>>>>>>>>>>>>>>>>>>>>>>>
	本章以下,\smlsharp{}で導入されたStandard MLの拡張機能を例を用い
て説明します.

	まず本章では,レコード多相性を基礎としたMLでのレコードを用いたプ
ログラミングを学びます.

	レコード多相性は,特別な付加機能ではなく,MLの原則「式は型が正し
い限り自由に組み合わせることができる」に従ってレコードを含むプログラミ
ングを行う上での基本機能です.
	Standard MLにはこの機能が欠けているため,レコードの特性を生かし
たMLスタイルのプログラムが書けませんでした.
	この理由から,本書では,レコードの説明を遅延していました.
	まず,レコードの基礎から学びましょう.
\else%%%%%%%%%%%%%%%%%%%%%%%%%%%%%%%%%%%%%%%%%%%%%%%%%%%%%%%%%%%%%%%%%%%%
	The rest of this part introduce the extensions newly introduce
by \smlsharp{} through examples.

	In this chapter, we learn programming with records based on 
record polymorphism.

	We note that record polymorphism is not a special additional
feature in record programming, but the basic mechanism required to
realize the ML's principle in program construction: ``expressions are
composed as far as they are type consistent''.
% 	レコード多相性は,特別な付加機能ではなく,MLの原則「式は型が正し
% い限り自由に組み合わせることができる」に従ってレコードを含むプログラミ
% ングを行う上での基本機能です.
	Standard ML lacks this basic mechanism and therefore in Standard
ML, one cannot write ML-style program in manipulating records.
	For this reason, we postpone explanation of records until this
chapter.

	Let's start learning record programming basics.
\fi%%%%<<<<<<<<<<<<<<<<<<<<<<<<<<<<<<<<<<<<<<<<<<<<<<<<<<<<<<<<<<<<<<<<<<
\section{\txt{レコード構文}{Record expressions}}
\label{sec:extensionRecordExpression}

\ifjp%>>>>>>>>>>>>>>>>>>>>>>>>>>>>>>>>>>>>>>>>>>>>>>>>>>>>>>>>>>>>>>>>>>>
	MLではレコード式は以下の文法で定義します.
\begin{tt}
\begin{eqnarray*}
expr &\mbox{\ \ }::=& \cdots \\
     &|& \mbox{\tt \{$l_1$=$expr_1$,$\cdots$, $l_n$=$expr_n$\}}
\end{eqnarray*}
\end{tt}
	$l$はラベルと呼ぶ文字列です.
	レコードの定義の簡単な例を示します.
\begin{tt}
\begin{quote}
\# val point = \{X =0.0, Y=0.0\};\\
val point = \{X =0.0, Y=0.0\} : \{X:real, Y:real\}
\end{quote}
\end{tt}
	1から始まる連続した数字をラベルとして持つレコードは,組み型と解
釈され組として表示されます.
\begin{tt}
\begin{quote}
\#  \{1 = 1.1, 2 = fn => x + 1, 3 = "SML\#"\};\\
val it = (1.1, fn, "SML\#") : real * (int -> int) * string\\
\# (1,2);\\
val it = (1,2) : int * int
\end{quote}
\end{tt}
	以前第\ref{sec:tutorialMultiargfun}節で説明した複数引数の
関数は{\tt powerUncurry (n,C)}のように定義しましたが,これも
数字をラベルとしたレコードの受け取ることを表しています.

	「式は型が正しい限り自由に組み合わせることができる」というMLの
原則に従い,レコードの要素にはMLで定義できる任意の型を含むことができま
す.
	従って,レコードを生成する関数に対しても,リストの場合と同様に多
相型が推論されます.
\begin{program}
\# fun f x y = \{X=x, Y=y\};\\
val f = \_ : ['a,'b. 'a -> 'b -> \{X:'a, Y:'b\}]\\
\# fun g x y = (x,y);\\
val g = \_ : ['a,'b. 'a -> 'b -> 'a * 'b]
\end{program}
\else%%%%%%%%%%%%%%%%%%%%%%%%%%%%%%%%%%%%%%%%%%%%%%%%%%%%%%%%%%%%%%%%%%%%
	The syntax of record expressions is given below.
\begin{tt}
\begin{eqnarray*}
expr &\mbox{\ \ }::=& \cdots \\
     &|& \mbox{\tt \{$l_1$=$expr_1$,$\cdots$, $l_n$=$expr_n$\}}
\end{eqnarray*}
\end{tt}
	$l$ denotes a string called {\em labels}.
	Below is a simple record expression.
\begin{program}
\# val point = \{X =0.0, Y=0.0\};\\
val point = \{X =0.0, Y=0.0\} : \{X:real, Y:real\}
\end{program}
	A record whose labels are consecutive numbers starting with {\tt
1} is interpreted as a tuple and printed specially.
\begin{program}
\#  \{1 = 1.1, 2 = fn => x + 1, 3 = "SML\#"\};\\
val it = (1.1, fn, "SML\#") : real * (int -> int) * string\\
\# (1,2);\\
val it = (1,2) : int * int
\end{program}
	In Section~\ref{sec:tutorialMultiargfun}, we defined a
multiple-argument function, namely {\tt powerUncurry (n,C)}.
	This is a function that takes a tuple.

	Just like lists, record elements can contain values of any
types.
% 	「式は型が正しい限り自由に組み合わせることができる」というMLの
% 原則に従い,レコードの要素にはMLで定義できる任意の型を含むことができま
% す.	
	A record forming function is therefore polymorphic, as seen in
the following example.
\begin{program}
\# fun f x y = \{X=x, Y=y\};\\
val f = \_ : ['a,'b. 'a -> 'b -> \{X:'a, Y:'b\}]\\
\# fun g x y = (x,y);\\
val g = \_ : ['a,'b. 'a -> 'b -> 'a * 'b]
\end{program}
\fi%%%%<<<<<<<<<<<<<<<<<<<<<<<<<<<<<<<<<<<<<<<<<<<<<<<<<<<<<<<<<<<<<<<<<<

\section{\txt{フィールド取り出し演算}{Field selection operation}}
\label{sec:extensionFieldselection}

\ifjp%>>>>>>>>>>>>>>>>>>>>>>>>>>>>>>>>>>>>>>>>>>>>>>>>>>>>>>>>>>>>>>>>>>>
	レコードに対する演算は,レコードのフィールドの取り出しです.
	このために,組み込み関数
\begin{tt}\begin{quote}
{\tt \#$l$}
\end{quote}\end{tt}
が用意されています.
	この関数はラベル$l$を含むレコードを受け取り,そのラベルのフィー
ルドの値を返す関数です.
	MLの多相型の原則に従えば,この関数は,ラベル$l$を含む任意のレコー
ド型からそのフィールドの値の型への関数です.
	\smlsharp{}では,以下のような型が推論されます.
\begin{tt}\begin{quote}
\# \#X;\\
val it = \_ : ['a\#\{X:'b\},'b. 'a -> 'b]
\end{quote}\end{tt}
	表現{\tt 'a\#\{X:'b\}}は,型が{\tt 'b}のフィールドXを含む任意のレ
コード型を代表する型変数です.
	従って,推論された型は,{\tt X}を含む任意のレコードを受け取り,
{\tt X}のフィールドの値を返す関数を表しており,{\tt \#X}の動作を正確に
表現しています.
	この関数の適用例を以下に示します.
\begin{tt}\begin{quote}
\# \#X \{X=1.1, Y=2.2\}\\
val it = 1.1 : real
\end{quote}\end{tt}
	これら演算を含む関数に対しては,レコードに関する多相型が推論され
ます.
\begin{tt}\begin{quote}
\# fun f x = (\#X x, \#y y);\\
val f = \_ : ['a\#\{X:'b,Y:\'c\},'b, 'c. 'a -> 'b * 'c]
\end{quote}\end{tt}
	この関数は,引数に対して{\tt X}と{\tt Y}のフィールド取り出しが行
われているため,引数は少なくともこれら2つのフィールドを含むレコードでな
ければなりません.
	\smlsharp{}が推論する型は,その性質を正確に反映した最も一般的な
多相型です.
\else%%%%%%%%%%%%%%%%%%%%%%%%%%%%%%%%%%%%%%%%%%%%%%%%%%%%%%%%%%%%%%%%%%%%
	The basic operation on records is to select a field value by
specifying a field label.
	For this, the following primitive function is defined for any
label $l$.
\begin{program}
\#$l$
\end{program}
	This function takes a record containing a field labeled with
$l$, and returns the value of that label.
	According to the ML principle, this function should be
applicable to any record containing label $l$.
	For this primitive, \smlsharp{} infers the following
polymorphic typing.
\begin{program}
\# \#X;\\
val it = \_ : ['a\#\{X:'b\},'b. 'a -> 'b]
\end{program}
	The notation {\tt 'a\#\{X:'b\}} is a type variable {\tt
'a} which represents an arbitrary record type that contains a field
labeled {\tt X} of type {\tt 'b}.
	The inferred type is a most general type of {\tt \#X}.
	Below show some examples.
\begin{program}
\# \#X \{X=1.1, Y=2.2\};\\
val it = 1.1 : real\\
\# \#X \{X = 1, Y = 2, Z = 3\};\\
val it = 1 : int
\end{program}
	Functions using these operations are polymorphic in record
structures, as seen in the following example.
\begin{tt}\begin{quote}
\# fun f x = (\#X x, \#y y);\\
val f = \_ : ['a\#\{X:'b,Y:\'c\},'b, 'c. 'a -> 'b * 'c]
\end{quote}\end{tt}
	The type of {\tt f} indicates that this is a function that takes
any record containing {\tt X:'b} and {\tt Y:'c} fields and returns a pair 
of type {\tt 'b * 'c}.
	This is a most general polymorphic type of this function.
	So this function can be freely combined with other expressions
as far the combination is type consistent.
\fi%%%%<<<<<<<<<<<<<<<<<<<<<<<<<<<<<<<<<<<<<<<<<<<<<<<<<<<<<<<<<<<<<<<<<<

\section{\txt{レコードパターン}{Record patterns}}
\label{sec:extensionRecordpattern}

\ifjp%>>>>>>>>>>>>>>>>>>>>>>>>>>>>>>>>>>>>>>>>>>>>>>>>>>>>>>>>>>>>>>>>>>>
	レコードのフィールド取り出しは,パターンマッチングの機能を用いて
も行うことができます.
	Standard MLには以下のパターンが含まれています.
\begin{tt}
\begin{eqnarray*}
pat &\mbox{\ \ }::=& \cdots \\
     &|& \mbox{\tt \{$field\_list$\}}
\\
     &|& \mbox{\tt \{$field\_list$,...\}}
\\
field &\mbox{\ \ }::=& \mbox{\tt $l$=$pat$} \ | \ l
\end{eqnarray*}
\end{tt}
	最初のパターンは指定されたフィールドからなるレコードにマッチするパ
ターン,2番目のパターンは少なくとも指定されたフィールドを含むレコードに
マッチするパターンです.
	フィールドはフィールド名のみ書くこともできます.
	その場合,フィールド名と同じ変数が指定されたものとみなされます.
	以下は,レコードパターンを用いたフィールド取り出しの例です.
\begin{tt}\begin{quote}
\# fun f \{X=x,Y=y\} = (x,y);\\
val f = \_ = ['a, 'b. \{X:'a, Y:'b\} -> 'a * 'b]\\
\# fun f \{X=x,Y=y,...\} = (x,y);\\
val f = \_ = ['a\#\{X:'b, Y:'c\}, 'b, 'c. 'a -> 'a * 'b]\\
\# fun f \{X,Y,...\} = (X,Y);\\
val f = \_ = ['a\#\{X:'b, Y:'c\}, 'b, 'c. 'a -> 'a * 'b]
\end{quote}\end{tt}
	このレコードパターンは,他のパターンと自由に組み合わせて使用でき
ます.
\begin{tt}\begin{quote}
\# fun f (\{X,...\}::\_) = X;\\
val f = \_ = ['a\#\{X:'b\},b. 'a list -> 'b]\\
\end{quote}\end{tt}
	この例では,{\tt X}フィールドを含むレコードのリストの先頭のレコー
ドの{\tt X}フィールドの値を返しています.
\else%%%%%%%%%%%%%%%%%%%%%%%%%%%%%%%%%%%%%%%%%%%%%%%%%%%%%%%%%%%%%%%%%%%%
	Record field selection can also be done through pattern
matching.
	Standard ML has the following patterns.
\begin{tt}
\begin{eqnarray*}
pat &\mbox{\ \ }::=& \cdots \\
     &|& \mbox{\tt \{$field\_list$\}}
\\
     &|& \mbox{\tt \{$field\_list$,...\}}
\\
field &\mbox{\ \ }::=& \mbox{\tt $l$=$pat$} \ | \ l
\end{eqnarray*}
\end{tt}
	The first pattern matches a record having the specified set of
labels, and the second pattern matches any record containing the set of
specified labels.
	When only a label is specified in a field then it
is interpreted as a variable with the same name is specified.
	For example, a pattern {\tt \{X, Y\}} is interpreted as 
{\tt \{X = X, Y = Y\}}.
	The following are examples of field selection through record patterns.
\begin{tt}\begin{quote}
\# fun f \{X=x,Y=y\} = (x,y);\\
val f = \_ = ['a, 'b. \{X:'a, Y:'b\} -> 'a * 'b]\\
\# fun f \{X=x,Y=y,...\} = (x,y);\\
val f = \_ = ['a\#\{X:'b, Y:'c\}, 'b, 'c. 'a -> 'a * 'b]\\
\# fun f \{X,Y,...\} = (X,Y);\\
val f = \_ = ['a\#\{X:'b, Y:'c\}, 'b, 'c. 'a -> 'a * 'b]
\end{quote}\end{tt}
	Record pattern can be freely combined with other patterns.
\begin{tt}\begin{quote}
\# fun f (\{X,...\}::\_) = X;\\
val f = \_ = ['a\#\{X:'b\},b. 'a list -> 'b]\\
\end{quote}\end{tt}
	In this example, {\tt f} takes a list of records containing {\tt
X} field, and returns the {\tt X} field of the first record in the list.
\fi%%%%<<<<<<<<<<<<<<<<<<<<<<<<<<<<<<<<<<<<<<<<<<<<<<<<<<<<<<<<<<<<<<<<<<

\section{\txt{フィールドの変更}{Functional record update}}
\label{sec:extensionFieldupdate}

\ifjp%>>>>>>>>>>>>>>>>>>>>>>>>>>>>>>>>>>>>>>>>>>>>>>>>>>>>>>>>>>>>>>>>>>>
	\smlsharp{}言語では,以下の構文によるレコードのフィールドの(関
数的)変更が定義されています.
\begin{tt}
\begin{eqnarray*}
expr &\mbox{\ \ }::=& \cdots \\
     &|& \mbox{\tt $expr$ {\#} \{$l_1$=$expr_1$,$\cdots$, $l_n$=$expr_n$\}}
\end{eqnarray*}
\end{tt}
	この構文は,レコード式$expr$の値をそれぞれ指定された値に変更して
得られるレコードを生成します.
	この構文は,常に新しいレコードを生成し,もとのレコードは変更され
ません.
	この構文もMLの型の原則に従いレコード多相性を持ちます.
	以下は,この構文含む関数の例とその型推論の例です.
\begin{tt}\begin{quote}
\# fun f modify x = x \# \{X=3\};\\
val f = \_ : ['a\#\{X:'b\}, 'b.  'a -> 'a]
\end{quote}\end{tt}
	この構文とレコードパターンを組み合わせた以下の形の関数は,レコー
ドを扱うプログラムでよく登場するイディオムと言えます.
\begin{tt}\begin{quote}
\# fun reStructure (p as \{Salary,...\}) = p \# \{Salary=Salary * (1.0 - 0.0803)\};\\
val f = \_ : ['a\#\{Salary:real\}.  'a -> 'a]
\end{quote}\end{tt}
	関数{\tt reStructure}は,従業員レコード{\tt p}を受け取り,その
{\tt Salary}フィールドを8.03\%減額する関数です.
	型情報から理解されるとおり,この関数は,{\tt Salary}フィールドを
含む任意の従業員レコードに適用可能な,汎用の減額関数となっています.
\else%%%%%%%%%%%%%%%%%%%%%%%%%%%%%%%%%%%%%%%%%%%%%%%%%%%%%%%%%%%%%%%%%%%%
	\smlsharp{} contains functional record update expressions, whose
syntax is given below.
\begin{tt}
\begin{eqnarray*}
expr &\mbox{\ \ }::=& \cdots \\
     &|& \mbox{\tt $expr$ {\#} \{$l_1$=$expr_1$,$\cdots$, $l_n$=$expr_n$\}}
\end{eqnarray*}
\end{tt}
	This expression creates a new record by modifying the value of
each label $l_i$ to $expr_i$.
	This is an expression to create a new record; the original
record $expr$ is not mutated.
	This expression has a polymorphic type according to the ML's
principle of most general typing.
	Below is an example using this expression.
\begin{tt}\begin{quote}
\# fun f modify x = x \# \{X=3\};\\
val f = \_ : ['a\#\{X:'b\}, 'b.  'a -> 'a]
\end{quote}\end{tt}
	The following is a useful idiom in record programming.
\begin{tt}\begin{quote}
\# fun reStructure (p as \{Salary,...\}) = p \# \{Salary=Salary * (1.0 - 0.0803)\};\\
val f = \_ : ['a\#\{Salary:real\}.  'a -> 'a]
\end{quote}\end{tt}
	Function {\tt reStructure} takes an employee record {\tt p} and 
reduces its {\tt Salary} field by 8.03\%.
	As seen in its typing, this function can be applied to any
record as far as it contains a ,{\tt Salary} field.
\fi%%%%<<<<<<<<<<<<<<<<<<<<<<<<<<<<<<<<<<<<<<<<<<<<<<<<<<<<<<<<<<<<<<<<<<

\section{\txt{レコードプログラミング例}{Record programming examples}}
\label{sec:extensionRecordProgramming}

\ifjp%>>>>>>>>>>>>>>>>>>>>>>>>>>>>>>>>>>>>>>>>>>>>>>>>>>>>>>>>>>>>>>>>>>>
	多相レコード演算機能をもった言語(現在のところ\smlsharp{}しか
ありませんが)では,レコード構造は,種々の属性に着目したモジュール性の高
いプログラムを型安全に構築していく上で大きな武器となります.
	これによって,多相関数による汎用性あるプログラムが,大規模なプロ
グラム開発にスケールするようになります.

	そのような例として,放物線を描いて落下する物体をプロットする場合
を考えてみましょう. 
	物体の位置は直交座標で表現するとします.
	そして,物体は,現在位置を表す{\tt X:real}と{\tt Y:real}のフィー
ルドと現在の速度を表すを含むレコード{\tt Vx:real}と{\tt Vy:real}のフィー
ルドを含むレコードで表現することにします.
	これだけを決めておけば,物体のその他の属性や構造とは独立に,放物
線上を移動する関数を設計できます.
	物体の運動は,物体を単位時間経過後の位置に移動させる関数{\tt
tic}を書き,この関数をくり返し呼び続ければ実現できます.
\begin{tt}\begin{quote}
val tic : ['a\#\{X:real,Y:real, Vx:real, Vy:real\}. 'a * real -> 'a]
\end{quote}\end{tt}
	簡単のために単位時間を1とします.
	直行座標系での物体の移動ベクトルは,それぞれの座標のベクトルの合
成ですから,{\tt X}軸と{\tt Y}軸について独立に関数を書きそれを合成すれば
よいはずです.
	位置は,それぞれ現在の位置に,単位時間あたりの移動距離,つまり速
度を加えればよいので,{\tt X}軸方向は{\tt Y}軸方向それぞれ,独立に以下の
ようにコードできます.
	(対話型セッションで型と共に示すます.)
\begin{tt}\begin{quote}
\# fun moveX (p as \{X,Vx,...\}, t) = p \# \{X = X + Vx\};\\
val moveX : ['a\#\{X:real,Vx:real\}. 'a -> 'a]\\
\# fun moveY (p as \{Y,Vy,...\}, t) = p \# \{Y = Y + Vy\};\\
val moveY : ['a\#\{X:real,Vx:real\}. 'a -> 'a]
\end{quote}\end{tt}
	次に,それぞれの軸方向の速度を変更する関数を書きます.
	{\tt X}軸方向は等速運動をし,{\tt Y}軸方向は重力加速度による等加
速度運動をする場合の関数は,位置の変更と同様,以下のように簡単にコードで
きます.
\begin{tt}\begin{quote}
\# fun accelerateX (p as \{Vx,...\}, t) = p\\
val accelerateX : ['a\#\{Vx:real\}. 'a -> 'a]\\
\# fun accelerateY (p as \{Vy,...\}, t) = p \# \{Vy = Vy + 9.8\}\\
val accelerateY : ['a\#\{Vy:real\}. 'a -> 'a]\\
\end{quote}\end{tt}
	{\tt accelerateX}は変化がないので,不要ですが,雛形として記述し
ておくと将来の変更に便利です.
	
	単位時間後の物体を求める関数{\tt tic}は,以上の関数を合成するこ
とによって得られます.
\begin{tt}\begin{quote}
fun tic p =\\
let\\
\myem  val p = accelerateX p\\
\myem  val p = accelerateY p\\
\myem  val p = moveX p\\
\myem  val p = moveY p\\
in\\
\myem  p\\
end
\end{quote}\end{tt}
	このようにして得られたプログラムはモジュール性に富みかつ型安全で
堅牢なシステムとなります.
	たとえば,水平方向の速度は風の抵抗に現在の速度に対して10\%減少
していく運動なども,{\tt accelerateY}を
\begin{tt}\begin{quote}
\# fun accelerateX (p as \{Vx,...\}, t) = p \# \{Vx = Vx * 0.90\}\\
val accelerateX : ['a\#\{Vx:real\}. 'a -> 'a]\\
\end{quote}\end{tt}
と変更するだけで実現できます.

	この{\tt tic}関数に対しては,本節の冒頭で書いた型が推論され,位
置と速度属性をもつ任意の物体に適用可能です.
\else%%%%%%%%%%%%%%%%%%%%%%%%%%%%%%%%%%%%%%%%%%%%%%%%%%%%%%%%%%%%%%%%%%%%
	In a language supporting record polymorphic (currently
\smlsharp{} seems to be the only one), one can write generic code by
focusing only on relevant properties of problems.
	This provide a powerful tool for modular construction of
programs that scales to large software development.
	
	To understand a flavor of this feature, let us consider a
problem to simulate object movement in a parabolic path in a Cartesian
coordinate system.
	An object can be represented as a record containing
{\tt X:real} and {\tt Y:real} fields representing the current
position vector, and {\tt Vx:real} and {\tt Vy:real} fields representing
the current velocity vector.
	An object may have a lot of other properties, but for writing
a program to simulate objects movement, these attributes are sufficient.
	Object movement is simulated by repeatedly applying a function
{\tt move} that moves an object from the current location to the
location after one unit of time.
\begin{program}
val move : ['a\#\{X:real,Y:real, Vx:real, Vy:real\}. 'a * real -> 'a]
\end{program}
	In this exercise, we assume that unit of time is 1 second.
	In the Cartesian coordinate system, since a position vector and
a velocity vector are compositions of those of the two coordinates, 
we can write write functions on {\tt X} and {\tt Y} coordinates
independently and compose them.
	The next position is obtained by adding the velocity.
	So we can write functions on {\tt X} and {\tt Y} independently
as below (which shows the code and its typing in an interactive
session).
\begin{program}
\# fun moveX (p as \{X,Vx,...\}, t) = p \# \{X = X + Vx\};\\
val moveX : ['a\#\{X:real,Vx:real\}. 'a -> 'a]\\
\# fun moveY (p as \{Y,Vy,...\}, t) = p \# \{Y = Y + Vy\};\\
val moveY : ['a\#\{X:real,Vx:real\}. 'a -> 'a]
\end{program}
	Next, we need to write functions that change velocities.
	Here we assume that on {\tt X} coordinate, objects maintain its
uniform motion, and on {\tt Y}, objects are uniformly accelerated by
gravity.
	Then acceleration functions can be code as follows.
\begin{tt}\begin{quote}
\# fun accelerateX (p as \{Vx,...\}, t) = p\\
val accelerateX : ['a\#\{Vx:real\}. 'a -> 'a]\\
\# fun accelerateY (p as \{Vy,...\}, t) = p \# \{Vy = Vy + 9.8\}\\
val accelerateY : ['a\#\{Vy:real\}. 'a -> 'a]\\
\end{quote}\end{tt}
	{\tt accelerateX} is constant and therefore not needed in this
simple case, but we write one for future refinement.
	
	The function {\tt next} can be obtained by composing all of them
below.
\begin{tt}\begin{quote}
fun tic p =\\
let\\
\myem  val p = accelerateX p\\
\myem  val p = accelerateY p\\
\myem  val p = moveX p\\
\myem  val p = moveY p\\
in\\
\myem  p\\
end
\end{quote}\end{tt}
	The resulting code is highly modular and type safe, and can be
easily refined.
	For example, if we want to add deacceleration on {\tt X}
coordinate by 1\% per second, one need only re-write {\tt accelerateY}
to the following.
\begin{tt}\begin{quote}
\# fun accelerateX (p as \{Vx,...\}, t) = p \# \{Vx = Vx * 0.99\}\\
val accelerateX : ['a\#\{Vx:real\}. 'a -> 'a]\\
\end{quote}\end{tt}

	For this {\tt next} function, \smlsharp{} infers the type we 
designed at the beginning of this section, and therefore can be applied
to any object having many other attributes.
\fi%%%%<<<<<<<<<<<<<<<<<<<<<<<<<<<<<<<<<<<<<<<<<<<<<<<<<<<<<<<<<<<<<<<<<<


\section{\txt{オブジェクトの表現}{Representing objects}}
\label{sec:extensionObject}

\ifjp%>>>>>>>>>>>>>>>>>>>>>>>>>>>>>>>>>>>>>>>>>>>>>>>>>>>>>>>>>>>>>>>>>>>
	レコードは種々のデータ構造の定義の基本であり,データベースや
オブジェクト指向プログラミングなどで使われています.
	第\ref{chap:tutorialDatabase}章で詳しく説明する通り,多相レコー
ド操作を基本に,データベースの問い合わせ言語SQLを完全な形でML言語内にシー
ムレスに取り込むことができます.
	オブジェクト指向プログラミングに関しては,計算モデルが異なるため,
その機能を完全に表現することはできませんが,オブジェクトの操作に関しては,
多相型レコードで表現できます.
	
	オブジェクトは,状態を持ち,メソッドセレクタをメッセージとして受
け取り,オブジェクトの属するクラスの対応するメソッドを起動し状態を更新し
ます.
	クラスは,メッセージによって選び出されるメッソド集合ですから,関
数のレコードで表現できます.
	各メソッド関数は,オブジェクト状態を受け取りそれを更新するコード
です.
	例えば,
{\tt X}座標,
{\tt Y}座標,
{\tt Color}属性
を持ち得る
{\tt pointClass}
のオブジェクト表現を考えてみましょう.
	オブジェクトは,例えば{\tt \{X = 1.1, Y = 2.2\}}のようなレコード
への参照を内部に持つとします.
	すると,各メソッドはこのオブジェクトを{\tt self}として受け取り更
新する関数と表現できます.
	例えば{\tt X}座標に新しい値をセットするメソッドは,以下のように
コードできます.
\begin{tt}
\begin{quote}
\# fn self => fn x => self := (!self \# \{X = x\});\\
val it = \_ : ['a\#\{X: 'b\},'b. 'a ref -> 'b -> unit]
\end{quote}
\end{tt}
	このメソッドは,{\tt X}を含む任意のオブジェクトに適用できます.
	これらメソッドにメソッド名を付けてレコードにしたものをクラスと考
えます.
	例えば,{\tt pointClass}は以下のように定義できます.
\begin{tt}
\begin{quote}
val pointClass =\\
\{\\
\myem  getX = fn self => \#X (!self),\\
\myem  setX = fn self => fn x => self := (!self \# {X = x}),\\
\myem  getY = fn self => \#Y (!self),\\
\myem  setY = fn self => fn x => self := (!self \# {Y = x}),\\
\myem  getColor = fn self => \#Color (!self),\\
\myem  setColor = fn self => fn x => self := (!self \# {Color = x})\\
\}
\end{quote}
\end{tt}
	オブジェクトは,メッセージを受け取り,このクラスの中からメソッド
スイートの中から,対応する関数を選択し自分の状態に適応する関数です.
	関数を値として使用できるMLでは,以下のようにコードできます.
\begin{tt}
\begin{quote}
local\\
\myem  val state =  ref \{ X = 0.0, Y = 0.0 \}\\
in \\
\myem val myPoint = fn method => method pointClass state\\
end
\end{quote}
\end{tt}
	{\tt Color}属性をもつオブジェクトも同様です.
\begin{tt}
\begin{quote}
local\\
\myem  val state =  ref \{ X = 0.0, Y = 0.0, Color = "Red" \}\\
in \\
\myem val myColorPoint = fn method => method pointClass state\\
end
\end{quote}
\end{tt}
	この定義の下で,以下のようなオブジェクト指向スタイルのコーディン
グが可能です.
\begin{tt}\begin{quote}
\# myPoint \# setX 1.0;\\
val it = () : unit\\
\# myPoint \# getX;\\
val it = 1.0 : real\\
\# myColorPoint \# getX;\\
val it = 0.0 : real\\
\# myColorPoint \# getColor;\\
val it = "Red" : string\\
\# myPoint \# getColor;\\
(interactive):15.1-15.12 Error:\\
\myem  (type inference 007) operator and operand don't agree\\
\myem ...
\end{quote}\end{tt}
	最後の例のように,完全な静的な型チェックも保証されています.
\else%%%%%%%%%%%%%%%%%%%%%%%%%%%%%%%%%%%%%%%%%%%%%%%%%%%%%%%%%%%%%%%%%%%%
	Records are the fundamental data structures, and they are the
basis for various data manipulation models such as relational databases
and object-oriented programming.
	As explained in Chapter\ref{chap:tutorialDatabase}, 
\smlsharp{} seamlessly integrate SQL based on record polymorphism.
	Object-oriented programming is based on a different computation
model than functional programming, and we do not hope to represent it in
a functional language.
	However, generic object manipulation underlying object-oriented
programming is naturally represented using record polymorphism.
	
	An object has a statue, receives method selector and invoke the
designated method on its state.
	A class can be regarded as the set of methods that belongs to the
class.
	So we can represent a class structure as a record of methods.
	Each method is coded as a function that takes a state of an
object and update it. 
	For example, consider an object in a {\tt pointClass} having
{\tt X} coordinate,
{\tt Y} coordinate,
{\tt Color} property.
	An object state can be represented by a reference to a record of
the those values such as {\tt \{X = 1.1, Y = 2.2\}}. 
	Then a method can be defined as a function that takes an object
state as {\tt self} and update the state.
	For example, a method to set a value in the {\tt X} coordinate
can be coded as follows.
\begin{program}
\# fn self => fn x => self := (!self \# \{X = x\});\\
val it = \_ : ['a\#\{X: 'b\},'b. 'a ref -> 'b -> unit]
\end{program}
	This method can be applied to any objects that contain 
{\tt X} attributes.
	A class can be a record consisting of these methods with
appropriate names.
	For example,{\tt pointClass} can be defined as below.
\begin{program}
val pointClass =\\
\{\\
\myem  getX = fn self => \#X (!self),\\
\myem  setX = fn self => fn x => self := (!self \# {X = x}),\\
\myem  getY = fn self => \#Y (!self),\\
\myem  setY = fn self => fn x => self := (!self \# {Y = x}),\\
\myem  getColor = fn self => \#Color (!self),\\
\myem  setColor = fn self => fn x => self := (!self \# {Color = x})\\
\}
\end{program}
	An object receives a message and invoke the corresponding method
on itself.
	In \smlsharp{}, this is coded below.
\begin{program}
local\\
\myem  val state =  ref \{ X = 0.0, Y = 0.0 \}\\
in \\
\myem val myPoint = fn method => method pointClass state\\
end
\end{program}
	We can similarly define an object having {\tt Color} attribute.
\begin{program}
local\\
\myem  val state =  ref \{ X = 0.0, Y = 0.0, Color = "Red" \}\\
in \\
\myem val myColorPoint = fn method => method pointClass state\\
end
\end{program}

	Under these definition, we can write the following code.
\begin{tt}\begin{quote}
\# myPoint \# setX 1.0;\\
val it = () : unit\\
\# myPoint \# getX;\\
val it = 1.0 : real\\
\# myColorPoint \# getX;\\
val it = 0.0 : real\\
\# myColorPoint \# getColor;\\
val it = "Red" : string\\
\# myPoint \# getColor;\\
(interactive):15.1-15.12 Error:\\
\myem  (type inference 007) operator and operand don't agree\\
\myem ...
\end{quote}\end{tt}
	As shown in the last example, the system detect all the type
errors statically.
\fi%%%%<<<<<<<<<<<<<<<<<<<<<<<<<<<<<<<<<<<<<<<<<<<<<<<<<<<<<<<<<<<<<<<<<<

\section{\txt{多相バリアントの表現}{Polymorphic variants}}
\label{sec:extensionVariant}

\ifjp%>>>>>>>>>>>>>>>>>>>>>>>>>>>>>>>>>>>>>>>>>>>>>>>>>>>>>>>>>>>>>>>>>>>
	レコード構造は,各要素に名前を付けたデータ構造です.
	この構造の双対をなす概念にラベル付きバリアントがあります.
	多くの方にとって不慣れな概念と思われますが,必要とされる機能はレ
コードとほぼ同一であるため,多相レコードが表現できれば,多相バリアントも
表現できます.
	多相バリアントに興味のある方への参考に,以下,その考え方とコード
例を簡単に紹介します.

	レコードが名前付きのデータの集合であるのに対して,ラベル付きバリ
アントは,名前付きの処理の集合の下での処理要求ラベルのついたデータと考え
ることができます.
	直行座標系と極座標系の両方のデータを扱いたい場合を考えます.
	それぞれのデータは処理の仕方がちがいますから,データにどちらの座
標系かを表すタグ(名前)を付けます.
	これらデータを多相関数と一緒に使用できるようにする機構が多相バリ
アントです.
	これは,バリアントタグを,メソッド集合から適当なメソッドを選び出
すセレクタと考えると,多相レコードを使って表現できます.
	同一の点の2つのそれぞれの座標系での表現は,例えば以下のようにコー
ド可能です.		
\begin{tt}\begin{quote}
\# val myCPoint = fn M => \#CPoint M \{X=1.0, Y = 1.0\};\\
val myCPoint = fn : ['a\#\{CPoint: \{X: real, Y: real\} -> 'b\}, 'b. 'a -> 'b]\\
\# val myPPoint = fn M => \#PPoint M \{r=1.41421356, theta = 45.0 \};\\
val myPPoint = fn : ['a\#\{PPoint: \{r: real, theta: real\} -> 'b\}, 'b. 'a -> 'b]
\end{quote}\end{tt}
	つまりタグ$T$をもつバリアントは,タグ$T$を処理するメソッドスイー
トを受け取り,自分自身にその処理を呼び出す行うオブジェクトと表現します.
	あとは,必要な処理をそれぞれの表現に応じて書き,タグに応じた名前
をもつレコードにすればよいわけです.
	例えば,原点からの距離を計算するメソッドは以下のように実現できます.
\begin{tt}\begin{quote}
val distance = \\
\{\\
\myem CPoint = fn {x,y,...} => Real.Math.sqrt (x *x + y* y),\\
\myem PPoint = fn {r, theta,...} => r\\
\};
\end{quote}\end{tt}
	多相バリアントデータをメソッドスイートに適用することによって起動できます.
\begin{tt}\begin{quote}
\# myCPoint distance ;\\
val it = 1.41421356 : real\\
\# myPPoint distance ;\\
val it = 1.41421356 : real
\end{quote}\end{tt}
	これによって,表現の異なるデータのリストなどを多相関数によって安
全に処理できるようになります.
\else%%%%%%%%%%%%%%%%%%%%%%%%%%%%%%%%%%%%%%%%%%%%%%%%%%%%%%%%%%%%%%%%%%%%
	Records are labeled collection of component values.
	There is a dual concept of this structure, namely {\em labeled
variants}.
	This may not be familiar to many of you, but the essential
ingredients are the same as those of records, so in a system where
polymorphic records are supported, polymorphic variants can also be
represented.
	For a type theoretical account, see \cite{ohor95toplas}.
	In this section, we briefly introduce them through simple
examples.

	Once you understand records as labeled collections of values,
you can think of labeled variants as a value attached with a service 
request label in a context where a labeled collection of services are
defined. 
	For a simple example, let us consider a system where we have two
point representations, one in Cartesian coordinates and the other in
polar coordinates.
	In this system, each representation is processed differently, so
each point data is attached a label indicating its representation.
	Polymorphic variant is a mechanisms to those data with
polymorphic functions.
	By regarding the data label as a service selector from a given
set of services, this system can be represented by polymorphic records.
	For example, the following show two representations of the same
point.
\begin{tt}\begin{quote}
\# val myCPoint = fn M => \#CPoint M \{X=1.0, Y = 1.0\};\\
val myCPoint = fn : ['a\#\{CPoint: \{X: real, Y: real\} -> 'b\}, 'b. 'a -> 'b]\\
\# val myPPoint = fn M => \#PPoint M \{r=1.41421356, theta = 45.0 \};\\
val myPPoint = fn : ['a\#\{PPoint: \{r: real, theta: real\} -> 'b\}, 'b. 'a -> 'b]
\end{quote}\end{tt}
	A variant data with a tag $T$ is considered as an object that
receives a method suits, select appropriate suit using $T$ as the
selector, and applies the selected method to itself.
	One you understand this idea, then all you have to do is to
write up the set of necessary methods for each tag.
	For example, a set of methods to compute the distance from the
origin of coordinates can be coded as below.
\begin{program}
val distance = \\
\{\\
\myem CPoint = fn {x,y,...} => Real.Math.sqrt (x *x + y* y),\\
\myem PPoint = fn {r, theta,...} => r\\
\};
\end{program}
	This method suit is invoked by applying an variant object to it.
\begin{tt}\begin{quote}
\# myCPoint distance ;\\
val it = 1.41421356 : real\\
\# myPPoint distance ;\\
val it = 1.41421356 : real
\end{quote}\end{tt}
	In this way, various heterogeneous collections can be processed
in type safe way.
\fi%%%%<<<<<<<<<<<<<<<<<<<<<<<<<<<<<<<<<<<<<<<<<<<<<<<<<<<<<<<<<<<<<<<<<<

\chapter{
\txt{\smlsharp{}の拡張機能:その他の型の拡張}
    {\smlsharp{} feature:other type system extensions}}
\label{chap:tutorialOthertyping}

\ifjp%>>>>>>>>>>>>>>>>>>>>>>>>>>>>>>>>>>>>>>>>>>>>>>>>>>>>>>>>>>>>>>>>>>>
	\smlsharp{}では,レコード多相性に加え,以下の2つの拡張をしてい
ます.
\begin{enumerate}
\item ランク1多相性
\item 第一級のオーバーローディング
\end{enumerate}
	本章では,これら機能を簡単に説明します.
\else%%%%%%%%%%%%%%%%%%%%%%%%%%%%%%%%%%%%%%%%%%%%%%%%%%%%%%%%%%%%%%%%%%%%
	In addition to record polymorphism, \smlsharp{} extend the Standard
ML type system  with the following.
\begin{enumerate}
\item rank 1 polymorphism, and
\item first-class overloading.
\end{enumerate}
	This chapter briefly introduce them.
\fi%%%%<<<<<<<<<<<<<<<<<<<<<<<<<<<<<<<<<<<<<<<<<<<<<<<<<<<<<<<<<<<<<<<<<<

\section{\txt{ランク1多相性}{Rank 1 polymorphism}}
\label{sec:extensionRank1}

\ifjp%>>>>>>>>>>>>>>>>>>>>>>>>>>>>>>>>>>>>>>>>>>>>>>>>>>>>>>>>>>>>>>>>>>>
	Standard MLの多相型システムが推論できる多相型は,すべての多相型
変数が最も外側で束縛される型です.
	例えば関数
\begin{program}
fun f x y = (x,y)
\end{program}
に対しては,
\begin{program}
val f = \_ : ['a,'b. 'a -> 'b -> 'a * 'b]
\end{program}
の型が推論されます.
	\smlsharp{}では,このような関数に対して以下のようなネストした多
相型を推論可能に拡張しています.
\begin{program}
\# fun f x y = (x,y);\\
val f = \_ : ['a. 'a -> ['b.'b -> 'a * 'b]]
\end{program}
	この型は,{\tt 'a}としてある型$\tau$を受け取って多相関数
{\tt ['b.'b -> $\tau$ * 'b]}を返す関数型です.
	実際,以下のような動作をします.
\begin{program}
\# f 1;\\
val it = \_ : ['b.'b -> int * 'b]\\
\# it "ML";\\
val it = (1,"ML") : int * string
\end{program}

	型変数を$t$,種々の型にインスタンス化ができない単相型を$\tau$,
多相型を$\sigma$で表すと,Standard ML言語の多相型はおおよそ以下のように
定義されるランク0と呼ばれる型です.
\begin{eqnarray*}
\tau &::=& t \vbar b \vbar \tau \func \tau \vbar \tau * \tau
\\
\sigma &::=& \tau \vbar \forall (t_1,\ldots,t_n).\tau
\end{eqnarray*}
	これに対して,\smlsharp{}で推論可能な多相型は,多相型が関数の引
数の位置以外の位置にくることを許す以下のようなランク1と呼ばれる型です.
\begin{eqnarray*}
\tau &::=& t \vbar b \vbar \tau \func \tau \vbar \tau * \tau
\\
\sigma &::=& \tau \vbar \forall (t_1,\ldots,t_n).\tau 
\vbar \tau \func \sigma
\vbar \sigma * \sigma 
\end{eqnarray*}
	この拡張は,もともとレコード多相性の効率的な実現を目指した技術的
な拡張であり,純粋なMLの型理論の枠組みでは殆ど違いはありません.
	しかし,Standard MLの改訂版で導入された値多相性制約のもとでは,
重要な拡張となっています.
	次節で値多相性とランク1多相性の関連を解説します.
\else%%%%%%%%%%%%%%%%%%%%%%%%%%%%%%%%%%%%%%%%%%%%%%%%%%%%%%%%%%%%%%%%%%%%
	Polymorphic types Standard ML type system can infer are those
whose type variables are bound at the top-level.
	For example, for a function
\begin{program}
fun f x y = (x,y)
\end{program}
the following type is inferred.
\begin{program}
val f = \_ : ['a,'b. 'a -> 'b -> 'a * 'b]
\end{program}
	In contrast, \smlsharp{} infers the following nested
polymorphic type
\begin{program}
\# fun f x y = (x,y);\\
val f = \_ : ['a. 'a -> ['b.'b -> 'a * 'b]]
\end{program}
	This type can be think of as a type function that receives a
type $\tau$ through type variable {\tt 'a} and return a polymorphic type
{\tt ['b.'b -> $\tau$ * 'b]}.
	It behaves as follows.
\begin{program}
\# f 1;\\
val it = \_ : ['b.'b -> int * 'b]\\
\# it "ML";\\
val it = (1,"ML") : int * string
\end{program}

	Let $t$ represent type variables,
$\tau$ monomorphic types (possibly including type variables), and let
$\sigma$ polymorphic types.
	Then the set of polymorphic types Standard ML can infer is
roughly given by the following grammar. 
\begin{eqnarray*}
\tau &::=& t \vbar b \vbar \tau \func \tau \vbar \tau * \tau
\\
\sigma &::=& \tau \vbar \forall (t_1,\ldots,t_n).\tau
\end{eqnarray*}
	We call this set rank 0 types.

	In contract, \smlsharp{} can infer the following set called rank
1 types.
\begin{eqnarray*}
\tau &::=& t \vbar b \vbar \tau \func \tau \vbar \tau * \tau
\\
\sigma &::=& \tau \vbar \forall (t_1,\ldots,t_n).\tau 
\vbar \tau \func \sigma
\vbar \sigma * \sigma 
\end{eqnarray*}
	We have introduced this as a technical extension for making 
compilation of record polymorphism  more efficient.
	In a type system for pure ML term without imperative feature,
this extension does not increase the expressive power of ML.
% 	この拡張は,もともとレコード多相性の効率的な実現を目指した技術的
% な拡張であり,純粋なMLの型理論の枠組みでは殆ど違いはありません.
	However, with the value restriction introduced in the revision
of Standard ML type system, rank 1 extension become important extension.
	Next section explain this issue.
\fi%%%%<<<<<<<<<<<<<<<<<<<<<<<<<<<<<<<<<<<<<<<<<<<<<<<<<<<<<<<<<<<<<<<<<<

\section{\txt{ランク1多相性による値多相性制約の緩和}
{Value polymorphism restriction and rank 1 typing}
}
\label{sec:extensionValuerestriction}

\ifjp%>>>>>>>>>>>>>>>>>>>>>>>>>>>>>>>>>>>>>>>>>>>>>>>>>>>>>>>>>>>>>>>>>>>
	MLの多相型システムは,第\ref{sec:tutorialRef}節で学んだ手続き的
機能を導入すると,整合性が崩れることが知られています.
	例えば参照型構成子{\tt ref}は,任意に型の参照を作ることができる
ので,MLの型システムの原則に従いナイーブに導入すると,以下のような多相
型を持つように思います.
\begin{program}
val ref : ['a. 'a -> 'a ref]\\
val := :  ['a. 'a ref * 'a -> unit]
\end{program}
	しかし,この型付けのもとでMLの型推論を実行すると,以下のようなコー
ドが書けてしまい,型システムが破壊されていまいます.
\begin{program}
val idref = ref(fn x => x);\\
val \_ = idref := (fn x => x + 1);\\
bval \_ = !idref "wrong"
\end{program}
	まず1行目で{\tt idref}の型が{\tt ['a. ('a -> 'a) ref]}と推理さ
れます.
	この型は{\tt (int -> int) ref}としても使用できる型です.
	また,代入演算子{\tt :=}も多相型を持ちますから
{\tt (int -> int) ref}型と{\tt int -> int}型を受け取ることができます.
	従って2行目の型が正しく受け付けられ実行されます.
	この時{\tt idref}の値は{\tt int -> int}の値への参照に変更されて
いますが,その型は以前のまま{\tt ['a. ('a -> 'a) ref]}です.
	従って3行目の型が正しく,実行されてしまいますが,その結果は,算
術演算が文字列に適用され実行時型エラーとなります.

	この問題を避けるために,MLでは,
\begin{quote}
多相型を持ち得る式は値式に限る
\end{quote}
という値多相性制約が導入されています.
	値式とは,計算が実行されない式のことであり,定数や変数,関数式
{\tt fn x => x},それらの組などです.
	関数呼び出しを行う式は値式ではありません.
	たとえば,{\tt ref (fn x => x)}はプリミティブ{\tt ref}が呼び
出されますから値式ではありません.
	この制約によって,関数呼び出しの結果に多相型を与えることが禁止さ
れます.
	これによって,参照を作り内部に参照を含む{\tt ref (fn x => x)}の
ような式は多相型を持てなくなり,上記の不整合が避けられます.

	\smlsharp{}もこの制約に従っています.
	しかし,\smlsharp{}では,多相型が組の要素や関数の結果などの部分式
にも与えることができるため,この値式制約が限定されたものとなり,ユーザに
とって大幅にその制約が緩和される効果があります.
	例えば以下の関数定義と関数適用をみてみましょう.
\begin{program}
val f = fn x => fn y => (x,ref y)\\
val g = f 1\\
\end{program}
	Standard MLの型システムでは,関数{\tt f}には
\begin{program}
val f = \_ : ['a, 'b. 'a -> 'b -> 'a * 'b ref]
\end{program}
のような多相型があたえられますが,この関数の適用{\tt f 1}の結果に現れる
型変数を再び束縛することは値多相性制約に反しますので,{\tt g}には多相型
を与えることができません.
	これに対して\smlsharp{}では,ランク1多相性により,この関数に以
下の型が推論されます.
\begin{program}
val f = \_ : ['a. 'a -> ['b. 'b -> 'a * 'b ref]]
\end{program}
	この結果の型に対して推論された多相型は,関数適用の前に推論された
ものであり,関数適用の結果に対して推論されたものではありません.
	従って,関数適用の結果はその多相型に{\tt 'a}の実際の値が代入され
た以下のような多相型となります.
\begin{program}
val g = \_ :  ['a. 'b -> int * 'b ref]
\end{program}
\else%%%%%%%%%%%%%%%%%%%%%%%%%%%%%%%%%%%%%%%%%%%%%%%%%%%%%%%%%%%%%%%%%%%%
	It is known that ML's polymorphic typing breaks down when 
imperative features in Section~\ref{sec:tutorialRef} are introduced.
	Since reference cells can be created for value of any type,
one might think that they are polymorphic primitives having the
following types.
\begin{program}
val ref : ['a. 'a -> 'a ref]\\
val := :  ['a. 'a ref * 'a -> unit]
\end{program}
	However, under these typings, the following incorrect code could 
be written, which would destroy the system.
\begin{program}
val idref = ref(fn x => x);\\
val \_ = idref := (fn x => x + 1);\\
bval \_ = !idref "wrong"
\end{program}
	{\tt ['a. ('a -> 'a) ref]} is inferred for {\tt idref}.
	This type can be used as {\tt (int -> int) ref}, and 
so does the type of {\tt :=}.
	So the second line code is accepted.
	At this point, {\tt idref} is changed to a reference of value of
type {\tt int -> int}, but its type remains to be {\tt ['a. ('a -> 'a)
ref]} and the third line is accepted, and executed.
	However, this result in applying an integer function to a string.

	In order to avoid this problem, Standard ML introduces the
restriction:
\begin{quote}
polymorphism is restricted to {\em value expressions}
\end{quote}
	This is called {\em value polymorphism restriction}.
	Value expressions are those that do not execute any computation,
such as constants, variables, and function expressions.
	{\tt ref (fn x => x)} calls primitive {\tt ref} and therefore
not a value expression.
	With this restriction, function applications cannot be given a
polymorphic type.
	In this (rather crude) way, ML prevents a function containing a
reference cannot have a polymorphic type, which cause the above problem.

	\smlsharp{} follows this value polymorphism restriction.
	However, in \smlsharp{},polymorphic types can be given to
sub-expression such as function body or record component, value
restriction is significantly reduced in practice.
	For example, consider the following function definition and
application.
\begin{program}
val f = fn x => fn y => (x,ref y)\\
val g = f 1\\
\end{program}
	In Standard ML, {\tt f} is given the following type.
\begin{program}
val f = \_ : ['a, 'b. 'a -> 'b -> 'a * 'b ref]
\end{program}
	Then an application of {\tt f} such as {\tt f\ 1} contains a
free type variable, but since this is not a value expression and
therefore the free type variable cannot be rebind to make a polymorphic
type.
	In contrast, \smlsharp{} infer the following rank 1 type for
{\tt f}.
\begin{program}
val f = \_ : ['a. 'a -> ['b. 'b -> 'a * 'b ref]]
\end{program}
	The function result type {\tt ['b. 'b -> 'a * 'b ref]} is
inferred before the function application.
	As a result, the type of {\tt f\ 1} is obtained by replacing
{\tt 'a } with {\tt int} in {\tt ['b. 'b -> 'a * 'b ref]} without
re-inferring a polymorphic type as seen below.
\begin{program}
val g = \_ :  ['a. 'b -> int * 'b ref]
\end{program}
\fi%%%%<<<<<<<<<<<<<<<<<<<<<<<<<<<<<<<<<<<<<<<<<<<<<<<<<<<<<<<<<<<<<<<<<<

\section{
\txt{第一級オーバーローディング}
    {First-class overloading}
}
\label{sec:extensionOverloading}

\ifjp%>>>>>>>>>>>>>>>>>>>>>>>>>>>>>>>>>>>>>>>>>>>>>>>>>>>>>>>>>>>>>>>>>>>
	MLでよく使われる組込み演算は,オーバーロードされています.
	例えば,加算演算子{\tt +}は,{\tt int, real, word}などを含む型の
加算に使用できます.
\begin{tt}
\begin{quote}
1 + 1;\\
val it = 2 : int\\
1.0 + 1.0;\\
val it = 2.0 : real\\
0w1 + 0w1;\\
val it = 0wx2 : word
\end{quote}
\end{tt}
	この{\tt +}は多相関数と違い,適用される値によって対応する組込
み演算(上の例の場合{\tt Int.+, Real.+, Word.+}のどれか)が選択されます.
	Standard MLでは,このオーバーロード機構はトップレベルで解決する戦
略がとられています.
	トップレベルでもオーバーロードが解消しなかった場合,コンパイラはあ
らかじめ決められた型を選択します.
	例えば,関数
\begin{tt}
\begin{quote}
fun plus x = x + x
\end{quote}
\end{tt}
を定義すると,演算子のオーバーロードを解決するための{\tt x}に関する型情報
がないため,あらかじめ決められている{\tt int}型が選択されます.
	
	この戦略は,データベースの問い合わせ言語をシームレスに取り込む上
で,大きな制約となります.
	SQLで提供されている種々の組込み演算の多くは,データベースのカラ
ムに格納できる型に対してオーバーロードされています.
	これらの演算子の型をすべてSQL式が書かれた時点で決定してしまうと,
SQLの柔軟性が失われ,\smlsharp{}からデータベースを柔軟に使用する上での利
点を十分に活かすことができません.
	そこで\smlsharp{}では,オーバーロードされた演算子を,第一級の関数
と同様に使用できる機構を導入しています.
	オーバーロードされた演算子を含む式は,その演算子が対応可能な型に関
して多相的になります.
	例えば関数{\tt plus}に対して以下のような型が推論されます.
\begin{tt}
\begin{quote}
fun plus x = x + x;\\
val plus = fn\\
\myem  : ['a::{int, IntInf.int, real, Real32.real, word, Word8.word}. 'a -> 'a]
\end{quote}
\end{tt}
	この関数は,型変数{\tt 'a}に{\tt int, IntInf.int, real,
Real32.real, word, Word8.word}の何れかを代入しえ得られる型をもつ関数とし
て使用できます.
	型変数に付加されている制約{\tt ::\{...\}}は,オーバーロード演算子
に許される型に制限があるための制約です.
\else%%%%%%%%%%%%%%%%%%%%%%%%%%%%%%%%%%%%%%%%%%%%%%%%%%%%%%%%%%%%%%%%%%%%
	In ML, some of commonly used primitives are overloaded.
	For example, binary addition {\tt +} can be used on several
types included {\tt int, real, word} as shown below.
\begin{tt}
\begin{quote}
1 + 1;\\
val it = 2 : int\\
1.0 + 1.0;\\
val it = 2.0 : real\\
0w1 + 0w1;\\
val it = 0wx2 : word
\end{quote}
\end{tt}
	Different from polymorphic functions, a different implementation
for {\tt +} (i.e. one of {\tt Int.+, Real.+, Word.+} in the above
example)is selected based on the context in which it is used. 
	In Standard ML this overloading is resolved at the top-level
at the end of each compilation unit.
	If there remain multiple possibilities then a predetermine one
is selected.
	For example, if you write 
\begin{tt}
\begin{quote}
fun plus x = x + x
\end{quote}
\end{tt}
in Standard ML, then {\tt Int.+}  is selected for {\tt +} and {\tt plus}
is bound to a function of type {\tt int -> int}.
	
	This strategy works fine, but this will become a big obstacle in
integrating SQL, where most of the primitives are overloaded.
	If we determine the types of all the primitives in a SQL query
at the time of is definition, then we cannot make full use of ML
polymorphism in dealing with databases.
	For this reason, \smlsharp{} introduces a mechanism to treat
overloaded primitive functions as first-class functions.
	For explain, \smlsharp{} infers the following polymorphic type
for {\tt plus}.
\begin{tt}
\begin{quote}
fun plus x = x + x;\\
val plus = fn\\
\myem  : ['a::{int, IntInf.int, real, Real32.real, word, Word8.word}. 'a -> 'a]
\end{quote}
\end{tt}
	This function can be used as a function of any type obtained by
replacing {\tt 'a} with one of {\tt int, IntInf.int, real,
Real32.real, word, Word8.word}.
	The constraint {\tt 'a::\{...\}} on type variable {\tt 'a}
indicates the set of allowable instance types.
\fi%%%%<<<<<<<<<<<<<<<<<<<<<<<<<<<<<<<<<<<<<<<<<<<<<<<<<<<<<<<<<<<<<<<<<<


\chapter{
\txt{\smlsharp{}の拡張機能:Cとの直接連携}
    {\smlsharp{} feature: direct interface to C}
}
\label{chap:tutorialCFFI}

\ifjp%>>>>>>>>>>>>>>>>>>>>>>>>>>>>>>>>>>>>>>>>>>>>>>>>>>>>>>>>>>>>>>>>>>>
	\smlsharp{}は,C言語とのシームレスな直接連携をサポートしています.
	C言語は事実上のシステム記述言語です.
	OSが提供するシステムサービスなども,C言語のインターフェイスとし
て提供されています.
	\smlsharp{}では,これら機能を,特別なライブラリなどを開発するこ
となく直接利用することができます.
	本節では,その利用方法を学びます.
\else%%%%%%%%%%%%%%%%%%%%%%%%%%%%%%%%%%%%%%%%%%%%%%%%%%%%%%%%%%%%%%%%%%%%
	C is the standard language in system programming.
	\smlsharp{} support direct interface to C.
	This feature allows the programmer to use various OS libraries
and other low-level functions implemented in C.
	This chapter explain this feature.
\fi%%%%<<<<<<<<<<<<<<<<<<<<<<<<<<<<<<<<<<<<<<<<<<<<<<<<<<<<<<<<<<<<<<<<<<

\section{\txt{C関数の使用の宣言}{Declaring and using C functions}}
\label{sec:extensionCdecl}

\ifjp%>>>>>>>>>>>>>>>>>>>>>>>>>>>>>>>>>>>>>>>>>>>>>>>>>>>>>>>>>>>>>>>>>>>
	\smlsharp{}からC言語で書かれた関数を呼び出すために必要なコードは,
その関数の宣言のみです.
	宣言は以下の文法で記述します.
\begin{program}
val $\mathit{id}$ = \_import "$\mathit{symbol}$" : $\mathit{type}$
\end{program}
	$\mathit{symbol}$は使用するC関数の名前,$\mathit{type}$は
その型の記述です.
	型については次節で説明します.
	この宣言によって,\smlsharp{}コンパイラは,C言語でコンパイルされ
たライブラリやオブジェクトファイルの中の指定された名前を持つ関数をリンクし,
\smlsharp{}の変数$\mathit{id}$として利用可能にします.
	リンク先のコードは,\smlsharp{}が動作しているOSのシステム標
準の呼び出し規約に従っているものであれば,C言語以外でコンパイルされたもの
でも構いません.
	このリンクはコンパイル時に行われます.
        従って,この{\tt \_import}宣言を含む\smlsharp{}プログラムをリンクする
ときは,$\mathit{symbol}$をエクスポートするライブラリまたはオブジェクトファイル
を\smlsharp{}プログラムと共にリンクする必要があります.
	ただし,Cの標準ライブラリの一部(Unix系OSではlibc,libm)に
含まれる関数は,それらライブラリは\smlsharp{}コンパイラが常にリンクするため,
リンクの指定無しに使用することができます.

	この宣言は,{\tt val}宣言を書けるところならどこに書くことができ
ます.
	この宣言で定義された\smlsharp{}の変数$\mathit{id}$は,
\smlsharp{}で定義された関数と同様に使用することができます.

	例えば,Cの標準ライブラリでは,以下の関数が提供されています.
\begin{program}
int puts(char *);
\end{program}
	この関数は,文字列を受け取り,その文字列と改行文字を標準出力に印
字し終了状態を返します.
	終了状態は,正常なら印字した文字数,正常に印字できなかった場合は
整数定数{\tt EOF}(具体的な値は処理系依存.Linuxでは$-1$)です.
	この型は\smlsharp{}では,{\tt string -> int}型に対応します.
	この関数を使用したい場合は,以下のように宣言します.
\begin{program}
val puts = \_import "puts" : string -> int
\end{program}
	このように,{\tt \_import}キーワードの後にC言語関数の名前と型を
宣言するだけで,通常のML関数と同様に使用することができます.
	以下は,対話型セッションでの宣言と利用の例です.
\begin{program}
\# val puts = \_import "puts" : string -> int;\\
val puts = \_ : string -> int\\
\# puts "My first call to a C library";\\
My first call to a C library\\
val it = 29 : int\\
map puts  ["I","became","fully","operational","in","April","2nd","2012."];\\
I\\
became\\
fully\\
operational\\
in\\
April\\
2nd\\
2012.\\
val it = [2, 7, 6, 12, 3, 6, 4, 5] : int list
\end{program}
	この例から理解されるとおり,インポートされたC関数も,MLプログラ
ミングの原則「式は型が正しい限り自由に組み合わせることができる」に従い,
\smlsharp{}のプログラムで利用できます.
\else%%%%%%%%%%%%%%%%%%%%%%%%%%%%%%%%%%%%%%%%%%%%%%%%%%%%%%%%%%%%%%%%%%%%
	In order to use C function, you only have to declare it in
\smlsharp{} in the following syntax.
\begin{program}
val $\mathit{id}$ = \_import "$\mathit{symbol}$" : $\mathit{type}$
\end{program}
	$\mathit{symbol}$ is the name of the C function.
	$\mathit{type}$ is its type.
	Next section explain how to write the type of a C function.

	This declaration instructs \smlsharp{} compiler to link the
named function and bind the \smlsharp{} variable $\mathit{id}$ to
that function.
	A target function linked by this declaration can be any code
as far as it is in a standard calling convention of the OS in which
\smlsharp{} runs.
	The linking to the function is performed at linking time.
	So, to produce an executable file of \smlsharp{} program containing
this {\tt \_import} declaration, it is required to specify either
a library or an object file to the command line of \smlsharp{} command to
link it with the \smlsharp{} program.
	Some of standard C libraries (including libc and libm in Unix
family OS) are linked by default.

	This declaration can appear whenever {\tt val} declaration is
allied.
	After this declaration, variable $\mathit{id}$ can be used as an ordinary
variable defined in \smlsharp{}.

	As an example, consider the standard C library function.
\begin{program}
int puts(char *);
\end{program}
	This function takes a string, appends a newline code and outputs
it to the standard output, and returns the number of characters actually
printed.
	If printing fails, then it returns the integer representing
{\tt EOF} (which is $-1$ in Linux).
	This function can be used by writing the following declarations.
\begin{program}
val puts = \_import "puts" : string -> int
\end{program}
	As seen in this example, C function can bound and used just by
writing {\tt \_import} keyword followed by the name and the type of the
desired function.
	The following is interactive session using {\tt puts}.
\begin{program}
\# val puts = \_import "puts" : string -> int;\\
val puts = \_ : string -> int\\
\# puts "My first call to a C library";\\
My first call to a C library\\
val it = 29 : int\\
map puts  ["I","became","fully","operational","in","April","6th","2012."];\\
I\\
became\\
fully\\
operational\\
in\\
April\\
2nd\\
2012.\\
val it = [2, 7, 6, 12, 3, 6, 4, 5] : int list
\end{program}
	The imported C functions can be freely used according to the
ML's programming principle -- ``expressions are freely
composed as far as they are type consistent''.
\fi%%%%<<<<<<<<<<<<<<<<<<<<<<<<<<<<<<<<<<<<<<<<<<<<<<<<<<<<<<<<<<<<<<<<<<

\section{\txt{C関数の型}{Declaring types of C functions}}
\label{sec:extensionCtype}

\ifjp%>>>>>>>>>>>>>>>>>>>>>>>>>>>>>>>>>>>>>>>>>>>>>>>>>>>>>>>>>>>>>>>>>>>
	\smlsharp{}から利用可能なC関数は,\smlsharp{}の型システムで表現
できる型を持つ関数です.
	{\tt \_import}宣言には,インポートするC関数の名前に加え,そのC関数の
型をMLライクな記法で記述します.
	{\tt val}宣言で与えられた変数にはC関数が束縛されます.
	この変数の型は,C関数の型から生成された\smlsharp{}の関数型です.
	以下,{\tt \_import}宣言に書くCの関数型の構文と,それに対応する
MLの型のサマリを示します.

	{\tt \_import}宣言には以下の形でC言語の関数の型を書きます.
\begin{program}
($\tau_1$, $\tau_2$, $\cdots$, $\tau_n$) -> $\tau$
\end{program}
	これは$n$個の$\tau_1, \tau_2, \ldots, \tau_n$型の引数を受け取り,
$\tau$型の値を返す関数を表します.
	引数が1つの場合は引数リストの括弧を省略できます.
	引数と返り値の型に何が書けるかは後述します.
	関数が引数を何も受け取らない場合,引数リストの括弧だけを書きます.
\begin{program}
() -> $\tau$
\end{program}
	関数が返り値を返さない場合(返り値の型が{\tt void}型の場合),
返り値の型として{\tt ()}を指定します.
\begin{program}
($\tau_1$, $\tau_2$, $\cdots$, $\tau_n$) -> ()
\end{program}
	C関数が可変長の引数リストを持つ場合は以下の記法を使います.
\begin{program}
($\tau_1$, $\cdots$, $\tau_m$, ...($\tau_{m+1}$, $\cdots$, $\tau_{n}$)) -> $\tau$
\end{program}
	これは,$m$個の固定の引数とそれに続く0個以上の可変個の引数を取る関数の
型です.
	\smlsharp{}は可変個の関数引数をサポートしていないため,
可変部分に与える引数のリストも指定する必要があります.
	この記法は,可変引数リストに$\tau_{m+1}, \ldots, \tau_{n}$型の引数を
与えることを表します.

	C関数の引数および返り値の型には,Cの型に対応する\smlsharp{}の型を
指定します.
	以下,C関数の引数および返り値に書ける\smlsharp{}の型を,
相互運用型と呼びます.
	相互運用型には以下の型を含みます.
\begin{itemize}
\item {\tt IntInf}を除く全ての整数型.{\tt int},{\tt word},{\tt char}など.
\item 全ての浮動小数点数型.{\tt real},{\tt Real32.real}など.
\item 要素の型が全て相互運用型であるような組型.
	{\tt int * real} など.
\item 要素の型が相互運用型であるようなベクタ型,配列型,および{\tt ref}型.
	{\tt string},{\tt Word8Array.array},{\tt int ref}など.
\end{itemize}
	これら相互運用型とCの型の対応はおおよそ以下の通りです.
\begin{itemize}
\item 整数型および浮動小数点数型の対応は以下の通りです.
\begin{center}
\begin{tabular}{|c|c|c|l|}
\hline
\smlsharp{}の相互運用型 & 対応するCの型 & 備考\\
\hline
{\tt char} & {\tt char} & Cの{\tt char}同様,符号は処理系に依存します\\
{\tt Word8.word} & {\tt unsigned char} & 1バイトは8ビットであることを仮定します\\
{\tt int} & {\tt int} & 処理系定義の自然な大きさの整数\\
{\tt word} & {\tt unsigned int} &\\
{\tt Real32.real} & {\tt float} & IEEE754形式の32ビット浮動小数点数を仮定します\\
{\tt real} & {\tt double} & IEEE754形式の64ビット浮動小数点数を仮定します\\
\hline
\end{tabular}
\end{center}
\item
	{\tt $\tau$ vector}型および{\tt $\tau$ array}型は,
$\tau$型に対応するCの型を要素型とする配列へのポインタ型に対応します.
	{\tt array}型のポインタが指す先の配列はCから書き換えることができます.
	一方,{\tt vector}型のポインタが指す配列は書き換えることはできません.
	すなわち,{\tt vector}型に対応するCのポインタ型には,
ポインタが指す先のデータの型に{\tt const}修飾子が付きます.
\item
	{\tt string}型は{\tt const char}型を要素型とする配列へのポインタ型に
対応します.
	このポインタが指す先の配列の最後には必ずヌル文字が入っています.
	従って,\smlsharp{}の{\tt string}型の値をCのヌル終端文字列への
ポインタとして使用することができます.
%	なお,Standard ML Basis Libraryでは,{\tt string}型は
%{\tt char vector}型の別名として定義されています.
\item
	{\tt $\tau$ ref}型は,$\tau$型に対応するCの型へのポインタ型に
対応します.
	ポインタが指す先には書き換え可能な長さ1の配列があります.
\item
	組型{\tt $\tau_1$ * $\cdots$ * $\tau_n$}は,
$\tau_1,\ldots,\tau_n$に対応する型のメンバーをこの順番で持つ
構造体へのポインタ型に対応します.
	このポインタが指す先の構造体はCから書き換えることはできません.
        $\tau_1,\ldots,\tau_n$が全て同じ型ならば,
その型を要素型とする配列へのポインタにも対応します.
\end{itemize}
	C関数には,\smlsharp{}の値がそのままの形で渡されます.
	C関数の呼び出しの際にデータ変換が行われることはありません.
	従って,\smlsharp{}で作った配列をC関数に渡し,C関数で変更し,
その変更を\smlsharp{}側で取り出すことができます.

	C関数全体の型は,引数リストを組型とするMLの関数型に対応付けられます.
	ただし,C関数の引数や返り値に書ける相互運用型には,以下の制約があり
ます.
\begin{itemize}
\item 配列型や組型など,Cのポインタ型に対応する相互運用型を,C関数の返り値の
型として指定することはできません.
\end{itemize}

\else%%%%%%%%%%%%%%%%%%%%%%%%%%%%%%%%%%%%%%%%%%%%%%%%%%%%%%%%%%%%%%%%%%%%

	C functions that can be imported to \smlsharp{} are those whose
types are representable in \smlsharp{}.
	The type in {\tt \_import} declaration is the type of C function
written in \smlsharp{} notation.
	The variable specified in the {\tt val} declaration is bound
to the imported C function.
	The type of this variable is an \smlsharp{} function type
that is correpsonding to the type of the C function.

	A C function type in {\tt \_import} declaration is of the form:
\begin{program}
($\tau_1$, $\tau_2$, $\cdots$, $\tau_n$) -> $\tau$
\end{program}
	This notation represents a C function that takes $n$ arguments
of type $\tau_1, \tau_2, \ldots, \tau_n$ and returns a value of $\tau$ type.
	Parentheses can be omitted if there is just one argument.
	What $\tau$ is is described below.
        If there is no argument, write as follows:
\begin{program}
() -> $\tau$
\end{program}
	If the return type is {\tt void} in C, write as follows:
\begin{program}
($\tau_1$, $\tau_2$, $\cdots$, $\tau_n$) -> ()
\end{program}
	If the C function has a variable length argument list, use the
following notation:
\begin{program}
($\tau_1$, $\cdots$, $\tau_m$, ...($\tau_{m+1}$, $\cdots$, $\tau_{n}$)) -> $\tau$
\end{program}
	This is the type of a C function that takes at least $m$ arguments
followed by variable number of arguments.
	Since \smlsharp{} does not support variable length argument lists,
user need to specify the number and types of arguments in the variable
length part of the argument list.
	This notation means that arguments of $\tau_{m+1}$, $\ldots$, $\tau_{n}$
type are passed to the function.

	In the argument list and the return type of the above C function type
notation, you can specify any \smlsharp{} type that is corresponding to a C type.
	In what follows, we refer to such \smlsharp{} types as {\em interoperable types}.
	The set of interoperable types include the following:
\begin{itemize}
\item Any integer type except {\tt IntInf}, such as {\tt int}, {\tt word}, 
and {\tt char}.
\item Any floating-point number type, such as {\tt real} and {\tt Real32.real}.
\item Any tuple type whose any field type is an interoperable type, such as
{\tt int * real}.
\item Any vector, array and ref type whose element type is an interoperable
type, such as {\tt string}, {\tt Word8Array.array}, {\tt int ref}.
\end{itemize}
	The correspondence between these interoperable types and
C types are given as follows:
\begin{itemize}
\item The following table shows the correspondence on integer types and
floating-point number types.
\begin{center}
\begin{tabular}{|c|c|c|l|}
\hline
\smlsharp{}'s interoperable type & corresponding C type & note\\
{\tt char} & {\tt char} & Signedness is not specified, similar to C.\\
{\tt Word8.word} & {\tt unsigned char} & We assume a byte is 8 bit.\\
{\tt int} & {\tt int} & Natural size of integers\\
{\tt word} & {\tt unsigned int} &\\
{\tt Real32.real} & {\tt float} & IEEE754 32-bit floating-point numbers\\
{\tt real} & {\tt double} & IEEE754 64-bit floating-point numbers
\end{tabular}
\end{center}
\item
	Any {\tt $\tau$ vector} and {\tt $\tau$ array} type is
corresponding to the type of a pointer to an array whose element type is
the C type corresponding to $\tau$ type.
	The array pointed by an {\tt array} pointer is mutable.
	In contrast, that of an {\tt vector} pointer is immutable.
	In other words,
the element type of an array pointed by an {\tt vector} pointer is qualified
by {\tt const} qualifier.
\item
	{\tt string} type is corresponding to the type of a pointer to an
array of {\tt const char}.
	The last element of the array pointed by a {\tt string} pointer is
always terminated by a null character.
	So a {\tt string} pointer can be regarded as a pointer to a
null-terminated string.
\item
	Any {\tt $\tau$ ref} type is corresponding to the type of a pointer to
the C type corresponding to $\tau$.
	A {\tt ref} pointer points to a mutable array of just one element.
\item
	Any tuple tpye {\tt $\tau_1$ * $\cdots$ * $\tau_n$} is corresponding to
the type of a pointer to a immutable structure
whose members are $\tau_1,\ldots,\tau_n$ type in this order.
	If all of $\tau_1,\ldots,\tau_n$ are same type,
this tuple type is also corresponding to the type of a pointer to an array.
\end{itemize}
	Values constructed by \smlsharp{} are passed transparently to
C functions without any modification and conversion.
	So, user can pass an array allocated in \smlsharp{} program to
a C function that modifies the given array, and obtain the modification
by the C function in \smlsharp{} program.

	The entire type of a C function is converted to an \smlsharp{}
function type whose argument type is a tuple type of the argument list types
of the C function.
	There is the following limitation in usage of interoperable types
in the C function notation:
\begin{itemize}
\item
	Interoperable types that corresponds to a C pointer type, such as
array type and tuple type, cannot be specified as a return type of
a C function.
\end{itemize}

\fi%%%%<<<<<<<<<<<<<<<<<<<<<<<<<<<<<<<<<<<<<<<<<<<<<<<<<<<<<<<<<<<<<<<<<<

\section{\txt{基本的なC関数のインポート例}
             {Basic examples of importing C functions}}

\ifjp%>>>>>>>>>>>>>>>>>>>>>>>>>>>>>>>>>>>>>>>>>>>>>>>>>>>>>>>>>>>>>>>>>>>

	いくつかのCの標準ライブラリ関数をインポートしてみましょう.
	まず,インポートするC関数のプロトタイプ宣言を探します.
	ここでは以下のC関数をインポートすることにします.
\begin{program}
double pow(double, double);\\
void srand(unsigned);\\
int rand(void);
\end{program}
	次に,引数および返り値の型に対応する\smlsharp{}の相互運用型を
用いて,\smlsharp{}プログラム中に{\tt \_import}宣言を書きます.
\begin{program}
val pow = \_import "pow" : (real, real) -> real\\
val srand = \_import "srand" : word -> ()\\
val rand = \_import "rand" : () -> int
\end{program}
	これらのC関数は以下の型の関数として\smlsharp{}にインポートされます.
\begin{program}
val pow : real * real -> real\\
val srand : word -> unit\\
val rand : unit -> int
\end{program}

	可変長引数リストの記法を用いれば,{\tt printf}(3)関数をインポート
することも可能です.
	{\tt printf}関数のプロトタイプ宣言は以下の通りです.
\begin{program}
int printf(const char *, ...);
\end{program}
	このプロトタイプ宣言に対応付けて,以下の{\tt \_import}宣言で
{\tt printf}関数をインポートします.
\begin{program}
val printfIntReal = \_import "printf" : (string, ...(int, real)) -> int
\end{program}
	この{\tt printfIntReal}関数の型は以下の通りです.
\begin{program}
val printfIntReal : string * int * real -> int
\end{program}
	ただし,このようにしてインポートした{\tt printfIntReal}関数を呼び出す
際は,その第一引数は,追加の引数として整数と浮動小数点数をこの順番で要求する
ような出力フォーマットである必要があります.

	ポインタ型と対応する相互運用型の取り扱いには注意が必要です.
	C言語ではポインタ引数を,巨大なデータ構造を参照渡しする場合と,
関数の結果をストアするためのメモリ領域を指す場合の両方に用います.
	ポインタ引数の使い方の違いによって,対応する相互運用型は異なります.
	従って,ポインタ型の引数を持つC関数をインポートする場合は,
そのプロトタイプ宣言を見るだけでなく,そのポインタ引数の意味をマニュアル
等で調べ,その使われ方に正しく対応する相互運用型を選択する必要があります.

	ポインタを引数に取るC関数をいくつかインポートしてみましょう.
	例えば,C標準ライブラリ関数{\tt modf}をインポートすることを
考えます.
	この関数のプロトタイプ宣言は以下の通りです.
\begin{program}
double modf(double, double *);
\end{program}
	この関数の場合,第二引数のポインタは,計算結果の出力先を表します.
	従って,この関数をインポートするときは,
書き換え可能な値を持つ相互運用型を第二引数の型として指定します.
\begin{program}
val modf = \_import "modf" : (real, real ref) -> real
\end{program}
	{\tt char}へのポインタ型には特に注意が必要です.
	C言語ではこの型を,ヌル終端文字列を表す型として使ったり,
最も汎用的なバッファの型として使ったりします.
	例えば,C標準ライブラリ関数{\tt sprintf}を考えます.
	そのプロトタイプ宣言は以下の通りです.
\begin{program}
int sprintf(char *, const char *, ...);
\end{program}
	2つ現れる{\tt char}へのポインタは,第一引数は出力先バッファ,
第二引数はヌル終端文字列を表します.
	このポインタの使われ方の違いに従って,これら2つのポインタ型に
異なる相互運用型を当てはめ,以下のようにインポートします.
\begin{program}
val sprintfInt = \_import "sprintf" : (char array, string, ...(int)) -> int
\end{program}

	ここまでは,C標準ライブラリ関数を例に,C関数をインポートする
方法を説明してきました.
	しかし,\smlsharp{}は,ユーザーが書いたC関数を含む,任意のC関数を
インポートすることができます.
	\smlsharp{}からユーザー定義のC関数に構造体の受け渡す簡単な例を
図\ref{fig:sampleStruct}に示します.

\begin{figure}
\begin{center}
\begin{minipage}{0.9\textwidth}
sample.cファイル:
\begin{program}
\#include <math.h>\\
double f(const struct \{double x; double y;\} *s) \{\\
\myem  return sqrt(s->x * s->x + s->y * s->y);\\
\}
\end{program}
sample.smlファイル:
\begin{program}
val f = \_import "f" : real * real -> real\\
val x = (1.1, 2.2);\\
val y = f x;\\
print ("result : " \verb|^| Real.toString y \verb|^| "\verb|\|n");\\
\end{program}
実行例:
\begin{program}
\# gcc -c sample.c\\
\# smlsharp sample.sml sample.o\\
\# a.out\\
result : 2.459675\\
\end{program}
\end{minipage}
\end{center}
\caption{構造体引数を持つユーザー定義C関数の呼び出し例}
\label{fig:sampleStruct}
\end{figure}

\else%%%%%%%%%%%%%%%%%%%%%%%%%%%%%%%%%%%%%%%%%%%%%%%%%%%%%%%%%%%%%%%%%%%%

	Let us import some standard C library functions to \smlsharp{}.
	At first, we need to look up their prototype declarations
from the manual or the header file of the functions we intend to import.
	Here, suppose we want to import the following functions.
\begin{program}
double pow(double, double);\\
void srand(unsigned);\\
int rand(void);
\end{program}
	Next, we need to write the types of the above functions by using
the interoperable types that corresponds to the argument and return types
of those functions.
\begin{program}
val pow = \_import "pow" : (real, real) -> real\\
val srand = \_import "srand" : word -> ()\\
val rand = \_import "rand" : () -> int
\end{program}
	Then these C functions are imported to \smlsharp{} as \smlsharp{}
functions of the following types.
\begin{program}
val pow : real * real -> real\\
val srand : word -> unit\\
val rand : unit -> int
\end{program}

	The {\tt printf} function can also be imported to \smlsharp{} by
using the notation of the variable length argument list.
	The prototype of {\tt printf} is given below.
\begin{program}
int printf(const char *, ...);
\end{program}
	Corresponding to this prototype, we can import {\tt printf} by
the following {\tt \_import} declaration.
\begin{program}
val printfIntReal = \_import "printf" : (string, ...(int, real)) -> int
\end{program}
	The type of {\tt printfIntReal} is as follows.
\begin{program}
val printfIntReal : string * int * real -> int
\end{program}
	We note that when calling this {\tt printIntReal} function, its
first argument must be a output format string that requires just two
arguments of an integer and a floating-point number in this order.

	Importing a function that takes pointer arguments needs special
attention.
	In C, we often use pointer arguments in two ways;
passing a large data structure by a reference to it, or
specifying a buffer to store the result of a function.
	An interoperable type is corresponding to one of the usage of
pointer arguments.
	So we need to carefully choose an interoperable type representing
a pointer argument by its usage according to the manual of the C function
we intend to import.

	Let us import some C functions that have pointer arguments.
	Suppose we want to import {\tt modf} function.
	The prototype of {\tt modf} is as follows.
\begin{program}
double modf(double, double *);
\end{program}
	According to the manual of {\tt modf}, the second argument of pointer
type must specify the destination buffer of the result of this function.
	So we specify an interoperable type of a mutable value to the type
of the second argument.
\begin{program}
val modf = \_import "modf" : (real, real ref) -> real
\end{program}
	We need to pay special attention to pointers to {\tt char}.
	In C, there are a lot of meaning a {\tt char} pointer exactly means.
	For example, a {\tt char} pointer usually means a null-terminated
string.
	In the other case, a {\tt char} pointer is used as a pointer to
most generic type of binary data buffers.
	Suppose we try to import {\tt sprintf} function.
	Its prototype is given below.
\begin{program}
int sprintf(char *, const char *, ...);
\end{program}
	The first {\tt char} pointer is a pointer to a destination buffer,
and the second one represents a null-terminated string.
	According to this difference of usage of two pointers, we give
the following type annotation to the {\tt \_import} declaration.
\begin{program}
val sprintfInt = \_import "sprintf" : (char array, string, ...(int)) -> int
\end{program}

	We can import any user-defined C function to \smlsharp{},
while we only use standard C library functions to describe how to
import C functions so far.
	Figure~\ref{fig:sampleStruct} shows an example of importing
an user-defined function to \smlsharp{} and passing a structure from
\smlsharp{} to the C function.

\begin{figure}
\begin{center}
\begin{minipage}{0.9\textwidth}
sample.c file:
\begin{program}
\#include <math.h>\\
double f(const struct \{double x; double y;\} *s) \{\\
\myem  return sqrt(s->x * s->x + s->y * s->y);\\
\}
\end{program}
sample.sml file:
\begin{program}
val f = \_import "f" : real * real -> real\\
val x = (1.1, 2.2);\\
val y = f x;\\
print ("result : " \verb|^| Real.toString y \verb|^| "\verb|\|n");\\
\end{program}
Execution:
\begin{program}
\# gcc -c sample.c\\
\# smlsharp sample.sml sample.o\\
\# a.out\\
result : 2.459675\\
\end{program}
\end{minipage}
\end{center}
\caption{Passing a tuple to user-defined C function}
\label{fig:sampleStruct}
\end{figure}

\fi%%%%<<<<<<<<<<<<<<<<<<<<<<<<<<<<<<<<<<<<<<<<<<<<<<<<<<<<<<<<<<<<<<<<<<


%\section{\txt{C関数の型}{Declaring types of C functions}}
%\label{sec:extensionCtype}
%
%%   and isInteroperableBuiltinTy dir (ty, args) =
%%       case ty of
%%         BuiltinType.INTty => true
%%       | BuiltinType.WORDty => true
%%       | BuiltinType.WORD8ty => true
%%       | BuiltinType.CHARty => true
%%       | BuiltinType.STRINGty => exportOnly dir
%%       | BuiltinType.REALty => true
%%       | BuiltinType.REAL32ty => true
%%       | BuiltinType.UNITty => false
%%       | BuiltinType.PTRty =>
%%         List.all (isInteroperableArgTy dir) args orelse
%%         (
%%           case args of
%%             [ty] => (case TU.derefTy ty of
%%                        T.CONSTRUCTty {tyCon, args=[]} =>
%%                        TypID.eq (#id tyCon, #id BE.UNITtyCon)
%%                      | _ => false)
%%           | _ => raise bug "non singleton arg in PTRty"
%%         )
%%       | BuiltinType.EXNty => false
%%       | BuiltinType.INTINFty => exportOnly dir
%%       | BuiltinType.BOXEDty => false
%%       | BuiltinType.ARRAYty =>
%%         exportOnly dir andalso List.all (isInteroperableArgTy dir) args
%%       (* FIXME : check the following *)
%%       | BuiltinType.VECTORty =>
%%         exportOnly dir andalso List.all (isInteroperableArgTy dir) args
%%       | BuiltinType.EXNTAGty => false
%
%\ifjp%>>>>>>>>>>>>>>>>>>>>>>>>>>>>>>>>>>>>>>>>>>>>>>>>>>>>>>>>>>>>>>>>>>>
%	\smlsharp{}から利用可能なC関数は,\smlsharp{}の型システムで表現
%できる型を持つ関数です.
%	{\tt \_import}宣言にこれらC関数の名前と型を記述することによって
%使用可能になります.
%	{\tt \_import}宣言に書く型は,使用するC関数の型をそれに対応する
%\smlsharp{}の記法で書いたものです.
%	\smlsharp{}コンパイラは,C関数の型に従い引数をセットし関数を呼び
%出コードを生成すると同時に,{\tt val}宣言で与えられた変数の値にC関数をセッ
%トします.
%	この変数は,C関数の型に対応して\smlsharp{}が与える型として使用で
%きます.
%	\smlsharp{}で表現可能なC言語の型のサマリを表
%\ref{fig:interoperableType}に示します.
%\begin{figure}
%\begin{center}
%\begin{tabular}{|c|c|c|l|}
%\hline
%Cの型 & \smlsharp{}での記法 & 対応する\smlsharp{}の型 & 制限(備考)
%\\ \hline
%int & int & int & 
%\\
%char & char &char &
%\\
%unsigined int & word &word &
%\\
%float & float &float &
%\\
%double & real &real &
%\\
%const char * & string &string & exportのみ
%\\
%$const \tau$ * & $\tau$ vector & $\tau$ vector & 要素が相互運用型の時
%\\
%$\tau$ * & $\tau$ array & $\tau$ array & 要素が相互運用型の時
%\\
%構造体へのポインタ & レコード & レコード & 要素が相互運用型の時
%\\
%$\tau_0$\ $id$($\tau_1$,$\ldots$,$\tau_n$) & 
%($\tau_1$,$\ldots$,$\tau_n$) -> $\tau_0$ & 
%$\tau_1$ * $\ldots$ * $\tau_n$ -> $\tau_0$ & 
%要素が相互運用型の時
%\\
%void* & unit ptr & \_ & (Cのポインタ型)
%\\\hline
%\end{tabular}\\
%\caption{Cと\smlsharp{}の型の対応}
%\label{fig:interoperableType}
%\end{center}
%\end{figure}
%
%	この表の通り,ほぼCの関数の型が,そのまま\smlsharp{}の関数の型と
%なります.
%	例えば,標準のライブラリ関数などは,以下のように宣言するだけで
%\smlsharp{}の関数として使用することができます.
%\begin{program}
%\# val sin = \_import "sin" : real -> real\\
%val sin = \_ : real -> real\\
%\# sin (Math.pi / 6.0);
%val it = 0.5 : real
%\end{program}
%	{\tt int}や{\tt real}などの原子型ばかりでなく,次節で説明すると
%おり,複数引数関数や構造体を受け取る関数なども利用できます.
%\else%%%%%%%%%%%%%%%%%%%%%%%%%%%%%%%%%%%%%%%%%%%%%%%%%%%%%%%%%%%%%%%%%%%%
%	C functions that can be imported to \smlsharp{} are those whose
%types are representable in \smlsharp{}.
%	The type in {\tt \_import} declaration is the type of C function
%written in \smlsharp{} notation.
%	\smlsharp{} compiler generates a small piece of code to set up
%parameters according to the type specification and to call the function,
%and binds the variable to the code.
%	The bound variable can then be used as a function the
%corresponding \smlsharp{} type determined by the compiler.
%
%	The table \ref{fig:interoperableType} shows 
%the C types representable in \smlsharp{} and the corresponding
%\smlsharp{} types.
%\begin{figure}
%\begin{center}
%\begin{tabular}{|c|c|c|l|}
%\hline
%C type & \smlsharp{} notation & corresponding \smlsharp{} type & restriction (note)
%\\ \hline
%int & int & int & 
%\\
%char & char &char &
%\\
%unsigined int & word &word &
%\\
%float & float &float &
%\\
%double & real &real &
%\\
%const char * & string &string & export only
%\\
%$const \tau$ * & $\tau$ vector & $\tau$ vector & 
%if element types are representable
%\\
%$\tau$ * & $\tau$ array & $\tau$ array & if element types are representable
%\\
%pointer to a struct & record type & record type & if element types are representable
%\\
%$\tau_0$\ $id$($\tau_1$,$\ldots$,$\tau_n$) & 
%($\tau_1$,$\ldots$,$\tau_n$) -> $\tau_0$ & 
%$\tau_1$ * $\ldots$ * $\tau_n$ -> $\tau_0$ & 
%if element types are representable
%\\
%void* & unit ptr & \_ & (C pointer type )
%\\\hline
%\end{tabular}\\
%\caption{Cと\smlsharp{}の型の対応}
%\label{fig:interoperableType}
%\end{center}
%\end{figure}
%
%	As show in this table, most of C types directly corresponds to
%\smlsharp{} types.
%
%	The following shows some simple examples.
%\begin{program}
%\# val sin = \_import "sin" : real -> real\\
%val sin = \_ : real -> real\\
%\# sin (Math.pi / 6.0);
%val it = 0.5 : real
%\end{program}
%	As we shall explain in the next section,
%functions taking multiple parameters and structs can also be imported.
%\fi%%%%<<<<<<<<<<<<<<<<<<<<<<<<<<<<<<<<<<<<<<<<<<<<<<<<<<<<<<<<<<<<<<<<<<
%
%\section{
%\txt{複数引数関数と構造体}
%    {Multiple argument functions and structs}
%}
%\label{sec:extensionStructFunction}
%
%\ifjp%>>>>>>>>>>>>>>>>>>>>>>>>>>>>>>>>>>>>>>>>>>>>>>>>>>>>>>>>>>>>>>>>>>>
%	複数引数関数や構造体を受け取るC関数を利用する場合,以下の点を理
%解する必要があります.
%\begin{itemize}
%\item 複数引数関数.\\
%	C関数は複数引数をとります.
%	例えば,{\tt libm}ライブラリには指数$x^y$を計算する以下の関数が
%提供されています.
%\begin{program}
%double pow(double x, double y);
%\end{program}
%	{\tt pow}は{\tt double}型の引数を2つとり,{\tt double}型の値を
%返す関数です.
%	このように複数の引数をとる関数を\smlsharp{}で使用するには,C言語
%の型宣言に従い,複数の引数の型を括弧で囲んで以下のように宣言します.
%\begin{program}
%val C\_pow = \_import "pow" : (real, real) -> real
%\end{program}
%	現在の版の\smlsharp{}では複数引数を取る関数は提供していません.
%	そこで,\smlsharp{}コンパイラは,複数引数を受け取るC関数を,タプ
%ルを受け取る関数に変換します.
%	上の宣言に対しては,以下の\smlsharp{}変数宣言が生成されます.
%\begin{program}
%val C\_pow = \_ : real * real -> real\\
%\end{program}
%	この以降,{\tt C\_pow}はタプルを受け取る関数として使用できます.
%	以下に使用例を示します.
%\begin{program}
%\# val C\_pow = \_import "pow" : (real, real) -> real;\\
%val C\_pow = \_ : real * real -> real\\
%\# val x = C\_pow (2.0,3.0);\\
%val x = 8.0 : real\\
%\# map C\_pow [(1.0,2.0), (2.0,3.0), (3.0,4.0)];\\
%val it = [1.0, 8.0, 81.0] : real list
%\end{program}
%
%\item 構造体引数.\\
%	前節の表\ref{fig:interoperableType}に示した通り,Cの構造体はMLの
%レコードに対応します.
%	例えば,
%\begin{program}
%struct S {double x; double y;};\\
% int f(struct S* s);
%\end{program}
%と宣言されたC関数は,以下のように宣言して使用することができます.
%\begin{program}
%val f =  \_import : (real * real) -> int;
%\end{program}
%	この宣言によって,以下のような変数が定義されます.
%\begin{program}
%val f = \_ : real * real -> real
%\end{program}
%	この場合は,C関数に\smlsharp{}で作られたレコードへのポインターがそ
%のまま渡されます.
%	\smlsharp{}のレコード構造は対応するC言語の構造体と同一のレイアウ
%トを実現しているため,C言語で構造体としてアクセスできます.
%	ここで,C言語はint等を含めメモリ内容を直接変更することができる言
%語であることに注意する必要があります.
%	C関数で,引数として渡された構造体のメンバーを変更した場合,その
%変更はそのまま\smlsharp{}に反映され,レコードの内容が変更されます.
%	構造体の受け渡す簡単な例を図\ref{fig:sampleStruct}に示します.
%
%\begin{figure}
%\begin{center}
%\begin{tabular}{l}
%\begin{minipage}{0.9\textwidth}
%samle.cファイル:
%\begin{program}
%struct S \{double X; double Y;\};\\
%\myem int f(struct S* s) \{\\
%\myem  s->X = 2.0;\\
%\myem  return(s->X * s->Y);\\
%\}
%\end{program}
%sample.smlファイル:
%\begin{program}
%val f = \_import "f" : real * real -> real\\
%val x = (1.1, 2.2);\\
%val y = f x;\\
%print ("The value of y:" \^ Real.toString y \^ "\verb|\|n");\\
%print ("The value of the first x:" \^ Real.toString (\#1 x) \^ "\verb|\|n");
%\end{program}
%実行例:
%\begin{program}
%\# gcc -c sample.c\\
%\# smlsharp sample.sml sample.o\\
%\# a.out\\
%The value of y:4.4\\
%The value of the first x:2.0
%\end{program}
%\end{minipage}
%\end{tabular}
%\caption{構造体引数を持つC関数の呼び出し}
%\label{fig:sampleStruct}
%\end{center}
%\end{figure}
%\end{itemize}
%\else%%%%%%%%%%%%%%%%%%%%%%%%%%%%%%%%%%%%%%%%%%%%%%%%%%%%%%%%%%%%%%%%%%%%
%	In using C functions that takes multiple arguments or struct,
%the programmer should aware the following points.
%\begin{itemize}
%\item Multiple argument functions\\
%	C functions in general take multiple arguments.
%	For example,{\tt libm} library contains the following function
%to compute power.
%\begin{program}
%double pow(double x, double y);
%\end{program}
%	{\tt pow} takes two parameters of type {\tt double} and returns
%a value of type {\tt double}.
%	In \smlsharp{}, this is declared by listing the argument types
%in parenthesis like C as follows.
%\begin{program}
%val C\_pow = \_import "pow" : (real, real) -> real
%\end{program}
%	Since the current version of \smlsharp{} does not provide
%user defined multiple argument functions, \smlsharp{} compiler converts
%this function to a function that takes a tuple of values.
%	For the above declaration, \smlsharp{} generates the following binding.
%\begin{program}
%val C\_pow = \_ : real * real -> real\\
%\end{program}
%	After this declaration, {\tt C\_pow} is used as a function to
%takes a pair as shown in the following.
%\begin{program}
%\# val C\_pow = \_import "pow" : (real, real) -> real;\\
%val C\_pow = \_ : real * real -> real\\
%\# val x = C\_pow (2.0,3.0);\\
%val x = 8.0 : real\\
%\# map C\_pow [(1.0,2.0), (2.0,3.0), (3.0,4.0)];\\
%val it = [1.0, 8.0, 81.0] : real list
%\end{program}
%
%\item Struct argument\\
%	As show in Table~\ref{fig:interoperableType}, a pointer to a
%struct in C corresponds to a record in \smlsharp{}.
%	For example,a C function
%\begin{program}
% struct S {double x; double y;};\\
% int f(struct S* s);
%\end{program}
%can be imported as:
%\begin{program}
%val f =  \_import : (real * real) -> int;
%\end{program}
%	This declaration yields the following binding.
%\begin{program}
%val f = \_ : real * real -> real
%\end{program}
%	In this case, a record generated in \smlsharp{} is directly
%passed to the C function.
%	Since a record in \smlsharp{} has the same memory layout as the
%corresponding struct in C, the passed record can be directly accessed by
%the C code.
%	However, it should be noted that C is an imperative language
%that can mutate any memory contents.
%	If C code mutates a record, then the original \smlsharp{} record
%is mutated.
%	Figure~\ref{fig:sampleStruct} show a simple example of passing a
%record to C.
%
%\begin{figure}
%\begin{center}
%\begin{tabular}{l}
%\begin{minipage}{0.9\textwidth}
%samle.c file:
%\begin{program}
%struct S \{double X; double Y;\};\\
%\myem int f(struct S* s) \{\\
%\myem  s->X = 2.0;\\
%\myem  return(s->X * s->Y);\\
%\}
%\end{program}
%sample.sml file:
%\begin{program}
%val f = \_import "f" : real * real -> real\\
%val x = (1.1, 2.2);\\
%val y = f x;\\
%print ("The value of y:" \^ Real.toString y \^ "\verb|\|n");\\
%print ("The value of the first x:" \^ Real.toString (\#1 x) \^ "\verb|\|n");
%\end{program}
%Execution:
%\begin{program}
%\# gcc -c sample.c\\
%\# smlsharp sample.sml sample.o\\
%\# a.out\\
%The value of y:4.4\\
%The value of the first x:2.0
%\end{program}
%\end{minipage}
%\end{tabular}
%\caption{Passing a tuple to C function}
%\label{fig:sampleStruct}
%\end{center}
%\end{figure}
%\end{itemize}
%\fi%%%%<<<<<<<<<<<<<<<<<<<<<<<<<<<<<<<<<<<<<<<<<<<<<<<<<<<<<<<<<<<<<<<<<<
%
%% 	特別な型であっても,それがポインタでありかつその内部構造を取り出
%% す必要がなければ,{\tt unit ptr}と宣言することにより表現可能な型として取
%% り扱うことができます.

%\section{\txt{\smlsharp{}からのCポインタの使用}{Using C pointers in \smlsharp{}}}
%
%\ifjp%>>>>>>>>>>>>>>>>>>>>>>>>>>>>>>>>>>>>>>>>>>>>>>>>>>>>>>>>>>>>>>>>>>>
%執筆中
%\else%%%%%%%%%%%%%%%%%%%%%%%%%%%%%%%%%%%%%%%%%%%%%%%%%%%%%%%%%%%%%%%%%%%%
%to be written.
%\fi%%%%<<<<<<<<<<<<<<<<<<<<<<<<<<<<<<<<<<<<<<<<<<<<<<<<<<<<<<<<<<<<<<<<<<
%
%\section{\txt{コールバック関数}{Callback functions}}
%
%\ifjp%>>>>>>>>>>>>>>>>>>>>>>>>>>>>>>>>>>>>>>>>>>>>>>>>>>>>>>>>>>>>>>>>>>>
%執筆中
%\else%%%%%%%%%%%%%%%%%%%%%%%%%%%%%%%%%%%%%%%%%%%%%%%%%%%%%%%%%%%%%%%%%%%%
%to be written.
%\fi%%%%<<<<<<<<<<<<<<<<<<<<<<<<<<<<<<<<<<<<<<<<<<<<<<<<<<<<<<<<<<<<<<<<<<

\section{\txt{動的リンクライブラリの使用}{Using dynamically linked libraries}}
\label{sec:tutorialDynamiclinc}

\ifjp%>>>>>>>>>>>>>>>>>>>>>>>>>>>>>>>>>>>>>>>>>>>>>>>>>>>>>>>>>>>>>>>>>>>
	これまで説明したC関数宣言
\begin{program}
val $\mathit{id}$ = \_import "$\mathit{symbol}$" : $\mathit{type}$
\end{program}
は$\mathit{symbol}$の名前を持つC関数を静的にリンクする指示です.
	\smlsharp{}は,この宣言を含むプログラムをオブジェクトファイルに
コンパイルする時,C関数の名前をリンカによって解決すべき外部名として
書き出します.
	C関数はリンク時に\smlsharp{}のオブジェクトファイルと共にリンクさ
れます.
	対話型モードも第\ref{chap:tutorialSeparatecompilation}章で説明す
る分割コンパイルのしくみを使って実装されているため,この宣言は対話型モー
ドでも使用できます.

	しかしながら,実行時にしか分からないライブラリの関数などを利用し
たい場合などは,この静的なリンクは使用できません.
	\smlsharp{}は,そのような場合も対応できる動的リンク機能を以下
のモジュールとして提供しています.
\begin{program}
structure DynamicLink : sig\\
\myem  type lib\\
\myem  type codeptr\\
\myem  datatype scope = LOCAL | GLOBAL\\
\myem  datatype mode = LAZY | NOW\\
\myem  val dlopen : string -> lib\\
\myem  val dlopen' : string * scope * mode -> lib\\
\myem  val dlsym : lib * string -> codeptr\\
\myem  val dlclose : lib -> unit\\
end
\end{program}
	これら関数は,Unix系OSで提供されている同名のシステムサービスと同
等の機能をもっています.
\begin{itemize}
\item  {\tt dlopen}は,共有ライブラリの名前を受け取り,共有ライブラリをオー
プンします.
\item  {\tt dlopen'}は,より詳細なオープン機能を指定できます.
	{\tt scope}と{\tt mode}は,OSのシステムサービス{\tt dlopen}に
それぞれ
{\tt RTLD\_LOCAL},
{\tt RTLD\_GLOBAL}および
{\tt RTLD\_LAZY},
{\tt RTLD\_NOW}
を指定します.
	詳しくは,OSの{\tt dlopen}関数のマニュアルを参照ください.
\item {\tt dlsym}は, {\tt dlopen}でオープンされた共有ライブラリとライ
ブラリ内の関数名を受け取り,その関数へのポインタを返します.
\item  {\tt dlclose}は,共有ライブラリをクローズします.
\end{itemize}
	{\tt dlsym}で返される関数ポインタは,以下の構文によって
\smlsharp{}の関数に変換することができます.
\begin{program}
$\mathit{exp}$ : \_import $\mathit{type}$
\end{program}
	$\mathit{exp}$は,{\tt codeptr}型を持つ\smlsharp{}の式です.
	$\mathit{type}$は,静的リンクを行う{\tt \_import}宣言で
記述する型と同一の構文によって記述されたCの型です.
	この式は,対応する\smlsharp{}の型を持つ式となります.
%	この構文は,式の型を変更する構文ですので,指定する型が正しい必要があります.

	動的リンクライブラリの利用手順は以下の通りです.
\begin{enumerate}
\item C言語などで動的リンクライブラリを作成します.
	例えばLinuxで{\tt gcc}コンパイラを用いる場合,
{\tt -shared}スイッチを指定すれば作成できます.
\item \smlsharp{}で以下のコードを実行します.
\begin{enumerate}
\item {\tt dlopen}でライブラリをオープンします.
\item {\tt dlsym}で関数ポインタを取り出します.
\item 型を指定し,C関数を\smlsharp{}の変数に束縛します.
\end{enumerate}
\end{enumerate}
	図\ref{fig:sampleDynamicLinc}に動的リンクの利用例を示します.
	共有ライブラリの作成は\smlsharp{}によるコードの実行前であればい
つでもよいので,リンク時には存在しないC関数なども,利用することができま
す.

\begin{figure}
\begin{center}
\begin{tabular}{l}
\begin{minipage}{0.9\textwidth}
samle.cファイル:
\begin{program}
int f(int s) \{\\
\myem  return(s * 2);\\
\}
\end{program}
実行例:
\begin{program}
\$ gcc -shared -o sample.so sample.c\\
\$ smlsharp\\
SML\# version 1.00 (2012-04-02 JST) for x86-linux\\
\# val lib = DynamicLink.dlopen "sample.so";\\
val lib = \_ : lib\\
\# val fptr = DynamicLink.dlsym(lib, "f");\\
val fptr = ptr : unit ptr\\
\# val f = fptr : \_import int -> int;\\
val f = \_ : int -> int\\
\# f 3;\\
val it = 6 : int
\end{program}
\end{minipage}
\end{tabular}
\caption{C関数の動的リンク}
\label{fig:sampleDynamicLinc}
\end{center}
\end{figure}
\else%%%%%%%%%%%%%%%%%%%%%%%%%%%%%%%%%%%%%%%%%%%%%%%%%%%%%%%%%%%%%%%%%%%%

	The declaration
\begin{program}
val $\mathit{id}$ = \_import "$\mathit{symbol}$" : $\mathit{type}$
\end{program}
is statically resolved to a C function having the name $\mathit{symbol}$.
	When \smlsharp{} compiles a source file containing this
declaration, the compiler generates an object file containing $\mathit{symbol}$
as an external name.
	When an executable program is build from these objects files,
the system linker links these object files with C functions.
	This is also true in interactive mode, which is implemented by
a simple iteration that performs separate compilation, linking and
loading.

	However for a library that is only available at runtime, such
static resolution  is impossible.
	To cope with those situation, \smlsharp{} provide dynamic
linking through the following module.
\begin{program}
structure DynamicLink : sig\\
\myem  type lib\\
\myem  type codeptr\\
\myem  datatype scope = LOCAL | GLOBAL\\
\myem  datatype mode = LAZY | NOW\\
\myem  val dlopen : string -> lib\\
\myem  val dlopen' : string * scope * mode -> lib\\
\myem  val dlsym : lib * string -> codeptr\\
\myem  val dlclose : lib -> unit\\
end
\end{program}
	These functions provide the system services of the same names
provide in a Unix-family OS.
\begin{itemize}
\item  {\tt dlopen} opens a shared library.
\item  {\tt dlopen'} takes one of those parameters to control {\tt dlopen}.
{\tt RTLD\_LOCAL}, {\tt RTLD\_GLOBAL} and
{\tt RTLD\_LAZY}, {\tt RTLD\_NOW}.
	For its details, consult OS manual on {\tt dlopen}.
\item {\tt dlsym} takes a library handle obtained by {\tt dlopen}
and a function name, and returns a C pointer to the function.
\item  {\tt dlclose} closes the shared library.
\end{itemize}
	The C pointer returned by {\tt dlsym} can be converted to 
\smlsharp{} function by the following expression.
\begin{program}
$\mathit{exp}$ : \_import $\mathit{type}$
\end{program}
	$\mathit{exp}$ is a \smlsharp{} expression of type {\tt codeptr}.
	$\mathit{type}$ specifies the type of C function as in {\tt \_import} declaration.

	Figure~\ref{fig:sampleDynamicLinc} shows an example.
\begin{figure}
\begin{center}
\begin{tabular}{l}
\begin{minipage}{0.9\textwidth}
samle.c file:
\begin{program}
int f(int s) \{\\
\myem  return(s * 2);\\
\}
\end{program}
Execution:
\begin{program}
\$ gcc -shared -o sample.so sample.c\\
\$ smlsharp\\
SML\# version 1.00 (2012-04-02 JST) for x86-linux\\
\# val lib = DynamicLink.dlopen "sample.so";\\
val lib = \_ : lib\\
\# val fptr = DynamicLink.dlsym(lib, "f");\\
val fptr = ptr : unit ptr\\
\# val f = fptr : \_import int -> int;\\
val f = \_ : int -> int\\
\# f 3;\\
val it = 6 : int
\end{program}
\end{minipage}
\end{tabular}
\caption{Using dynamic link library}
\label{fig:sampleDynamicLinc}
\end{center}
\end{figure}
\fi%%%%<<<<<<<<<<<<<<<<<<<<<<<<<<<<<<<<<<<<<<<<<<<<<<<<<<<<<<<<<<<<<<<<<<

\chapter{
\txt{\smlsharp{}の拡張機能:SQLのシームレスな統合}
{\smlsharp{} feature: seamless SQL integration}
}
\label{chap:tutorialDatabase}

\ifjp%>>>>>>>>>>>>>>>>>>>>>>>>>>>>>>>>>>>>>>>>>>>>>>>>>>>>>>>>>>>>>>>>>>>
	データを扱う実用的なプログラムを書くためには,データベースシステ
ムとの連携が必要です.
	現在もっとも普及しているデータベースシステムは,問い合わせ言語
SQLを用いて操作します.
	種々のプログラミング言語で,データベース操作のためのマクロや関数
が提供されていますが,データベースを使いこなす本格的なプログラムを書くた
めには,SQL言語そのものを使いデータベースシステムを呼び出すコードを各必
要があります. 
	これまでの方法は,SQL文字列を生成するコードを書き,サーバに送る
ことでしたが,\smlsharp{}では,SQLそのものを型を持つ(したがって第一級の)
式として書くことができます.
	本節では,その利用方法を学びます.
\else%%%%%%%%%%%%%%%%%%%%%%%%%%%%%%%%%%%%%%%%%%%%%%%%%%%%%%%%%%%%%%%%%%%%
	Accessing databases is essential in most of practical programs
that manipulate data.
	The most widely used database query language is SQL.
	The conventional method of accessing databases to generate SQL
command string, which is cumbersome and error prone.
	\smlsharp{} integrates SQL expressions themselves as
polymorphically typed first-class citizens.
	This chapter explain this feature.
\fi%%%%<<<<<<<<<<<<<<<<<<<<<<<<<<<<<<<<<<<<<<<<<<<<<<<<<<<<<<<<<<<<<<<<<<
	
\section{\txt{関係データベースとSQL}{Relational databases and SQL}}
\label{sec:tutorialRelationalModel}

\ifjp%>>>>>>>>>>>>>>>>>>>>>>>>>>>>>>>>>>>>>>>>>>>>>>>>>>>>>>>>>>>>>>>>>>>
	現在の本格的なデータベースは,関係データモデルを基礎として実装さ
れた関係データベースです.
	\smlsharp{}のデータベース連携を理解し使いこなすための準備として,
まず本節で,関係データベースとSQLの基本を復習しましょう.

	関係データモデルでは,データを関係の集まりとして表します.
	関係は,人の名前,年齢,給与などの複数の属性の関連を表し,通
常以下のような属性名をラベルとして持つテーブルで表現します.

\begin{center}
\begin{tabular}{|c|c|c|}
\hline
name & age & salary
\\\hline
"Joe" & 21 & 10000
\\\hline
"Sue" & 31 & 20000
\\\hline
"Bob" & 41 & 20000
\\\hline
\end{tabular}
\end{center}

	関係データベースは,これら関係の集まりを操作するシステムです.
	関係$R$は,数学的には与えられた領域$A_1,A_2,\cdots,A_n$の
積集合$A_1\times A_2 \cdots \times A_n$の部分集合です.
	関係$R$の要素$t$は,$n$個のタプル$(a_1,\ldots,a_n)$です.
	実際のデータベースでは,タプルのそれぞれの要素にラベルを付け,レコー
ドとして表現します.
	例えば上のテーブルの1行目の要素は,レコード
{\tt \{name="Joe", age=21, salary=1000\}}に対応します.
	これら関係に対して,和集合演算,射影($n$項関係を$m (\le n)$項関
係に射影),選択(集合の特定の要素の選び出し),デカルト積 $R\times S$な
どの演算が定義されています.
	これら演算系を関係代数と呼びます.
	ここで留意すべき点は,関係モデルはレコードの集合に対する代数的な
言語である点です.
	代数的な言語は,関数定義を含まない関数型言語です.

	関係データベースでは,この関係代数をSQLと呼ばれる集合操作言語で
表現します.
	SQLの中心は,以下のSELECT式です.
\begin{program}
SELECT $t^1.l_1$ as $l_1'$,$\ldots$, $t^m.l_m$ as $l_m$\\
FROM $R_1$ as $t_1$, $\ldots$, $R_n$ as $t_n$\\
WHERE $P(t_1,\ldots, t_n)$
\end{program}
	ここでは,以下の約束に従ってメタ変数を使用しています.
\begin{itemize}
\item $R$:関係を表す式.
\item $t$:関係の中の一つのレコード要素を代表する変数
\item $l$:属性名
\item $t.l$:$t$の中の属性$l$の値を表す式.
\end{itemize}
	SELECT式の意味は以下の操作と理解できます.
\begin{enumerate}
\item {\tt FROM}節で列挙された関係式$R_i$を評価し,デカルト積
$
R_1 \times \cdots \times R_n
$
を作る.
\item デカルト積の任意の要素のタプルを$(t_1,\ldots,t_n)$とする.
\item デカルト積から,{\tt WHERE}節で指定された述語$P(t_1,\ldots,t_n)$を
満たす要素のみを選び出す.
\item
	上記の結果得られた集合の各要素$(t_1,\ldots,t_n)$に対して,
レコード
{\tt \{$l_1'$=$t^1.l_1$, $\ldots$, $l_m'$=$t^m.l_m$\}}
を構築する.
\item それらレコードをすべて集めた集合を,この式全体の結果とする.
\end{enumerate}
	例えば,上の例のテーブルを{\tt Persons}とし,以下のSQL式を考えて
みましょう.
\begin{program}
SELECT P.name as name, P.age as age\\
FROM Persons as P\\
WHERE P.salary > 10000
\end{program}
	この問い合わせ式は以下のように評価されます.
\begin{itemize}
\item 関係{\tt Persons}のみのデカルト積は関係{\tt Persons}それ自身である.
\item 関係{\tt Persons}の任意のタプルを{\tt P}とする.
\item {\tt P.Salary > 10000}の条件を満たすタプルを選び出し,以下の集合を得
る.

\begin{center}
\begin{tabular}{|c|c|c|}
\hline
name & age & salary
\\\hline
"Sue" & 31 & 20000
\\\hline
"Bob" & 41 & 20000
\\\hline
\end{tabular}
\end{center}
\item この集合の各要素{\tt P}に対して{\tt \{name=P.name, age=P.age\}}を
計算し,以下の集合を得る.

\begin{center}
\begin{tabular}{|c|c|}
\hline
name & age
\\\hline
"Sue" & 31
\\\hline
"Bob" & 41
\\\hline
\end{tabular}
\end{center}
このテーブルは,
{\tt \{\{name="Sue", age=31\}, \{name="Bob", age=31\}\}}
のようなレコードのリストの表現と理解できる.
\end{itemize}
\else%%%%%%%%%%%%%%%%%%%%%%%%%%%%%%%%%%%%%%%%%%%%%%%%%%%%%%%%%%%%%%%%%%%%
	Most of practical database systems are relational databases.
	To understand \smlsharp{} database integration, this section
review the basics notions of relational databases and SQL.

	In the relational model, data are represented by a set of
relations.
	A relation is a set of tuples, each of which represents
association of attribute values such as name, age, and salary.
	Such a relation is displayed as a table of the following form.

\begin{center}
\begin{tabular}{|c|c|c|}
\hline
name & age & salary
\\\hline
"Joe" & 21 & 10000
\\\hline
"Sue" & 31 & 20000
\\\hline
"Bob" & 41 & 20000
\\\hline
\end{tabular}
\end{center}

	A relational database is system to manipulate a collection of
such tables.
	A relation $R$ on the sets $A_1,A_2,\cdots,A_n$ of attribute
values is mathematically a subset of the Cartesian product
$A_1\times A_2 \cdots \times A_n$.
	Each element $t$ in $R$  is an $n$ element tuple $(a_1,\ldots,a_n)$.
	In an actual database system, each component of a tuple has
attribute name, and a tuple is represented as a labeled record.
	For example, the first line of the example table above is
regarded as a record  {\tt \{name="Joe", age=21, salary=1000\}}.
	On these relations, a family of operations are defined,
including union, projection, selection, and Cartesian product.
	A set of tables associated with a set of these operations is
called the relational algebra.
	One important thing to note on this model is that, as its name
indicates, the relational model is an algebra and that it is
manipulated by an algebraic language.
	An algebraic language is a functional language that does not
have function expression.

	In relational databases, the relational algebra is represented
by the language called SQL, which is language of set-value expressions.
	The central construct of SQL is the following SELECT expression.
\begin{program}
SELECT $t^1.l_1$ as $l_1'$,$\ldots$, $t^m.l_m$ as $l_m$\\
FROM $R_1$ as $t_1$, $\ldots$, $R_n$ as $t_n$\\
WHERE $P(t_1,\ldots, t_n)$
\end{program}
	Here we used the following meta variables.
\begin{itemize}
\item $R$: relation variables
\item $t$: tuple variables
\item $l$: labels, or attribute names
\item $t.l$: the $l$ attribute of tuple $l$
\end{itemize}
	The operational meaning of a SELECT expression can be understood
as follows.
\begin{enumerate}
\item Evaluate each $R_i$ in {\tt FROM} clause, and generate their Cartesian product
$
R_1 \times \cdots \times R_n
$
\item Let $(t_1,\ldots,t_n)$ be any representative tuple in the product.
\item Select the tuples that satisfies the predicate $P(t_1,\ldots,t_n)$
specified in {\tt WHERE} clause from the product.
\item
	For each element $(t_1,\ldots,t_n)$ in the select set, construct
a record  {\tt \{$l_1'$=$t^1.l_1$, $\ldots$, $l_m'$=$t^m.l_m$\}}.

\item Collect all these records.
\end{enumerate}
	For example, let the above example table be named as {\tt
Persons} and consider the following SQL.
\begin{program}
SELECT P.name as name, P.age as age\\
FROM Persons as P\\
WHERE P.salary > 10000
\end{program}
	This expression is evaluated as follows.
\begin{itemize}
\item The Cartesian product of the soul relation {\tt Persons} is {\tt
Persons} itself.
\item Let {\tt P} be any tuple in {\tt Persons}.
\item Select from {\tt Person} all the tuples {\tt P} such that {\tt
P.Salary > 10000}.
	We obtain the following set.

\begin{center}
\begin{tabular}{|c|c|c|}
\hline
name & age & salary
\\\hline
"Sue" & 31 & 20000
\\\hline
"Bob" & 41 & 20000
\\\hline
\end{tabular}
\end{center}

\item For each tuple {\tt P} in this set, compute the new tuple {\tt
\{name=P.name, age=P.age\}} to obtain the following set.

\begin{center}
\begin{tabular}{|c|c|}
\hline
name & age
\\\hline
"Sue" & 31
\\\hline
"Bob" & 41
\\\hline
\end{tabular}
\end{center}
The is the result of the expression.
	
This result represent the set (list) of records:
{\tt \{\{name="Sue", age=31\}, \{name="Bob", age=31\}\}}.
\end{itemize}
\fi%%%%<<<<<<<<<<<<<<<<<<<<<<<<<<<<<<<<<<<<<<<<<<<<<<<<<<<<<<<<<<<<<<<<<<

\section{\txt{\smlsharp{}へのSQL式の導入}{Integrating SQL in \smlsharp{}}}
\label{sec:tutorialIntoroducingSQL}

\ifjp%>>>>>>>>>>>>>>>>>>>>>>>>>>>>>>>>>>>>>>>>>>>>>>>>>>>>>>>>>>>>>>>>>>>
	SQLの{\tt SELECT}文は{\tt FROM}節に記述されたテーブルの集合から
一つのテーブルを作成する式です.
	SQL言語は,ある特定のデータベースへの接続の下でSQL式を評価する機
能を提供しますが,これをデータベース接続を受け取りテーブルを返す関数式に
一般化すれば,関数型言語の型システムに統合することができます.
	テーブルはレコードの構造をしているため,多相レコードの型付けがほ
ぼそのまま使用できます.
	しかし我々\smlsharp{}チームの研究によって,このデータベースを受
け取る関数の型付けには,多相レコード以外にさらに型理論的な機構が必要であ
ることが示されています\cite{ohori11}.
	そこで,データベース接続を受け取る関数には特別の文法を用意します.
	\smlsharp{}では,前節のSQL式の例は,以下のような式で表現されます.
\begin{program}
\_sql db => select \#P.name as name, \#P.age as age\\
\myem\myem\myem from \#db.Persons as P\\
\myem\myem\myem where SQL.>(\#P.salary, 10000)
\end{program}
	{\tt \_sql db => ...}は,データベース接続を変数{\tt db}として受
け取る問い合わせ関数であることを示しています.
	{\tt SQL.>}はデータベース問い合わせを実現するライブラリモジュー
ル{\tt SQL}の中に定義された{\tt SQL}の値のための大小比較プリミティブです.
	{\tt \#P.Name}はタプル{\tt P}の{\tt Name}フィールドの取り出し演算,
{\tt \#db.Persons}はデータベース{\tt db}からの{\tt Persons}テーブル取り
出し演算です.
	これらは,\smlsharp{}式では{\tt \#Name P}{\tt \#Persons db}と書か
れるレコードからのフィールド取り出し式に相当します.
	{\tt \_sql $x$ => $expr$}式では,SQLの文法に類似の構文を採用して
います.
	この式に対して以下の型が推論されます.
\begin{program}
val it = \_
\\\myem\ \ \   : ['a\#\{Persons: 'b\},
\\\myem\myem     'b\#\{age: 'f, name: 'd, Salary: int\},
\\\myem\myem     'c,
\\\myem\myem     'd::\{int, word, char, string, real, 'e option\},
\\\myem\myem     'e::\{int, word, char, bool, string, real\},
\\\myem\myem     'f::\{int, word, char, string, real, 'g option\},
\\\myem\myem     'g::\{int, word, char, bool, string, real\}.
\\\myem\myem\myem       ('a, 'c) SMLSharp\_SQL\_Prim.db
\\\myem\myem\myem\myem  -> \{age: 'd, name: 'f\} SMLSharp\_SQL\_Prim.query]
\end{program}
	この型は,{\tt 'a}の構造を持つデータベース接続の型{\tt  ('a, 'c)
db}から{\tt \{age: 'd, name: 'f\} query}型への関数型です.
	{\tt 'a}は,この問い合わせに必要なデータベース構造を表すレコード
多相型です.
	型変数{\tt 'c}は,データベース接続の一貫性を保証するために導入さ
れたものです.
	このようにSQL式は多相型を持つため,MLプログラミングの原理「式は
型が正しい限り自由に組み合わせることができる」に従って,第一級のデータ
として\smlsharp{}の他の機能とともに自由にプログラムすることができます.
\else%%%%%%%%%%%%%%%%%%%%%%%%%%%%%%%%%%%%%%%%%%%%%%%%%%%%%%%%%%%%%%%%%%%%
	We observe that SQL {\tt SELECT} command is an expression that
construct a table from a set of tables specified in {\tt FROM} clause.
	In practice, SQL commands are usually evaluated against a
particular database connection.
	Since database connection is conceptually a set of tables, 
if we generalize SQL as a language that takes a database connection and
returns a relation, an SQL command can be regarded as a function that
takes a set of tables and returns a table.
	As we have already observe that SQL {\tt SELECT} is an algebraic
expression.
	Then SQL is a functional language on tables.

	Since tables are set of records, table value functions should be
typable using record polymorphism.
	According to our analysis, there is one subtle point in typing
beyond record polymorphism \cite{ohori11}.
	For this reason, \smlsharp{} introduced the following special
syntax for SQL query functions.
\begin{program}
\_sql db => select \#P.name as name, \#P.age as age\\
\myem\myem\myem from \#db.Persons as P\\
\myem\myem\myem where SQL.>(\#P.salary, 10000)
\end{program}
	{\tt \_sql db => ...} represent a function that takes a database
connection through parameter {\tt db}.
	{\tt SQL.>} is SQL primitive for numerical comparison operation
defined in the {\tt SQL} module that implement various SQL specific
primitives.
	{\tt \#P.Name} selects {\tt Name} field from {\tt P}.
{\tt \#db.Persons} selects {\tt Persons} table from {\tt db}.
	These correspond to {\tt \#Name P} and {\tt \#Persons db} in
\smlsharp{} expression.
	In {\tt \_sql $x$ => $expr$} expression, we choose these syntax
that are closer to SQL SELECT commands.
	
	For this expression, \smlsharp{} infers the following
polymorphic type.
\begin{program}
val it = \_
\\\myem\ \ \   : ['a\#\{Persons: 'b\},
\\\myem\myem     'b\#\{age: 'f, name: 'd, Salary: int\},
\\\myem\myem     'c,
\\\myem\myem     'd::\{int, word, char, string, real, 'e option\},
\\\myem\myem     'e::\{int, word, char, bool, string, real\},
\\\myem\myem     'f::\{int, word, char, string, real, 'g option\},
\\\myem\myem     'g::\{int, word, char, bool, string, real\}.
\\\myem\myem\myem       ('a, 'c) SMLSharp\_SQL\_Prim.db
\\\myem\myem\myem\myem  -> \{age: 'd, name: 'f\} SMLSharp\_SQL\_Prim.query]
\end{program}
	This type indicates that this query is a function from a database
connection of type {\tt  ('a, 'c) db} to a table of tuple {\tt \{age: 'd,
name: 'f\}}.
	{\tt 'a} represents the structure of the input database.
	Type variable {\tt 'c} is introduce to enforce database
connection consistency.
	
	This is indeed a most general polymorphic type of the above
query,
	By the fact that \smlsharp{} can infer a most general
polymorphic type, we are guarantees that SQL can be used in \smlsharp{}
based on ML programming principle: ``expressions are freely
composed as far as they are type consistent'', i.e.\ 
SQL expressions can be freely combined with any other language construct
of \smlsharp{} as far as they are type consistent.
	This will open up flexible and type safe programming using
databases.

\fi%%%%<<<<<<<<<<<<<<<<<<<<<<<<<<<<<<<<<<<<<<<<<<<<<<<<<<<<<<<<<<<<<<<<<<

\section{\txt{問い合わせの実行}{Query execution}}
\label{sec:tutorialExecutingSQL}

\ifjp%>>>>>>>>>>>>>>>>>>>>>>>>>>>>>>>>>>>>>>>>>>>>>>>>>>>>>>>>>>>>>>>>>>>
	{\tt \_sql $x$ => $exp$}式で定義された問い合わせ関数にデータベー
ス接続を適用すれば,データベースをアクセスできます.
	そのために,以下の構文が用意されています.
\begin{itemize}
\item データベースサーバー式.
\begin{program}
\_sqlserver $serverLocation$ : $\tau$
\end{program}
	この式は,データベースサーバーを指定します.
	$serverLocation$はデータベースサーバーの場所と名前です.
	その具体的な内容は,使用するデータベースシステムのデータベース指
定構文に従います.
	$\tau$はデータベースの構造です.
	テーブルの名前とその構造をレコード型の文法で指定します.
	この式を評価すると,{\tt $\tau$ SQL.server}型をもつデータベースサー
バー定義が得られます.
\item データベースへの接続プリミティブ関数.
\begin{program}
SQL.connect : ['a. 'a SQL.server -> 'a SQL.conn]
\end{program}
	\smlsharp{}コンパイラは,{\tt $\tau$ SQL.server}型のデータベース定義
を受け取り,その中に指定されたデータベースに接続しその構造をチェックし
$\tau$と一致していることを確認した後,{\tt $\tau$ SQL.conn}の型をもつ$\tau$
型のデータベースへの接続を返します.
	
\item 問い合わせ実行のための構文.
\begin{program}
\_sqleval $query$ $connection$
\end{program}
	データベース問い合わせ関数$query$をデータベース接続$connection$
に対して評価し,{\tt $\tau$ SQL.rel}型をもつ結果を返します.

\item 問い合わせ結果のリストへの変換.
\begin{program}
SQL.fetchAll : ['a. 'a SQL.rel -> 'a list]\\
SQL.fetch : ['a. 'a SQL.rel -> ('a * 'a SQL.rel) option]
\end{program}
	{\tt SQL.fetchAll}は問い合わせによって得られた関係をリストに変換
します.
	{\tt SQL.fetch}は問い合わせによって得られた関係から先頭のレコー
ドとそれ以外の関係を返します.
\item 問い合わせの後処理.
\begin{program}
SQL.closeRel : ['a. 'a SQL.rel -> unit]\\
SQL.closeCon : ['a. 'a SQL.conn -> unit]
\end{program}
	{\tt SQL.closeRel}は問い合わせ処理の終了を,{\tt SQL.closeConn}
はデータベース接続の終了をそれぞれ
データベースサーバに通知します.
\end{itemize}
\else%%%%%%%%%%%%%%%%%%%%%%%%%%%%%%%%%%%%%%%%%%%%%%%%%%%%%%%%%%%%%%%%%%%%
	Database can be accessed by applying query function defined by
{\tt \_sql $x$ => $exp$} expression to a database connection.
	For this purpose, \smlsharp{} provides the following constructs.
\begin{itemize}
\item Database server expressions
\begin{program}
\_sqlserver $serverLocation$ : $\tau$
\end{program}
	This expression locates a database server.
	$serverLocation$ is the location information of a database
server.
	Its concrete syntax is determined by the database system to be
connected.
	$\tau$ is the type of the database to be connected.
	It describes the table names and their types using the syntax
of record types.
	Evaluating this expression always succeeds and  yields a
database server object of type {\tt $\tau$ SQL.server}, which contains
the database server information.
\item Database connection primitive.
\begin{program}
SQL.connect : ['a. 'a SQL.server -> 'a SQL.conn]
\end{program}
	This primitive takes a database server value, extract the
database location stored in the value, attempts to connect to the
database, and if successful it checks that the connected database
indeed has the structure described by $\tau$, and then return 
a database connection object of type {\tt $\tau$ SQL.conn}.
	
\item Syntax for executing database query.
\begin{program}
\_sqleval $query$ $connection$
\end{program}
	This primitive sends the $query$ to the connection $connection$
for evaluation and obtains the result of type {\tt $\tau$ SQL.rel}.

\item Conversion of the query result.
\begin{program}
SQL.fetchAll : ['a. 'a SQL.rel -> 'a list]\\
SQL.fetch : ['a. 'a SQL.rel -> ('a * 'a SQL.rel) option]
\end{program}
	{\tt SQL.fetchAll} converts the relation to a list of records.
	{\tt SQL.fetch} fetches the first tuple in the relation and
convert it to a record.
\item Post processing.
\begin{program}
SQL.closeRel : ['a. 'a SQL.rel -> unit]\\
SQL.closeCon : ['a. 'a SQL.conn -> unit]
\end{program}
	{\tt SQL.closeRel} terminates the query and {\tt SQL.closeConn}
close a database connection.
\end{itemize}
\fi%%%%<<<<<<<<<<<<<<<<<<<<<<<<<<<<<<<<<<<<<<<<<<<<<<<<<<<<<<<<<<<<<<<<<<
		
\section{\txt{データベース問い合わせ実行例}{Query examples}}
\label{sec:tutorialSQLExample}

\ifjp%>>>>>>>>>>>>>>>>>>>>>>>>>>>>>>>>>>>>>>>>>>>>>>>>>>>>>>>>>>>>>>>>>>>
	以上の構文を使いデータベースの問い合わせを行なってみましょう.
	\smlsharp{}の通常のインストールでは,データベースサーバは
PostgreSQLと接続するように設定されています.
	データベースをアクセスするためには,PostgreSQLサーバをインストー
ルし起動しておく必要があります. 
	
	まず以下の手順で,PostgreSQLサーバでデータベースを構築しましょう.
	以下に簡単な手順を示します.
	詳しくは{\tt PostgreSQL}ドキュメントを参照してください.
\begin{enumerate}
%\item {\tt postgres}アカウントにログインします.
\item {\tt PostgreSQL}サーバを起動します.
	例えば{\tt pg\_ctl start -D /usr/local/pgsql/data}
ようにすると起動できるはずです.
\item コマンドラインで{\tt createuser myAccount}を実行し{\tt PostgreSQL}
のユーザのロールを作成する.
	{\tt myAccount}は,使用するユーザの名前です.
\item 
コマンドラインで{\tt createdb mydb}を実行しデータベースを作成する.
\begin{quote}

\end{quote}
\item SQL言語インタープリタを起動し,データベースの中にテーブルを作成す
る.
	例えば,第\ref{sec:tutorialRelationalModel}節の例のデータベースは,以下
のようにすれば作成できます.
\begin{program}
\$ psql\\
mydb\# CREATE TABLE Persons (\\
\myem\myem name text not null, age int not null, salary int not null
\myem );\\
mydb\# INSERT INTO Persons VALUES ('Joe', 21, 10000);\\
mydb\# INSERT INTO Persons VALUES ('Sue', 31, 20000);\\
mydb\# INSERT INTO Persons VALUES ('Bob', 41, 30000);\\
\end{program}
\end{enumerate}
	自分のアカウント(ここでは{\tt myAccount})に戻り,データベース
にアクセスできるか確認してみましょう.
	以下のような結果が得られれば,成功です.
\begin{program}
\$ psql mydb;\\
\\
mydb=\# SELECT * FROM Persons;\\
 name | age | salary \\
------+-----+--------\\
 Joe  |  21 |  10000\\
 Sue  |  31 |  20000\\
 Bob  |  41 |  20000\\
(3 行)
\end{program}

	ではいよいよ,このデータベースを\smlsharp{}からアクセスしてみま
しょう.
	第\ref{sec:tutorialIntoroducingSQL}節で定義した問い合わせ関数を{\tt myQuery}とします.
	対話型セッションで以下のような結果が得られるはずです.
\begin{program}
\$ smlsharp;\\
\# val myServer = \_sqlserver (dbname="mydb") : \{Persons:\{name:string, age:int, salary :int\};\\
val myServer = \_ : \{Persons: \{age: int, name: string, salary: int\}\} server\\
\# val conn = SQL.connet myServer;\\
val conn = \_ : \{Persons: \{age: int, name: string, salary: int\}\} conn\\
\# val rel = \_sqleval myQuery conn;\\
val rel = \{Persons: \{age: int, name: string, salary: int\}\} rel\\
\# SQL.fetchAll rel;\\
val it = \{\{age=32, name="Sue"\}, \{age=41, name="Bob"\}\} : \{age:int, name: string\} list
\end{program}
\else%%%%%%%%%%%%%%%%%%%%%%%%%%%%%%%%%%%%%%%%%%%%%%%%%%%%%%%%%%%%%%%%%%%%
	Let us now connect and use an actual database from \smlsharp{}.
	\smlsharp{} version \version{} support PostgreSQL.
	To use a database, you need to install and start PostgreSQL server.
	
	Here we show a standard step in Linux system to set up PostgreSQL server.
	For more details, consult {\tt PostgreSQL} document.
\begin{enumerate}
\item Login as postgres.
\item Start {\tt PostgreSQL} server.
	The command {\tt pg\_ctl start -D /usr/local/pgsql/data} will do this.
\item Create a {\tt PostgreSQL} user role by executing command
{\tt createuser myAccount}, where {\tt myAccount} is your user account.
\item Return to your account, and execute command {\tt createdb mydb} to
create a database called mydb. 
\item Use the SQL interpreter to create a table.
	For example, the table in
Section~\ref{sec:tutorialRelationalModel} can be created as follows. 
\begin{program}
\$ psql\\
mydb\# CREATE TABLE Persons (\\
\myem\myem name text not null, age int not null, salary int not null
\myem );\\
mydb\# INSERT INTO Persons VALUES ('Joe', 21, 10000);\\
mydb\# INSERT INTO Persons VALUES ('Sue', 31, 20000);\\
mydb\# INSERT INTO Persons VALUES ('Bob', 41, 30000);\\
\end{program}
\item Check that the database can be accessed.
	You should get the following output.
\begin{program}
\$ psql mydb;\\
\\
mydb=\# SELECT * FROM Persons;\\
 name | age | salary \\
------+-----+--------\\
 Joe  |  21 |  10000\\
 Sue  |  31 |  20000\\
 Bob  |  41 |  20000\\
(3 行)
\end{program}
\end{enumerate}

	Now let's access this database from \smlsharp{}.
	Let the query function defined in Section~\ref{sec:tutorialIntoroducingSQL}
{\tt myQuery}.
	In an interactive session, you should get the following.
\begin{program}
\$ smlsharp;\\
\# val myServer = \_sqlserver (dbname="mydb") : \{Persons:\{name:string, age:int, salary :int\};\\
val myServer = \_ : \{Persons: \{age: int, name: string, salary: int\}\} server\\
\# val conn = SQL.connet myServer;\\
val conn = \_ : \{Persons: \{age: int, name: string, salary: int\}\} conn\\
\# val rel = \_sqleval myQuery conn;\\
val rel = \{Persons: \{age: int, name: string, salary: int\}\} rel\\
\# SQL.fetchAll rel;\\
val it = \{\{age=32, name="Sue"\}, \{age=41, name="Bob"\}\} : \{age:int, name: string\} list
\end{program}
\fi%%%%<<<<<<<<<<<<<<<<<<<<<<<<<<<<<<<<<<<<<<<<<<<<<<<<<<<<<<<<<<<<<<<<<<

\section{\txt{その他のSQL文}{Other SQL statements}}
\label{sec:tutorialOtherSQLElement}

\ifjp%>>>>>>>>>>>>>>>>>>>>>>>>>>>>>>>>>>>>>>>>>>>>>>>>>>>>>>>>>>>>>>>>>>>
	SQL言語は,データベースからデータを取り出す{\tt select}文以外に,
様々な機能が提供されています.
	\smlsharp{}の\version{}版では,以下の機能をサポートしています.
\begin{itemize}
\item タプルの追加と削除
\item テーブルの更新
\item トランザクションの実行
\end{itemize}
	図\ref{fig:sqlSyntax}に現在のSQLの構文の概要を示します.
\begin{figure}
\begin{center}
\[
\begin{array}{rcll}
  e &::=& ... & \mbox{(\smlsharp{} expressions)}
\\  &|& \code{\_sqlserver $serverDef$~:~$\tau$} 
	& \mbox{(server definition)}
\\  &|& \code{\_sql $x$ => $sql$} 
	& \mbox{(SQL expressions)}
\\  &|& \code{\_sqleval $e$}
	& \mbox{(query evaluation)}
\\  &|& \code{\_sqlexec $e$}
	& \mbox{(command execution)}
\\  &|& \code{\#$x$.$l$}
	& \mbox{(selection)}
\\[1ex]
serverDef &::=& (key = value,\ldots,key = value)
\\[1ex]
  sql &::=& select
\vbar insert
\vbar update
\vbar \code{begin}
\vbar \code{commit}
\vbar \code{rollback}
\\[1ex]
select &::=& 
     \code{select $e$ $[$ as $l$ $]$ $\ldots$ $e$ $[$ as $l$ $]$}
%           $[$ into $x$ $]$}
\\&& \code{$[$ from $e$ as $x$ $\ldots$ $e$ as $x$ $]$}
\\&& \code{$[$ where $e$ $]$}
%\\&& \code{$[$ order by $e$ $[$$\{$asc$|$desc$\}$$]$ $\cdots$
%                    $e$ $[$$\{$asc$|$desc$\}$$]$ $]$}
\\&& \cdots\ \mbox{(other clauses)}
\\[1ex]
insert &::=& 
\code{insert into \#$x$.$l$ ($l$, $\ldots$, $l$)}
\\&& \code{values ( $\{e|$default$\}$, $\ldots$, $\{e|$default$\}$)}
\\[1ex]
update &::=& 
   \code{update \#$x$.$l$ $[$ as $x$ $]$}
\\&& \code{set ($l$, $\ldots$, $l$) = ($e$, $\ldots$, $e$)}
\\&& \code{$[$ from $e$ as $x$ $\ldots$ $e$ as $x$ $]$}
\\&& \code{$[$ where $e$ $]$}
\\[1ex]
delete &::=& 
   \code{delete from \#$x$.$l$ $[$ as $x$ $]$}
\\&& \code{$[$ where $e$ $]$}
\end{array}
\]
\ \\
\caption{\smlsharp{}第\version{}版のSQL構文概要}
\end{center}
\label{fig:sqlSyntax}
\end{figure}

	これ以外のSQL言語の機能の追加も技術的には問題はありません.
	次版以降でより完全なSQL言語の統合をして行く予定です.
\else%%%%%%%%%%%%%%%%%%%%%%%%%%%%%%%%%%%%%%%%%%%%%%%%%%%%%%%%%%%%%%%%%%%%
	SQL contain many other commands SQL.
	\smlsharp{} version \version{} support the following:
\begin{itemize}
\item insert and delete,
\item table update, and
\item transaction.
\end{itemize}
	Figure~\ref{fig:sqlSyntax} shows the currently supported SQL expressions in
\smlsharp{}.
\begin{figure}
\begin{center}
\[
\begin{array}{rcll}
  e &::=& ... & \mbox{(\smlsharp{} expressions)}
\\  &|& \code{\_sqlserver $serverDef$~:~$\tau$} 
	& \mbox{(server definition)}
\\  &|& \code{\_sql $x$ => $sql$} 
	& \mbox{(SQL expressions)}
\\  &|& \code{\_sqleval $e$}
	& \mbox{(query evaluation)}
\\  &|& \code{\_sqlexec $e$}
	& \mbox{(command execution)}
\\  &|& \code{\#$x$.$l$}
	& \mbox{(selection)}
\\[1ex]
serverDef &::=& (key = value,\ldots,key = value)
\\[1ex]
  sql &::=& select
\vbar insert
\vbar update
\vbar \code{begin}
\vbar \code{commit}
\vbar \code{rollback}
\\[1ex]
select &::=& 
     \code{select $e$ $[$ as $l$ $]$ $\ldots$ $e$ $[$ as $l$ $]$}
%           $[$ into $x$ $]$}
\\&& \code{$[$ from $e$ as $x$ $\ldots$ $e$ as $x$ $]$}
\\&& \code{$[$ where $e$ $]$}
%\\&& \code{$[$ order by $e$ $[$$\{$asc$|$desc$\}$$]$ $\cdots$
%                    $e$ $[$$\{$asc$|$desc$\}$$]$ $]$}
\\&& \cdots\ \mbox{(other clauses)}
\\[1ex]
insert &::=& 
\code{insert into \#$x$.$l$ ($l$, $\ldots$, $l$)}
\\&& \code{values ( $\{e|$default$\}$, $\ldots$, $\{e|$default$\}$)}
\\[1ex]
update &::=& 
   \code{update \#$x$.$l$ $[$ as $x$ $]$}
\\&& \code{set ($l$, $\ldots$, $l$) = ($e$, $\ldots$, $e$)}
\\&& \code{$[$ from $e$ as $x$ $\ldots$ $e$ as $x$ $]$}
\\&& \code{$[$ where $e$ $]$}
\\[1ex]
delete &::=& 
   \code{delete from \#$x$.$l$ $[$ as $x$ $]$}
\\&& \code{$[$ where $e$ $]$}
\end{array}
\]
\ \\
\caption{SQL expressions in \smlsharp{} version \version{}}
\end{center}
\label{fig:sqlSyntax}
\end{figure}

	There is no technical problems in adding other commands.
\fi%%%%<<<<<<<<<<<<<<<<<<<<<<<<<<<<<<<<<<<<<<<<<<<<<<<<<<<<<<<<<<<<<<<<<<

\chapter{
\txt{\smlsharp{}分割コンパイルシステム}
 {Separate compilation in \smlsharp{}}
}
\label{chap:tutorialSeparatecompilation}

\ifjp%>>>>>>>>>>>>>>>>>>>>>>>>>>>>>>>>>>>>>>>>>>>>>>>>>>>>>>>>>>>>>>>>>>>
	実用言語としての\smlsharp{}の大きな特徴は,完全な分割コンパイル
のサポートです.
	以下の手順で大規模なプログラムを開発していくことができます.
\begin{enumerate}
\item 個々のプログラムモジュールのオブジェクトファイルへのコンパイル
\item オブジェクトファイルを,C言語のオブジェクトファイルやシステムライ
ブラリとともにリンクし実行形式ファイルを作成する.
\end{enumerate}
	さらに\smlsharp{}コンパイラは,各オブジェクトの依存関係をシステ
ムの{\tt make}コマンドが解釈できる形式で出力することができます.
	この機能を使用すれば,C言語などとともに大規模システムを効率よく
開発することが可能です.
	本章では,この分割コンパイルシステムの概要を説明します.
\else%%%%%%%%%%%%%%%%%%%%%%%%%%%%%%%%%%%%%%%%%%%%%%%%%%%%%%%%%%%%%%%%%%%%
	One major feature of \smlsharp{} as a practical language is the
support of  true separate compilation that is compatible with system
development in C.
	In \smlsharp{}, one can develop a large software system as follows.
\begin{enumerate}
\item Compile source files into object files.
\item Link object files and additional C object files and system
libraries to generate an executable file.
\end{enumerate}
	Furthermore, \smlsharp{} compiler can produce file dependency in
the format of {\tt Makefile}.
	Using these features, \smlsharp{} can be used with C to develop
a large software efficiently and reliably.
	This chapter outline how to use separate compilation of \smlsharp.
\fi%%%%<<<<<<<<<<<<<<<<<<<<<<<<<<<<<<<<<<<<<<<<<<<<<<<<<<<<<<<<<<<<<<<<<<

\section{\txt{分割コンパイルの概要}{Separate compilation overview}}
\label{sec:tutorialSeparateCompilation}

\ifjp%>>>>>>>>>>>>>>>>>>>>>>>>>>>>>>>>>>>>>>>>>>>>>>>>>>>>>>>>>>>>>>>>>>>
	分割コンパイルシステムは以下の手順で使用します.
\begin{enumerate}
\item システムを構成するソースをコンパイル単位に分割する.
	この分割は,どのような大きさでもよく,分割された各々の機能は,
\smlsharp{}で記述可能な宣言で実現できるものであれば,特に制約はありませ
ん.
	ここでは,システムを{\tt part1}と{\tt part2}に分割したとします.
\item 分割した各要素のインターフェイスを定義する.
	このインターフェイスは,第\ref{sec:tutorialInterfaceFile}節で
説明するインターフェイス言語で記述します.
	システムを{\tt part1}と{\tt part2}に分割するシナリオでは,まず,
インターフェイスファイル{\tt part1.smi}と{\tt part2.smi}を用意します.
\item 各インターフェイスファイルに対して,それを実現するソースファイルを
開発し,それぞれのソースファイルをコンパイルしオブジェクトファイルを作成
する.

	上記シナリオの場合,{\tt part1.smi}と{\tt part2.smi}を実現するソー
スファイル{\tt part1.sml}と{\tt part2.sml}を開発し,コンパイルし
オブジェクトファイル{\tt part1.o}と{\tt part2.o}を作成します.

	この開発は,それぞれ完全に独立に行うことができます.
	ML系言語の場合,ソースファイルの開発では,コンパイラによる型推論
が威力を発揮します.
	分割コンパイルシステムは,その名のとおりそれぞれ独立にコンパイル
できます.
	例えば,{\tt part2}が{\tt part1}の関数などを使用する場合でも,
{\tt part1.sml}を開発する以前に,{\tt part2.sml}をコンパイルできます.
	この機能により,システムのどの部分でも,MLの型推論の機能をフルに
活用し,開発を行うことができます.
\item オブジェクトファイル集合を必要なライブラリと共にリンクし,実行形式
ファイルを作成する.

	例えば,{\tt part1}と{\tt part2}以外に,それらから参照されるC関
数のファイルやC言語で書いた独自のライブラリを使用する場合,それらの場所
をコンパイラに指示することによって,それらとともにをリンクしOS標準の実行
形式ファイルを生成できます.
	なお,\smlsharp{}コンパイラは,\smlsharp{}ランタイムの実行に必要
なC言語の標準ライブラリを常にリンクするため,標準ライブラリは指定なしに
使用することができます.
\end{enumerate}
	より本格的かつ高度なシナリオは以下のようなものが考えられます.
\begin{enumerate}
\item システムをC言語で書く部分と\smlsharp{}言語で書く部分に分割する.
\item C言語の部分は,ヘッダファイル(.hファイル)とソースファイル(.cファイル)を開発する.
\item \smlsharp{}言語の部分は,インターフェイスファイル(.smiファイル)
とソースファイル(.smlファイル)を開発する.
\item \smlsharp{}コンパイラの依存性出力機能を利用し,Makefileを作成する.
\item システムを{\tt make}コマンドを用いて,コンパイルとリンクを行う.
\end{enumerate}
	\smlsharp{}コンパイラ自身,C言語と\smlsharp{}で書かれていて,こ
の手順で種々のツールを含むシステムがコンパイルリンクされ,\smlsharp{}コ
ンパイラが作成されます.
\else%%%%%%%%%%%%%%%%%%%%%%%%%%%%%%%%%%%%%%%%%%%%%%%%%%%%%%%%%%%%%%%%%%%%
	A standard step of separate compilation consists of the
following steps.
\begin{enumerate}
\item Decompose the system into a set of compilation units.
	Each component cab be of any size and can contain any sequence of
\smlsharp{} declarations.
	Here we suppose that we decompose the system into {\tt part1}
and {\tt part2}.

\item For each decomposed component, define its interface.
	The interface is described in the interface language we shall
explain in Section~\ref{sec:tutorialInterfaceFile}.
	In our scenario of decomposing the system into 
{\tt part1} and {\tt part2}, we first define their interface files
{\tt part1.smi} and {\tt part2.smi}.

\item 
	For each interface file, develop a source file that implements
it.
	Each source file is independently compiled to an object file.
	
	In the above scenario, develop {\tt part1.sml} and {\tt
part2.sml} for {\tt part1.smi} and  {\tt part2.smi}, compile them to
generate  {\tt part1.o} and {\tt part2.o}.
	Developing each of the source files and its compilation can be
done independently of the other source file. 
	For example, even if {\tt part2} uses some functions in {\tt
part1}, {\tt part2.sml} can be compiled before writing {\tt part1.sml}.
	In ML, type error detection during compilation plays an
important role in source file development.
	So the ability to separately compiling each source file
independently of any other source files are important feature in large
software development.

\item 
	Link the set of object files with necessary libraries and
external object files to generate an executable file.
	For example, if {\tt part1} and {\tt part2} uses C files
and libraries, then link {\tt part1.o} and {\tt part2.o} with those 
compiled C files and libraries to generate an executable file.
\end{enumerate}
	
	A more advanced scenario can be the following.
\begin{enumerate}
\item Decompose the system into the part to be written in C and the part
to be written in \smlsharp{}.
\item For the C part, develop header files (.h files) and source files
(.c files). 
\item For the \smlsharp{} part, develop interface files (.smi files)
and source files (.sml files).
\item Create {\tt Makefile} using the dependency analysis of \smlsharp{}.
\item Do {\tt make} to compile and link the necessary files.
\end{enumerate}
	The \smlsharp{} system itself is a large project including a few
tools, all of which are written in C and \smlsharp{}, and developed in
this scenario.
\fi%%%%<<<<<<<<<<<<<<<<<<<<<<<<<<<<<<<<<<<<<<<<<<<<<<<<<<<<<<<<<<<<<<<<<<

\section{\txt{分割コンパイル例}{Separate compilation example}}
\label{sec:tutorialSeparateCompilationExample}

\ifjp%>>>>>>>>>>>>>>>>>>>>>>>>>>>>>>>>>>>>>>>>>>>>>>>>>>>>>>>>>>>>>>>>>>>
	前節のシナリオに従い,乱数を使うおみくじプログラムを例に,分割コ
ンパイルをして実行形式ファイルを作ってみましょう.
	システムを
\begin{itemize}
\item {\tt random}:乱数発生器
\item {\tt main}:メインパート
\end{itemize}
に分割することにします.
	まず,このインターフェイスファイルを以下のように設計します.
\begin{itemize}
\item {\tt random.smi}の定義.
\begin{program}
structure Random =
\\
struct
\\\myem
  val intit : int * int -> unit
\\\myem
  val genrand : unit -> int
\\
end
\end{program}
	このインターフェイスファイルは,このインターフェイスファイルを実
装するソースファイルが,他のファイルは参照せずに{\tt Random} structureを提
供することを表しています.
\item {\tt main.smi}の定義.
\begin{program}
\_require "basis.smi"\\
\_require "./random.smi"
\end{program}
	このインターフェイスファイルは,\smlsharp{}の基本ライブラリ{\tt
"basis.smi"}(The Standard ML Basis Library)とこのディレクトリにある
{\tt random.smi}を利用し,外部には何も提供しないことを意味しています.
\end{itemize}

	このインターフェイスを使えば,{\tt main.sml}と{\tt random.sml}を独
立に開発できます.
	{\tt main.sml}は図\ref{fig:main}のように定義できます.
	このファイルは,{\tt random.smi}を実装するソースファイルが存在し
なくても,コンパイルしエラーがないか確認することができます.
\begin{program}
\$ smlsharp -c main.sml
\end{program}
	{\tt -c}は\smlsharp{}コンパイラにコンパイルしオブジェクトファイ
ルを生成を指示します.
	対話型での使用と同様に型チェックをした後,エラーがなければ,
オブジェクトファイル{\tt main.o}を作ります.
	インターフェイスファイルは,ファイル名の{\tt .sml}を{\tt .smi}に変
えたものが自動的に使用されます.
	ソースコード冒頭に{\tt \_interface $filePath$}宣言を書くことにより
インターフェースファイルを明示的に指定することもできます.

	次に,{\tt random.sml}を開発します.
	高品質の乱数発生関数の開発は,数学的な知識と注意深いコーディング
が要求されます.
	ここでは,これはスクラッチから開発するのではなく,既存のCでの実
装を使うことにします.
	種々ある乱数発生アルゴリズムの中で,その品質と速度の両方の点から
{\tt Mersenne Twister}が信頼できます.
	このアルゴリズムはCのソースファイル{\tt mt19937ar.c}として提供されています.

	そこでこのファイルをダウンロードしましょう.
	インターネットで{\tt Mersenne Twister}あるいは{\tt
mt19937ar.c}をサーチすれば簡単に見つけることができます.
	このファイルには以下の関数が定義されています.
\begin{program}
void init\_genrand(unsigned long s)\\
void init\_by\_array(unsigned long init\_key[], int key\_length)\\
unsigned long genrand\_int32(void)\\
long genrand\_int31(void)\\
double genrand\_real1(void)\\
double genrand\_real2(void)\\
double genrand\_real3(void)\\
double genrand\_res53(void)\\
int main(void)
\end{program}
	この中の{\tt main}は,このアルゴリズムをテストするためのメイン関
数です.
	我々は新たな実行形式プログラムを作成するので,{\tt main}関数は,
我々のトップレベルファイル{\tt main.sml}をコンパイルしたオブジェクトファ
イルに含まれているはずです.
	そこで,{\tt mt19937ar.c}ファイルの関数{\tt int main(void)}の定
義をコメントアウトする必要があります.
	その他の関数は,\smlsharp{}から利用できるライブラリ関数です.
	ここでは以下の2つを使うことにします.
\begin{itemize}
\item {\tt void init\_genrand(unsigned long s)}.
シーズの長さを受け取りアルゴリズムを初期化する関数です.
シーズ{\tt s}は非負な整数ならなんでも構いません.
\item {\tt long genrand\_int31(void)}.初期化された後は,呼ばれる毎に31ビッ
トの符号なし整数(32ビット非負整数)のランダムな列を返します.
\end{itemize}

	そこで,これを利用して,{\tt random.sml}を以下のように定義します.
\begin{program}
structure Random =\\
struct\\
\myem  val init = \_import "init\_genrand" : int -> unit\\
\myem  val genrand = \_import "genrand\_int31" : unit -> int\\
end
\end{program}
	このソースファイルも,以下のコマンドにより,このファイルだけで単
独にコンパイルできます.
\begin{program}
\$ smlsharp -c random.sml
\end{program}
	このソースファイルの定義と並行して,({\tt main}関数をコメントア
ウトした){\tt Mersenne Twister}をコンパイルし,オブジェクトファイルを生
成しておきます.
\begin{program}
\$ gcc -c -o mt.o mt19937ar.c
\end{program}
	以上ですべてのソースファイルがそれぞれコンパイルされ,オブジェク
トファイルが作られたはずです.
	それらオブジェクトファイルは,トップレベルのインターフェイスファ
イルと必要なオブジェクトファイルを指定することによって,実行形式ファイル
が作成されます.
\begin{program}
\$ smlsharp main.smi mt.o
\end{program}
	\smlsharp{}コンパイラは,{\tt smi}ファイルを解析し,このファイル
から参照されている{\tt smi}ファイルを再帰的にたどり,対応するオブジェク
トファイルのリストを作り,コマンドラインに指定されたC言語のオブジェクト
ファイルと共に,システムのリンカーを起動し,実行形式ファイルを作成します.
	
\begin{figure}[b]
\begin{center}
\begin{tabular}{l}
\begin{minipage}{0.9\textwidth}
\begin{program}
fun main() =\\
\myem  let\\
\myem\myem    fun getInt () = \\
\myem\myem\myem        case TextIO.inputLine TextIO.stdIn of\\
\myem\myem\myem\ \       NONE => 0\\
\myem\myem\myem        | SOME s => (case Int.fromString s of NONE => 0 | SOME i => i)\\
\myem\myem    val seed = (print "好きな数を入力してください(0で終了です).";  getInt())\\
\myem  in\\
\myem\myem    if seed = 0 then ()\\
\myem\myem    else\\
\myem\myem\myem      let\\
\myem\myem\myem\myem        val \_ = Random.init seed;\\
\myem\myem\myem\myem        val oracle = Random.genrand()\\
\myem\myem\myem\myem        val message = \\
\myem\myem\myem\myem\myem            "あなたの運勢は," \verb|^|\\
\myem\myem\myem\myem\myem            (case oracle mod 4 of 0 => "大吉" | 1 => "小吉"| 2 => "吉" | 3 => "凶")\\
\myem\myem\myem\myem\myem            \verb|^| "です.\verb|\n|"\\
\myem\myem\myem\myem        val message = print message\\
\myem\myem\myem      in\\
\myem\myem\myem\myem        main ()\\
\myem\myem\myem      end\\
\myem  end\\
val \_ = main();
\end{program}
\end{minipage}
\end{tabular}
\caption{main.smlの例}
\label{fig:main}
\end{center}
\end{figure}

\else%%%%%%%%%%%%%%%%%%%%%%%%%%%%%%%%%%%%%%%%%%%%%%%%%%%%%%%%%%%%%%%%%%%%
	Following the simple scenario in the previous section, let us
develop an Omikuji (Japanese written oracle drawing) program as an example.
	Instead of asking Japanese sacred split, we use extremely good
random number generator.
	The system is divided into
\begin{itemize}
\item {\tt random}: a random number generator, and
\item {\tt main}: the main part.
\end{itemize}
	The fist step is to design interface files as follows.
\begin{itemize}
\item {\tt random.smi}:
\begin{program}
structure Random =
\\
struct
\\\myem
  val intit : int -> unit
\\\myem
  val genrand : unit -> int
\\
end
\end{program}
	This interface file says that the implementing source file provide
{\tt Random} structure without using any other files.
\item {\tt main.smi}:
\begin{program}
\_require "basis.smi"\\
\_require "./random.smi"
\end{program}
	This interface file says that it uses  {\tt "basis.smi"} (The
Standard ML Basis Library) and {\tt random.smi}, and provide no resource.
\end{itemize}

	Using these interface files, {\tt main.sml} and {\tt random.sml}
are developed independently.
	{\tt main.sml} can be given as in Figure~\ref{fig:main}.
	This file can be compiled without any source file that implements
{\tt random.smi}, and checks syntax and type errors by typing:
\begin{program}
\$ smlsharp -c main.sml
\end{program}
	The {\tt -c} switch instructs \smlsharp{} compiler
to compile the specified source file to an object file.
	The compiler checks its syntax and types and if no error is
detected then it produces {\tt main.o} file.
	{\tt main.smi} is chosen as its interface file by default.
	A specific interface file can be specified by including the
directive {\tt \_interface $filePath$} at the beginning of the source
file.

	Next,we develop {\tt random.sml}.
	Development of a high quality random number generator requires
expert knowledge in algebraic number theory and careful coding.
	Here, instead of developing one from scratch, we try to find
some existing quality code.
	Among a large number of implementations, perhaps {\tt Mersenne
Twister} is the best in its quality and efficiency.
	So we decide to use this algorithm, which is available as a C
source file {\tt mt19937ar.c}.
	
	Let us obtain the C source from the Internet by searching for
{\tt Mersenne Twister} or {\tt mt19937ar.c}.
	The file contains the following function definitions.
\begin{program}
void init\_genrand(unsigned long s)\\
void init\_by\_array(unsigned long init\_key[], int key\_length)\\
unsigned long genrand\_int32(void)\\
long genrand\_int31(void)\\
double genrand\_real1(void)\\
double genrand\_real2(void)\\
double genrand\_real3(void)\\
double genrand\_res53(void)\\
int main(void)
\end{program}
	Among thme, {\tt main} is a main function that test the
algorithm.
	Since we are defining our executable, the main function should
be in the compiled object file of our top level source file {\tt main.sml}.
	So we comment out {\tt int main(void)} function in the
file. {\tt mt19937ar.c}.
	All the others functions can be used from our program.
	Here we decide to use the following two.
\begin{itemize}
\item {\tt void init\_genrand(unsigned long s)} for initializing the
algorithm with the seed length {\tt s}, which can be any non-negative
integer.
\item {\tt long genrand\_int31(void)} for generating 31 bit unsigned
(i.e.\ 32 bit non-negative) random number sequence.
\end{itemize}
	The {\tt random.sml} can then be defined as the following code
that simply call these functions.
\begin{program}
structure Random =\\
struct\\
\myem  val init = \_import "init\_genrand" : int -> unit\\
\myem  val genrand = \_import "genrand\_int31" : unit -> int\\
end
\end{program}

	This source file can be compiled independently of other files
by the following command.
\begin{program}
\$ smlsharp -c random.sml
\end{program}
	In doing this, we compile {\tt Mersenne Twister} (aftre
commenting out its {\tt main} function).
\begin{program}
\$ gcc -c -o mt.o mt19937ar.c
\end{program}
	We now have all the object files for the program.
	We can link them to an executable file by specifying the
top-level interface file and the external object files referenced
through {\tt \_import} declarations.
\begin{program}
\$ smlsharp main.smi mt.o
\end{program}
	\smlsharp{} analyzes {\tt smi} file,traverse all the interface
files referenced from this file, make a list of object files
corresponding to the interface files, and then links all the object
files with those specified in the command line argument to generate an
executable file.
	
\begin{figure}[b]
\begin{center}
\begin{tabular}{l}
\begin{minipage}{0.9\textwidth}
\begin{program}
fun main() =\\
\myem  let\\
\myem\myem    fun getInt () = \\
\myem\myem\myem        case TextIO.inputLine TextIO.stdIn of\\
\myem\myem\myem\ \       NONE => 0\\
\myem\myem\myem        | SOME s => (case Int.fromString s of NONE => 0 | SOME i => i)\\
\myem\myem    val seed = (print "input a number of your choice (0 for exit)";  getInt())\\
\myem  in\\
\myem\myem    if seed = 0 then ()\\
\myem\myem    else\\
\myem\myem\myem      let\\
\myem\myem\myem\myem        val \_ = Random.init seed;\\
\myem\myem\myem\myem        val oracle = Random.genrand()\\
\myem\myem\myem\myem        val message = \\
\myem\myem\myem\myem\myem            " あなたの運勢は," \verb|^|\\
\myem\myem\myem\myem\myem            (case oracle mod 4 of 0 => "大吉" | 1 => "小吉"| 2 => "吉" | 3 => "凶")\\
\myem\myem\myem\myem\myem            \verb|^| "です.\verb|\n|"\\
\myem\myem\myem\myem        val message = print message\\
\myem\myem\myem      in\\
\myem\myem\myem\myem        main ()\\
\myem\myem\myem      end\\
\myem  end\\
val \_ = main();
\end{program}
\end{minipage}
\end{tabular}
\caption{Example of main.sml}
\label{fig:main}
\end{center}
\end{figure}
\fi%%%%<<<<<<<<<<<<<<<<<<<<<<<<<<<<<<<<<<<<<<<<<<<<<<<<<<<<<<<<<<<<<<<<<<
	
\section{\txt{インターフェイスファイルの構造}{Structure of interface files}}
\label{sec:tutorialInterfaceFile}

\ifjp%>>>>>>>>>>>>>>>>>>>>>>>>>>>>>>>>>>>>>>>>>>>>>>>>>>>>>>>>>>>>>>>>>>>
	インターフェイスファイルは,分割コンパイルするコンパイル単位のイ
ンターフェイスを記述したファイルです.
	インターフェイスファイルの内容は,
Require宣言
と
Provide宣言
からなります.
	Require宣言は,コンパイル単位が使用する他のコンパイル単位を
以下の形の宣言として列挙します.
\begin{program}
\_require $smiFilePath$
\end{program}
	$smiFilePath$は,他のコンパイル単位のインターフェイスファイル
({\tt .smi}ファイル)です.
	よく使用されるインターフェイスの集合には,それらをまとめた名前が
付けられ,システムに登録されています.
	代表的なものは,The Definition of Standard ML Basis Libraryで実装
が義務付けられている基本ライブラリのインターフェイスファイルをすべて含む
{\tt basis.smi}です.
	プログラムのインターフェイスファイルの冒頭に
\begin{program}
\_require "basis.smi";
\end{program}
と書いておくと,基本ライブラリが使用可能となります.

	Provide宣言は,コンパイル単位が他のコンパイル単位に提供する
\smlsharp{}の資源を記述します.
	記述する資源は,おおよそ,Standard MLのソースファイルの宣言定義
できるものすべてと考えるとよいでしょう.
	具体的には,以下のものが含まれます.
\begin{itemize}
\item {\tt datatype}定義.
\item {\tt type}定義.
\item {\tt exception}定義.
\item {\tt infix}宣言.
\item 変数とその型の定義.
\item モジュールの定義
\item {\tt functor}の定義
\end{itemize}
	2分探索木のインターフェイスファイル{\tt queue.smi}とそのインター
フェイスを実装する{\tt queue.sml}の例を以下に示します.
\begin{figure}
\begin{center}
\begin{tabular}{l}
\begin{minipage}{0.9\textwidth}
queue.smiファイル:
\begin{program}
\_require "basis.smi"\\
structure Queue =\\
struct\\
\myem  datatype 'a queue = Q of 'a list * 'a list\\
\myem  exception Dequeue\\
\myem  val empty : 'a queue\\
\myem  val isEmpty : 'a queue -> bool\\
\myem  val enqueue : 'a queue * 'a -> 'a queue\\
\myem  val dequeue : 'a queue -> 'a queue * 'a\\
end
\end{program}
queue.smlファイル:
\begin{program}
structure Queue =\\
struct\\
\myem  datatype 'a queue = Q of 'a list * 'a list\\
\myem  exception Dequeue\\
\myem  val empty = Q ([],[])\\
\myem  fun isEmpty (Q ([],[])) = true\\
\myem\myem      | isEmpty \_ = false\\
\myem  fun enqueue (Q(Old,New),x) = Q (Old,x::New)\}\\
\myem  fun dequeue (Q (hd::tl,New)) = (Q (tl,New), hd)\\
\myem\myem      | dequeue (Q ([],\_) = raise Dequeue\\
\myem\myem      | dequeue (Q(Old,New) = dequeue (Q(rev New,[]))\\
end
\end{program}
\end{minipage}
\end{tabular}
\caption{インターフェイスファイルの例}
\label{fig:interfacefileExample}
\end{center}
\end{figure}

	{\tt queue.smi}ファイルの冒頭に書かれている{\tt \_require
"basis.smi"}は,
\begin{itemize}
\item この{\tt queue.smi}ファイルが基本ライブラリを使うこと,
\item この{\tt queue.smi}ファイルの中で定義されずに使われている{\tt 'a
list}などの型は,基本ライブラリに定義されていること
\end{itemize}
を表しています.
	それ以降の定義がProvide宣言部分です.
	この例では,このインターフェイスを実装するモジュールが{\tt
Queue}というstructureを提供することを示しています.
	この例から理解されるとおり,インターフェイスファイルのProvide宣言
はStandard MLのシグネチャと似た構文で記述します.
	しかしシグネチャと考え方が大きく違うのは,{\tt datatype}や{\tt
exception}などは仕様ではなく,実体に対応するということです.
	{\tt queue.smi}の中で定義される{\tt datatype 'a queue}と{\tt
exception Dequeue}は,{\tt queue.sml}で定義される型および例外そのものと
して扱われます.
	従って,この{\tt queue.smi}を複数のモジュールで{\tt \_require}宣
言を通じて利用しても,{\tt queue.smi}で定義された同一のものとして扱われ
ます.
\else%%%%%%%%%%%%%%%%%%%%%%%%%%%%%%%%%%%%%%%%%%%%%%%%%%%%%%%%%%%%%%%%%%%%

	An interface file describes the interface of a source file to be
compiled separately.
	Its contents consists of 
{\bf Require} declarations and
{\bf Provide} declarations.
	Require declarations specifies the list of interface files
used in this compilation unit as a sequence of declarations of the
following form.
\begin{program}
\_require $smiFilePath$
\end{program}
	$smiFilePath$ is an interface file (smi file) of other compile unit.
	Frequently used interface files can be grouped together and given a
name.
	\smlsharp{} maintain a few of them.
	A useful example is {\tt basis.smi} which contain
all the (mandatory) names (types, functions etc) defined in Standard ML
Basis Library. 
	In order to make the basis library available, write the
following at the beginning of its interface file.
\begin{program}
\_require "basis.smi";
\end{program}

	Provide declarations describes the resources implemented in this
compile unit.
	Resources corresponds to those definable as \smlsharp{}
declarations, including the following.
\begin{itemize}
\item {\tt datatype} definitions
\item {\tt type} definitions
\item {\tt exception} declarations
\item {\tt infix} declarations
\item variable definitions
\item module definitions
\item {\tt functor} definitions
\end{itemize}
	Below is an example interface file {\tt queue.smi}
and its implementation {\tt queue.sml}.
\begin{figure}
\begin{center}
\begin{tabular}{l}
\begin{minipage}{0.9\textwidth}
queue.smi file:
\begin{program}
\_require "basis.smi"\\
structure Queue =\\
struct\\
\myem  datatype 'a queue = Q of 'a list * 'a list\\
\myem  exception Dequeue\\
\myem  val empty : 'a queue\\
\myem  val isEmpty : 'a queue -> bool\\
\myem  val enqueue : 'a queue * 'a -> 'a queue\\
\myem  val dequeue : 'a queue -> 'a queue * 'a\\
end
\end{program}
queue.sml file:
\begin{program}
structure Queue =\\
struct\\
\myem  datatype 'a queue = Q of 'a list * 'a list\\
\myem  exception Dequeue\\
\myem  val empty = Q ([],[])\\
\myem  fun isEmpty (Q ([],[])) = true\\
\myem\myem      | isEmpty \_ = false\\
\myem  fun enqueue (Q(Old,New),x) = Q (Old,x::New)\}\\
\myem  fun dequeue (Q (hd::tl,New)) = (Q (tl,New), hd)\\
\myem\myem      | dequeue (Q ([],\_) = raise Dequeue\\
\myem\myem      | dequeue (Q(Old,New) = dequeue (Q(rev New,[]))\\
end
\end{program}
\end{minipage}
\end{tabular}
\caption{Example of interface file}
\label{fig:interfacefileExample}
\end{center}
\end{figure}

	As seen in this example, Provide declarations in interface file
resembles Standard ML signatures.
	A big difference is that in provide declarations in an interface
file, {\tt datatype} and {\tt exception} are not specifications but they
represent unique entities.
	{\tt datatype 'a queue} and {\tt exception Dequeue} declared in
{\tt queue.smi} are treated as the generative type and exception in its
implementation.
	As a result, they corresponds to the same entity even if {\tt
queue.smi} is used in multiple interface files through  {\tt \_require}
declarations.

\fi%%%%<<<<<<<<<<<<<<<<<<<<<<<<<<<<<<<<<<<<<<<<<<<<<<<<<<<<<<<<<<<<<<<<<<

\section{\txt{型の隠蔽}{Opaque types}}
\label{sec:tutorialOpaqueTypeInterface}

\ifjp%>>>>>>>>>>>>>>>>>>>>>>>>>>>>>>>>>>>>>>>>>>>>>>>>>>>>>>>>>>>>>>>>>>>
	前節の例から理解されるとおり,インターフェイスファイルの基本的考え
方は,以下のとおりです.
\begin{enumerate}
\item {\tt datatype}や{\tt exception}などのコンパイル時に定義される資源
(静的資源)は,将来そのインターフェイスファイルで定義される実体そのものを
記述する.
\item 関数や変数などの実行時の値を表すものは,その型のみを宣言する.
\end{enumerate}
	これら情報が,コンパイラがこのインターフェイスファイルを使う別のソー
スコードをコンパイルする上で必要かつ十分な情報です.
	しかし,この原則だけでば,Standard MLのモジュールシステムが提供
する型情報の隠蔽の機能を使うことができません.
	例えば,前節のインターフェイスファイル{\tt queue.smi}では,{\tt
'a queue}の実装が,{\tt Q of 'a list * 'a list}と定義され公開されていま
すが,この実装の詳細は隠蔽したい場合が多いと思われます.

	この問題の解決のために,インタフェイスファイルに
\begin{program}
type $tyvars$ $tyid$ (= $typeRep$) \ \ \ (* 括弧はそのまま記述する *)\\
eqtype $tyvars$ $tyid$ (= $typeRep$) \ \ \ (* 括弧はそのまま記述する *)
\end{program}
の形の隠蔽された型の宣言を許しています.
	この宣言は,型$tyid$が定義されその実装の表現は$typeRep$であるが,
その内容はこのインターフェイスの利用者からは隠されることを表しています.
	シグネチャ同様,{\tt type}宣言は,同一性判定を許さない型,{\tt
eqtype}は同一性判定が可能な型を表します.
	実装情報$typeRep$には,その型のメモリ上での表現を代表する名前で
指定します.
	実装情報$typeRep$に指定できる主な名前は以下の通りです.
\begin{itemize}
\item
{\tt word},{\tt int},{\tt word8},{\tt char},{\tt real},{\tt real32}な
どの原子型の実装表現.
\item {\tt contag}.
	引数を持たないコンストラクタのみから成るデータ型の実装表現.
\item {\tt boxed}.
	ヒープ上のオブジェクトへのポインタによる実装表現.
	レコード型,配列型,関数型,{\tt contag}に該当しない形の
{\tt datatype}で宣言された型などが該当します.
\end{itemize}
	例えば,{\tt queue.smi}の{\tt datatype 'a queue}宣言のかわりに
\begin{program}
type 'a queue (= boxed)
\end{program}
と書くことができます.
	実装情報は型とは異なる概念であり,異なる名前空間を持つことに
注意してください.

	さらに,シグネチャの場合同様,インターフェイスファイルは,そのイン
ターフェイスを実装するソースが定義する変数をすべて網羅している必要はありま
せん.
	インターフェイスに宣言されたもののみが,そのインターフェイスを{\tt
\_require}宣言を通じて利用するユーザに見えるようになります.
\else%%%%%%%%%%%%%%%%%%%%%%%%%%%%%%%%%%%%%%%%%%%%%%%%%%%%%%%%%%%%%%%%%%%%
	The principle underlying interface declaration are the
following.
\begin{enumerate}
\item 
	Define compile-time resources (static entities) such as {\tt
datatype} and {\tt exception} themselves just the same way as they will
be defined in a source file.
\item 
	Declare only the types of runtime resources such as functions
and variables.
\end{enumerate}
	They are necessary and sufficient information to compile other
source file that uses this interface file.
	However, this principle alone cannot support ML's information
hiding through opaque signatures.
	For example, in the previous interface file {\tt queue.smi},
the {\tt 'a queue} type is explicitly defined to be {\tt Q of 'a list *
'a list} and this information is open to the user of this interface
file, but we often want to hide this implementation details.

	To solve this problem, the interface language introduces the
following opaque type declarations.
\begin{program}
type $tyvars$ $tyid$ (= $typeRep$) \ \ \ 
(* describe paranthesis as they are *)\\
eqtype $tyvars$ $tyid$ (= $typeRep$) \ \ \
(* describe paranthesis as they are *)
\end{program}
	These declaration reveals to the compiler that the internal
representation of $tyid$ is $typeRep$ but this information is not
available to the code that uses this interface file through {\tt
\_require} declaration.
	As in signature, {\tt type} declaration defines a type on which 
equality operation is not defined and {\tt eqtype} declaration defines a
type with on which  equality operation is define.
 	Possible internal representations $typeRep$ include the
following.
\begin{itemize}
\item Atomic data representations, such as
{\tt word}, {\tt int}, {\tt word8}, {\tt char}, {\tt real}, {\tt real32}.
\item {\tt contag}.
	The internal representation of a datatype consisting only of
constructors with no argument.
\item {\tt boxed}.
	Heap allocated objects.
	Records, arrays, functions, and any datatype that is not compliant
with {\tt contag} have this representation.
\end{itemize}
	For example,instead of {\tt datatype 'a queue} declaration in
{\tt queue.smi}, one can write the following.
\begin{program}
type 'a queue (= boxed)
\end{program}
	Note that an internal representation is not a type.

	Furthermore, as in signature, interface file need not
exhaustively list all the resources the implementation may define.
	Only those resources defined in the interface file become
visible to the code that use it through {\tt \_require} declaration.
\fi%%%%<<<<<<<<<<<<<<<<<<<<<<<<<<<<<<<<<<<<<<<<<<<<<<<<<<<<<<<<<<<<<<<<<<

\section{\txt{シグネチャの扱い}{Treatment of signatures}}
\label{sec:tutorialSignatureInInterface}

\ifjp%>>>>>>>>>>>>>>>>>>>>>>>>>>>>>>>>>>>>>>>>>>>>>>>>>>>>>>>>>>>>>>>>>>>
	Standard ML言語では,第\ref{sec:tutorialInterfaceFile}節で説明した資源以
外に,シグネチャも名前の付いた資源です.
	例えば,{\tt Queue}ストラクチャに対しては,{\tt QUEUE}シグネチャ
が定義されていると便利です.
	\smlsharp{}のインタフェイスのRequire宣言には,以下の構文により,
シグネチャファイルの使用宣言も書くことができます.
\begin{program}
\_require $sigFilePath$
\end{program}
	$sigFilePath$はシグネチャファイルのパス名です.
	この機構を理解するために,Standard MLのシグネチャの以下の性質を
理解する必要があります.
\begin{itemize}
\item シグネチャは,他のstructureで定義された型(の実体)の参照を含むこ
とができる.
\item シグネチャ自身は,型(の実体)を生成しない.
\end{itemize}
	\smlsharp{}コンパイラは,{\tt \_require $sigFilePath$}宣言を以下
のように取り扱います.
\begin{itemize}
\item シグネチャ以外のすべてのRequire宣言の下で,$sigFilePath$ファイルで
定義されたシグネチャを評価.
\item この宣言を含むインターフェイスファイルを{\tt \_require}宣言を通じて
利用するソースコードの冒頭で,評価済みの$sigFilePath$ファイルが展開され
ているとみなす.
\end{itemize}
	図\ref{fig:queueSignature}に,シグネチャを含み型が隠蔽された
待ち行列のインタフェイスと実装の例を示します.
	この例では,実装ファイルに型を隠蔽するシグネチャ制約を付けても,
利用者にとっての効果は変わりません.
\begin{figure}
\begin{center}
\begin{tabular}{l}
\begin{minipage}{0.9\textwidth}
queue-sig.smlファイル:
\begin{program}
signature Queue =\\
sig\\
\myem  datatype 'a queue = Q of 'a list * 'a list\\
\myem  exception Dequeue\\
\myem  val empty : 'a queue\\
\myem  val isEmpty : 'a queue -> bool\\
\myem  val enqueue : 'a queue * 'a -> 'a queue\\
\myem  val dequeue : 'a queue -> 'a queue * 'a\\
end
\end{program}
queue.smiファイル:
\begin{program}
\_require "basis.smi"\\
\_require "queue-sig.sml"\\
\ \\
structure Queue =\\
struct\\
\myem  type 'a queue (= boxed)\\
\myem  exception Dequeue\\
\myem  val empty : 'a queue\\
\myem  val isEmpty : 'a queue -> bool\\
\myem  val enqueue : 'a queue * 'a -> 'a queue\\
\myem  val dequeue : 'a queue -> 'a queue * 'a\\
end
\end{program}
queue.smlファイル:
\begin{program}
structure Queue : QUEUE =\\
struct\\
\myem  datatype 'a queue = Q of 'a list * 'a list\\
\myem  exception Dequeue\\
\myem  val empty = Q ([],[])\\
\myem  fun isEmpty (Q ([],[])) = true\\
\myem\myem      | isEmpty \_ = false\\
\myem  fun enqueue (Q(Old,New),x) = Q (Old,x::New)\}\\
\myem  fun dequeue (Q (hd::tl,New)) = (Q (tl,New), hd)\\
\myem\myem      | dequeue (Q ([],\_) = raise Dequeue\\
\myem\myem      | dequeue (Q(Old,New) = dequeue (Q(rev New,[]))\\
end
\end{program}
\end{minipage}
\end{tabular}
\caption{インターフェイスファイルの例}
\label{fig:queueSignature}
\end{center}
\end{figure}
\else%%%%%%%%%%%%%%%%%%%%%%%%%%%%%%%%%%%%%%%%%%%%%%%%%%%%%%%%%%%%%%%%%%%%
	In Standard ML,  in addition to resources explained in
Section~\ref{sec:tutorialInterfaceFile}, signatures are also named
resources.
	For example,one should be able to provide {\tt QUEUE} signature
as well as {\tt Queue} structure.
	In require declaration in an interface file, signature files can
be specified in the following syntax. 
\begin{program}
\_require $sigFilePath$
\end{program}
	$sigFilePath$ is a path to a signature file.
	To understand this mechanism, let us review some properties of
Standard ML signatures.
\begin{itemize}
\item A  signature may reference some types define in some other
structures.
\item A signature itself does not define any type.
\end{itemize}
	To deal with this situation properly, \smlsharp{} compiler 
treats {\tt \_require $sigFilePath$} declaration as follows.
\begin{itemize}
\item 
	Evaluate the signature in the file $sigFilePath$
in the context generated by all the other Require declarations.
\item 
	The signature declaration is inserted at the beginning of the
source file that use this interface file through {\tt \_require}.
\end{itemize}
	Figure \ref{fig:queueSignature} shows an interface file
containing  an opaque signature.
\begin{figure}
\begin{center}
\begin{tabular}{l}
\begin{minipage}{0.9\textwidth}
queue-sig.sml {\rm file:}
\begin{program}
signature Queue =\\
sig\\
\myem  datatype 'a queue = Q of 'a list * 'a list\\
\myem  exception Dequeue\\
\myem  val empty : 'a queue\\
\myem  val isEmpty : 'a queue -> bool\\
\myem  val enqueue : 'a queue * 'a -> 'a queue\\
\myem  val dequeue : 'a queue -> 'a queue * 'a\\
end
\end{program}
queue.smi {\rm file:}
\begin{program}
\_require "basis.smi"\\
\_require "queue-sig.sml"\\
\ \\
structure Queue =\\
struct\\
\myem  type 'a queue (= boxed)\\
\myem  exception Dequeue\\
\myem  val empty : 'a queue\\
\myem  val isEmpty : 'a queue -> bool\\
\myem  val enqueue : 'a queue * 'a -> 'a queue\\
\myem  val dequeue : 'a queue -> 'a queue * 'a\\
end
\end{program}
queue.sml {\rm file:}
\begin{program}
structure Queue : QUEUE =\\
struct\\
\myem  datatype 'a queue = Q of 'a list * 'a list\\
\myem  exception Dequeue\\
\myem  val empty = Q ([],[])\\
\myem  fun isEmpty (Q ([],[])) = true\\
\myem\myem      | isEmpty \_ = false\\
\myem  fun enqueue (Q(Old,New),x) = Q (Old,x::New)\}\\
\myem  fun dequeue (Q (hd::tl,New)) = (Q (tl,New), hd)\\
\myem\myem      | dequeue (Q ([],\_) = raise Dequeue\\
\myem\myem      | dequeue (Q(Old,New) = dequeue (Q(rev New,[]))\\
end
\end{program}
\end{minipage}
\end{tabular}
\caption{Example of interface file with signatures}
\label{fig:queueSignature}
\end{center}
\end{figure}
\fi%%%%<<<<<<<<<<<<<<<<<<<<<<<<<<<<<<<<<<<<<<<<<<<<<<<<<<<<<<<<<<<<<<<<<<

\section{\txt{ファンクタのサポート}{Functor support}}
\label{sec:tutorialFunctorInInterface}

\ifjp%>>>>>>>>>>>>>>>>>>>>>>>>>>>>>>>>>>>>>>>>>>>>>>>>>>>>>>>>>>>>>>>>>>>
	\smlsharp{}の分割コンパイルシステムは,ファンクタもオブジェク
トファイルに分割コンパイルし,他のコンパイル単位から{\tt \_require}宣言
を通じて利用することができます.
	ファンクタのインターフェイスファイルは,そのProvide宣言に以下のよ
うに記述します.
\begin{program}
functor $id$($signature$) =\\
struct\\
\myem (* この部分はstructureのProvideと同一 *)\\
end
\end{program}
	ここで,$signature$はStandard ML構文規則に従うシグネチャ宣言です.
	インターフェイスと似ていますが,インターフェイスではなく,通常のシグ
ネチャが書けます.
	以下に二分探索木を実現するファンクタのインターフェイ
スファイルの例をしめします.
\begin{program}
\_require "basis.smi"\\
functor BalancedBinaryTree\\
\myem  (A:sig\\
\myem\myem\myem      type key\\
\myem\myem\myem      val comp : key * key -> order\\
\myem\myem    end\\
\myem  ) =\\
struct\\
\myem type 'a binaryTree (= boxed)\\
\myem  val empty : 'a binaryTree\\
\myem  val isEmpty : 'a binaryTree -> bool\\
\myem  val singleton : key * 'a -> 'a binaryTree\\
\myem  val insert : 'a binaryTree * A.key * 'a -> 'a binaryTree\\
\myem  val delete : 'a binaryTree * key -> 'a binaryTree\\
\myem  val find : 'a binaryTree * A.key -> 'a option\\
end
\end{program}

	ただし,この機構を利用するプログラマは,以下の点に留意する必要が
あります.
\begin{itemize}
\item functorはモジュール分割のための道具ではない.
	分割コンパイルができないML系言語処理系では,モジュールの間の直接
の依存関係を断ち切る手段としてのファンクタの使用が推奨されることがありました.
	例えば,
\begin{quote}
\begin{minipage}{0.9\textwidth}
A.smlファイル:
\begin{program}
structure A =\\
struct\\
\myem   ...\\
end
\end{program}
B.smlファイル:
\begin{program}
structure B = \\
struct\\
\myem  open A\\
\myem  ...\\
end
\end{program}
\end{minipage}
\end{quote}
と書くと{\tt B.sml}ファイルが別な実装ファイル{\tt A.sml}ファイルに直接依
存してしまいます.
	この{\tt B.sml}をファンクタを使い以下のように書きなおせば
依存性は解消されます.
\begin{quote}
\begin{minipage}{0.9\textwidth}
B.smlファイル:
\begin{program}
functor B(A:sig ... end) = \\
struct\\
\myem  open A\\
\myem  ...\\
end
\end{program}
\end{minipage}
\end{quote}	
	この機能は,分割コンパイルの機能そのものです.
	分割コンパイルとリンクの機能を完全にサポートしている\smlsharp{}
では,この目的のためにファンクタを使う必要はなく,このような使用は
避けるべきです.

\item ファンクタの利用にはコストがかかる.
	ファンクタは,関数などの値以外に型もパラメタとして受け取る能力があ
ります.
	これが,ファンクタなしでは達成できないファンクタ本来の機能です.
	しかし,同時に型をパラメタとして受け取り,その型に応じた処理を行
うため,ファンクタ本体のコンパイルには,型が決まっているストラクチャに比べて
コンパイルにもコンパイルされたオブジェクトコードにもオーバヘッドが生まれ
ます.
	これは,高度な機能を使用する上で避けられないことです.
	ファンクタは,このオーバヘッドを意識して,コードすべき高度な機能です.
\end{itemize}

	現在の\smlsharp{}のファンクタの実装には以下の制限があります.
\begin{itemize}
\item
	ファンクタの引数に型引数を持つ抽象型コンストラクタが含まれている
場合,その型コンストラクタには{\tt boxed}実装表現を持つ型のみを適用する
ことができます.
	例えば,\smlsharp{}では以下の例はコンパイルエラーになります.
\begin{program}
\# functor F(type 'a t) = struct end
   structure X = F(type 'a t = int);
(interactive):2.17-2.34 Error:
  (name evaluation "440") Functor parameter restriction: t
\end{program}
\end{itemize}


\else%%%%%%%%%%%%%%%%%%%%%%%%%%%%%%%%%%%%%%%%%%%%%%%%%%%%%%%%%%%%%%%%%%%%
	\smlsharp{} can separately compile functor into an object file,
and can be used from other compilation unit through {\tt \_require}
declaration.
	To provide a functor, write the following in its interface file.
\begin{program}
functor $id$($signature$) =\\
struct\\
\myem (* the same as Provide declarations of structure *)\\
end
\end{program}
	$signature$ is a Standard ML signature.
	Below is an example of an interface file for binary search tree.
\begin{program}
\_require "basis.smi"\\
functor BalancedBinaryTree\\
\myem  (A:sig\\
\myem\myem\myem      type key\\
\myem\myem\myem      val comp : key * key -> order\\
\myem\myem    end\\
\myem  ) =\\
struct\\
\myem type 'a binaryTree (= boxed)\\
\myem  val empty : 'a binaryTree\\
\myem  val isEmpty : 'a binaryTree -> bool\\
\myem  val singleton : key * 'a -> 'a binaryTree\\
\myem  val insert : 'a binaryTree * A.key * 'a -> 'a binaryTree\\
\myem  val delete : 'a binaryTree * key -> 'a binaryTree\\
\myem  val find : 'a binaryTree * A.key -> 'a option\\
end
\end{program}

	In using functors in separate compilation, one should note the
following.
\begin{itemize}
\item {\bf Functor is not a mechanism for separate compilation.}
	In some existing practice of ML, probably due to the lack of
separate compilation, functors can be used to compile some modules
independently from the others.
	For example, if one write
\begin{quote}
\begin{minipage}{0.9\textwidth}
A.sml {\rm file:}
\begin{program}
structure A =\\
struct\\
\myem   ...\\
end
\end{program}
B.sml {\rm file:}
\begin{program}
structure B = \\
struct\\
\myem  open A\\
\myem  ...\\
end
\end{program}
\end{minipage}
\end{quote}
then {\tt B.sml} file directly depends on {\tt A.sml} file.
	If one rewrite {\tt B.sml} using a functor as below then this
dependency can be avoided.
\begin{quote}
\begin{minipage}{0.9\textwidth}
B.sml {\rm file:}
\begin{program}
functor B(A:sig ... end) = \\
struct\\
\myem  open A\\
\myem  ...\\
end
\end{program}
\end{minipage}
\end{quote}	
	This is exactly what separate compilation achieves.
	A system such as \smlsharp{} where a complete separate
compilation is supported, this form of functor usage is unnecessary 
and undesirable.

\item {\bf Usage of functor incurs some overhead.}
	Functors can take types as parameters and therefore strictly
more powerful than polymorphic functions.
	However, this type parameterization requires the compiler to
generate code that behaves differently depending on the argument types. 
	The resulting code  inevitably incurs more overhead than the
corresponding code with the type argument predetermined (i.e. ordinary
structures).
	The programmer who use functor should be aware of this const and
restrict functors in cases where the advanced feature of explicit type
parameterization is really required.
\end{itemize}

	In the current version of \smlsharp{} has the following limitation
on the usage of functors.
\begin{itemize}
\item
	If a functor has a formal abstract type constructor with type
arguments, only a type of {\tt boxed} internal representation can be
applied to the formal type constructor.
	The following example causes a compile error in \smlsharp{}.
\begin{program}
\# functor F(type 'a t) = struct end
   structure X = F(type 'a t = int);
(interactive):2.17-2.34 Error:
  (name evaluation "440") Functor parameter restriction: t
\end{program}
\end{itemize}

\fi%%%%<<<<<<<<<<<<<<<<<<<<<<<<<<<<<<<<<<<<<<<<<<<<<<<<<<<<<<<<<<<<<<<<<<


\section{\txt{リプリケーション宣言}{Replications}}
\label{sec:tutorialReplicationInInterface}

\ifjp%>>>>>>>>>>>>>>>>>>>>>>>>>>>>>>>>>>>>>>>>>>>>>>>>>>>>>>>>>>>>>>>>>>>
	インタフェイスファイルは,資源の実体の表現です.
	従って,例えば
\begin{program}
structure Foo = \\
struct\\
\myem structure A = Bar\\
\myem structure B = Bar\\
...\\
end

\end{program}
のような構造のモジュールに対して,シグネチャと違い,
\begin{program}
structure Foo = \\
struct\\
\myem structure A : SigBar\\
\myem structure B : SigBar\\
...\\
end
\end{program}
のような書き方はできません.
	これまでのメカニズムのみでは,{\tt Bar}の内容をくり返し書く必要があり
ます.
	この冗長さを抑止するために,インタフェイスファイルは,以下の資源
のリプリケーション(複製)宣言を許しています.
\begin{itemize}
\item {\tt structure $id$ =  $path$}
\item {\tt exception $id$ = $path$}
\item {\tt datatype $id$ = datatype $path$}
\item {\tt val $id$ = $path$}
\end{itemize}
	これによって,すでに定義済みの資源に対しては,インタフェイスファ
イルであっても,通常のソースファイルと同様にその複製の宣言を書くことがで
きます.
	例えば,{\tt Bar}ストラクチャを提供するインタフェイスファイルが{\tt
bar.smi}であるとすると,以下のように書くことができます.
\begin{program}
\_require "bar.smi"\\
structure Foo = \\
struct\\
\myem structure A = Bar\\
\myem structure B = Bar\\
...\\
end
\end{program}
\else%%%%%%%%%%%%%%%%%%%%%%%%%%%%%%%%%%%%%%%%%%%%%%%%%%%%%%%%%%%%%%%%%%%%
	An interface file describes resources themselves.
	So, for example, for a module of the form
\begin{program}
structure Foo = \\
struct\\
\myem structure A = Bar\\
\myem structure B = Bar\\
...\\
end

\end{program}
one cannot write its interface as
\begin{program}
structure Foo = \\
struct\\
\myem structure A : SigBar\\
\myem structure B : SigBar\\
...\\
end
\end{program}
	The interface language mechanism so far describes forces us to
repeat most of the contents of {\tt Bar} twice.
	To suppress this redundancy, the interface language allow
replication declaration of the following form.
\begin{itemize}
\item {\tt structure $id$ =  $path$}
\item {\tt exception $id$ = $path$}
\item {\tt datatype $id$ = datatype $path$}
\item {\tt val $id$ = $path$}
\end{itemize}
	If you know that in advance that two resources are replication of
the same resources, then you can declare them as replication using this
mechanism in an interface file.
	For example, {\tt Bar} structure is provided by an interface
file {\tt bar.smi} and that you know that both {\tt A} and {\tt B}
should be replications of {\tt Bar}, then you can simply write the
following interface file.
\begin{program}
\_require "bar.smi"\\
structure Foo = \\
struct\\
\myem structure A = Bar\\
\myem structure B = Bar\\
...\\
end
\end{program}
\fi%%%%<<<<<<<<<<<<<<<<<<<<<<<<<<<<<<<<<<<<<<<<<<<<<<<<<<<<<<<<<<<<<<<<<<

\section{\txt{インタフェイス言語の文法}{Syntax of the interface language}}
\label{sec:tutorialInterfaceSyntax}

\ifjp%>>>>>>>>>>>>>>>>>>>>>>>>>>>>>>>>>>>>>>>>>>>>>>>>>>>>>>>>>>>>>>>>>>>
	インタフェイス言語の文法の概要を以下に示します.
	\nonterm{conbind}などの定義が無い非終端記号は,The Definition of
Standard MLの同名のものと同一です.
	詳細な定義は(近々完成予定の)\smlsharp{}言語レファレンスマニュ
アルを参照ください.
\else%%%%%%%%%%%%%%%%%%%%%%%%%%%%%%%%%%%%%%%%%%%%%%%%%%%%%%%%%%%%%%%%%%%%
	This section outline the syntax of the interface language.
	Those non terminals such as \nonterm{conbind} that are not
defined here are the same as those in the Definition of Standard ML.
\fi%%%%<<<<<<<<<<<<<<<<<<<<<<<<<<<<<<<<<<<<<<<<<<<<<<<<<<<<<<<<<<<<<<<<<<

\[
\begin{array}{rcll}
\nonterm{interface} &::=& 
\nonterm{requireDeclList}\mbox{\ \ }\nonterm{provideDeclList} & \mbox{(interface)}
\\[1ex]
\nonterm{requireDeclList} &::=& \\
&|& \code{\_require}\sep\nonterm{smiFilePath}\vbar \nonterm{requireDeclList}\\
&|& \code{\_require}\sep\nonterm{sigFilePath}\vbar \nonterm{requireDeclList}\\
\\[1ex]
\nonterm{provideDeclList} &::=& \\
&|& \nonterm{provideDecl}\vbar \nonterm{requireDeclList}
\\[1ex]
\nonterm{provideDecl} &::=& \nonterm{idecl} \vbar \nonterm{ifundecl}
\\[1ex]
\nonterm{idecl} &::=& \code{val}\sep\nonterm{valdesc}\\
&|& \code{type}\sep\nonterm{itypbind}\\
&|& \code{eqtype}\sep\nonterm{tyvarSeq}\sep\nonterm{id}\code{(= }\nonterm{ty}\code{)}\\
&|& \code{datatype}\sep\nonterm{idatbind}\\
&|& \code{datatype}\sep\nonterm{id}\code{ = datatype}\sep\nonterm{longid}\\
&|& \code{exception}\sep\nonterm{exbind}\\
&|& \code{structure}\sep\nonterm{id}\code{ = }\nonterm{istrexp}\\
&|& \code{infix}\sep\nonterm{n}\sep\nonterm{id}\\
&|& \code{infixr}\sep\nonterm{n}\sep\nonterm{id}\\
&|& \code{noninfix}\sep\nonterm{n}\sep\nonterm{id}
\\[1ex]
\nonterm{itypbind} &::=& \nonterm{tyvarSeq}\sep\nonterm{id}\code{ = }\nonterm{ty}\\
        &|& \nonterm{tyvarSeq}\sep\nonterm{id} \code{( = }\nonterm{rep} \code{)}
\\[1ex]
\nonterm{rep} &::=& \code{boxed}
\ |\ \code{char}
\ |\ \code{codeptr}
\ |\ \code{contag}
\ |\ \code{int}
\ |\ \code{intInf}
\ |\ \code{ptr}
\\&&
\ |\ \code{real}
\ |\ \code{real32}
\ |\ \code{unit}
\ |\ \code{word}
\ |\ \code{word8}
\\[1ex]
\nonterm{idatbind} &::=& \nonterm{tyvarSeq}\sep\nonterm{id} \code{ = }
    \nonterm{conbinds}\sep $[$\sep\code{and}\sep\nonterm{idatbind} $]$
\\
 &|& \nonterm{tyvarSeq}\sep\nonterm{id}\sep
      \code{( = }\sep\nonterm{conbinds}\sep $[$\sep\code{and}\sep\nonterm{idatbind} $]$ \code{)}
\\[1ex]
\nonterm{istrexp} 
 &::=& \code{struct}\sep\nonterm{ideclSeq}\sep\code{end}
\\[1ex]
\nonterm{ifundec}
 &::=& \code{functor}\sep\nonterm{id}\sep\code{(}\nonterm{id}\code{ : }\nonterm{sigexp}\code{ )  = }\nonterm{istrexp}
\end{array}
\]

%   _require <interface file name>
% 
% ii. a list of provide specifications.
% Its syntax is defined roughly as follows.
% 
%    <itopdec> ::= 
% 
% Nonterminals not defined are analogous to those of Standard ML
% signature.
	


